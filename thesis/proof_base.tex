
  \begin{proof}

    This is proven by induction on the typing derivation of $t$.
    Unless otherwise specified, the type of $s$ and $s_D$ are respectively $R \rightarrow \llbracket \Gamma \rrbracket$ and $R \rightarrow \llbracket \D(\Gamma) \rrbracket$.
    \begin{enumerate}
      \item (\<var>)

        Prove: $S_\tau(\llbracket var\ v \rrbracket \circ s, \llbracket \D(var\ v) \rrbracket \circ s_D)$.

        Proceed by induction on the type evidence $v$.
        \begin{itemize}
          \item(\<Top>) Base case

          Prove: $S_\tau(\llbracket var\ Top \rrbracket \circ s, \llbracket \D(var\ Top) \rrbracket \circ s_D)$, where $s : R \rightarrow \llbracket \tau :: \Gamma \rrbracket$ and $s_D : R \rightarrow \llbracket \tau :: \Gamma \rrbracket$

          In this case the referenced $\tau$ exists at the top of the list.
          So both $\llbracket var\ Top \rrbracket$ and $\llbracket \D(var\ Top) \rrbracket$ denotate to fetching the top term.
          This is now proven by definition of $inst$, which states that the the term is semantically well-typed.

          \begin{align*}
            S&_\tau(\llbracket var\ Top \rrbracket \circ s, \llbracket \D(var\ Top) \rrbracket \circ s_D) \\
            &\Vdash \text{(Definition of $\D$)} \\
            & S_\tau(\llbracket var\ Top \rrbracket \circ s, \llbracket var\ Top \rrbracket \circ s_D) \\
            &\Vdash \text{(Definition of $\circ$)} \\
            & S_\tau(\lambda x. \llbracket var\ Top \rrbracket (s(x)), \lambda x. \llbracket var\ Top \rrbracket (s_D(x))) \\
            &\Vdash \text{(Definition of $\llbracket\rrbracket$)} \\
            & S_\tau(\lambda x. lookup\ \llbracket Top \rrbracket (s(x)), \lambda x. lookup\ \llbracket Top \rrbracket (s_D(x))) \\
            &\Vdash \text{(Rewrite using $s = \lambda x. hd(s(x))::tl(s(x))$)} \\
            & S_\tau(\lambda x. lookup\ \llbracket Top \rrbracket (hd(s(x))::tl(s(x))), \\
              & \;\;\; \lambda x. lookup\ \llbracket Top \rrbracket (hd(s_D(x))::tl(s_D(x)))) \\
            & \Vdash \text{(Simplify with lookup and $\llbracket Top \rrbracket$)} \\
            & S_\tau(\lambda x. hd(s(x)), \lambda x. hd(s_D(x))) \\
            & \Vdash \text{(By definition of $inst_{\tau::\Gamma}$)} \\
          \end{align*} \qed

          \item(\<Pop>) Induction step

          Prove: $S_\tau(\llbracket var\ (Pop\ v) \rrbracket \circ s, \llbracket \D(var\ (Pop\ v)) \rrbracket \circ s_D)$, where $s : R \rightarrow \llbracket \sigma :: \Gamma \rrbracket$ and $s_D : R \rightarrow \llbracket \D(\sigma :: \Gamma) \rrbracket$.

          Induction hypothesis:
          \begin{enumerate}\label{eqn:subst_ih_var_Pop}
            \item $\forall (f : R \rightarrow \llbracket \Gamma \rrbracket), (g : R \rightarrow \llbracket \D(\Gamma) \rrbracket). \\
            \;\;\;S_\tau(\llbracket var\ v \rrbracket \circ f, \llbracket \D(var\ v) \rrbracket \circ g)$
          \end{enumerate}

          Note that the \<var> term now denotates to ignoring the arbitrary unrelated type $\sigma$ and looking up $v$ in the rest of the list $\Gamma$.
          So $S_\tau(\llbracket var\ v \rrbracket \circ tl \circ s, \llbracket \D(var\ v) \rrbracket \circ tl \circ s_D)$, which is proven using the induction hypothesis by respectively instantiating $f$ and $g$ as $tl \circ s$ and $tl \circ s_D$.

          \begin{align*}
            S&_\tau(\llbracket var\ (Pop\ v) \rrbracket \circ s, \llbracket \D(var\ (Pop\ v)) \rrbracket \circ s_D) \\
            &\Vdash \text{(Definition of $\D$)} \\
            & S_\tau(\llbracket var\ (Pop\ v) \rrbracket \circ s, \llbracket var\ (Pop\ v) \rrbracket \circ s_D) \\
            &\Vdash \text{(Definition of $\circ$)} \\
            & S_\tau(\lambda x. \llbracket var\ (Pop\ v) \rrbracket (s(x)), \lambda x. \llbracket var\ (Pop\ v) \rrbracket (s_D(x))) \\
            &\Vdash \text{(Definition of $\llbracket\rrbracket$)} \\
            & S_\tau(\lambda x. lookup\ \llbracket Pop\ v \rrbracket (s(x)), \lambda x. lookup\ \llbracket Pop\ v \rrbracket (s_D(x))) \\
            &\Vdash \text{(Rewrite using $s = \lambda x. hd(s(x))::tl(s(x))$)} \\
            & S_\tau(\lambda x. lookup\ \llbracket Pop\ v \rrbracket (hd(s(x))::tl(s(x))), \\
              & \;\;\; \lambda x. lookup\ \llbracket Pop\ v \rrbracket (hd(s_D(x))::tl(s_D(x)))) \\
            & \Vdash \text{(Simplify with lookup and $\llbracket Pop\ v \rrbracket$)} \\
            & S_\tau(\lambda x. lookup\ \llbracket v \rrbracket (tl(s(x))), \lambda x. lookup\ \llbracket v \rrbracket (tl(s_D(x)))) \\
            & \Vdash \text{(Use IH. \ref{eqn:subst_ih_var_Pop} with $f = tl(s(x))$ and $g = tl(s_D(x))$)}
          \end{align*} \qed
        \end{itemize}
      \item (\<app>)

        Prove: $S_\tau(\llbracket app\ t_1\ t_2 \rrbracket \circ s, \llbracket \D(app\ t_1\ t_2) \rrbracket \circ s_D)$

        Induction hypotheses:
        \begin{enumerate}
          \item \label{eqn:subst_ih_app1}$S_{\sigma\rightarrow\tau}(\llbracket t_1 \rrbracket \circ s, \llbracket \D(t_1) \rrbracket \circ s_D)$
          \item \label{eqn:subst_ih_app2}$S_{\sigma}(\llbracket t_2 \rrbracket \circ s, \llbracket \D(t_2) \rrbracket \circ s_D)$
        \end{enumerate}

        First it is useful to rewrite the induction hypothesis \ref{eqn:subst_ih_app1} in a more usable format. Rewrite the statement using the definition of $S$ at function types.

        \begin{align*}
          S&_{\sigma\rightarrow\tau}(\llbracket t_1 \rrbracket \circ s, \llbracket \D(t_1) \rrbracket \circ s_D) \\
            & \Vdash \text{(Definition of \circ)} \\
            & S_{\sigma\rightarrow\tau}(\lambda x. \llbracket t_1 \rrbracket(s(x)), \lambda x. \llbracket \D(t_1) \rrbracket(s_D(x))) \\
            & \Vdash \text{(Definition of $S_{\rightarrow}$)} \\
            & \forall f_1, f_2.
              S_{\sigma}(f1, f2) \rightarrow \\
            &S_\tau(\lambda x. (\llbracket t_1 \rrbracket(s(x)))(f_1(x)), \lambda x. (\llbracket \D(t_1) \rrbracket(s_D(x)))(f_2(x)))
        \end{align*}

        The case for \<app> is now proven by applying the induction hypothesis \ref{eqn:subst_ih_app1} for the function term using the induction hypothesis \ref{eqn:subst_ih_app2} for the argument term to satify its premise.

        \begin{align*}
          S&_\tau(\llbracket app\ t_1\ t_2 \rrbracket \circ s, \llbracket \D(app\ t_1\ t_2) \rrbracket \circ s_D) \\
            &\Vdash \text{(Definition of $\D$)}\\
            & S_\tau(\llbracket app\ t_1\ t_2 \rrbracket \circ s, \llbracket app\ \D(t_1)\ \D(t_2) \rrbracket \circ s_D) \\
            &\Vdash \text{(Definition of \circ)}\\
            & S_\tau(\lambda x. \llbracket app\ t_1\ t_2 \rrbracket (s (x)), \lambda x. \llbracket app\ \D(t_1)\ \D(t_2) \rrbracket (s_D (x))) \\
            &\Vdash \text{(Definition of $\llbracket \rrbracket$)}\\
            & S_\tau(\lambda x. (\llbracket t_1\ \rrbracket(s(x))) (\llbracket t_2 \rrbracket(s(x))),\lambda x. (\llbracket \D(t_1)\ \rrbracket(s_D(x))) (\llbracket \D(t_2) \rrbracket(s_D(x))) \\
            &\Vdash \text{(Induction hypothesis \ref{eqn:subst_ih_app1})}\\
            & S_{\sigma}(\lambda x. \llbracket t_2 \rrbracket (s(x)), \lambda x. \llbracket \D(t_2) \rrbracket \circ (s_D(x))) \\
            &\Vdash \text{(Induction hypothesis \ref{eqn:subst_ih_app2})}
        \end{align*} \qed
      \item (\<abs>)

        Prove: $S_{\sigma\rightarrow\tau}(\llbracket abs\ t \rrbracket \circ s, \llbracket \D(abs\ t) \rrbracket \circ s_D)$

        Induction hypothesis:
        \begin{enumerate}
          \item \label{eqn:subst_ih_abs} $S_\sigma(\llbracket t \rrbracket \circ s, \llbracket \D(t) \rrbracket \circ s_D)$, where $s : R \rightarrow \llbracket \sigma::\Gamma \rrbracket$ and $s_D : R \rightarrow \llbracket \sigma::\Gamma \rrbracket$
        \end{enumerate}

        As is the case for \ref{eqn:subst_ih_app1}, simplify the goal statement using the definition of $S_\rightarrow$. So the proof obligation now becomes.

        Prove: $S_{\tau}(\lambda x. (\llbracket abs\ t \rrbracket (s(x)))(f_1(x)), \lambda x. (\llbracket \D(abs\ t) \rrbracket (s_D(x)))(f_2(x)))$

        Assume:
        \begin{enumerate}
          \item $f_1 : R \rightarrow \llbracket \sigma \rrbracket$
          \item $f_2 : R \rightarrow \llbracket \D(\sigma) \rrbracket$
          \item \label{eqn:subst_ass_abs3} $S_\sigma(f_1, f_2)$
        \end{enumerate}

        The proof proceeds by rewriting the goal until we can apply the induction hypothesis.
        Note that the assumption \ref{eqn:subst_ass_abs3}: $S_\sigma(f_1, f_2)$ ensures that the requirement of $inst_{\sigma::\Gamma}$ in the induction hypothesis \ref{eqn:subst_ih_abs} is satisfied.

        \begin{align*}
          S&_{\tau}(\lambda x. (\llbracket abs\ t \rrbracket (s(x)))(f_1(x)), \lambda x. (\llbracket \D(abs\ t) \rrbracket (s_D(x)))(f_2(x))) \\
            &\Vdash \text{(Definition of $\D$)}\\
            & S_{\tau}(\lambda x. (\llbracket abs\ t \rrbracket (s(x)))(f_1(x)), \lambda x. (\llbracket abs\ \D(t) \rrbracket (s_D(x)))(f_2(x))) \\
            &\Vdash \text{(Definition of $\llbracket \rrbracket$)}\\
            & S_{\tau}(\lambda x. (\llbracket t \rrbracket (f_1(x) :: s(x))), \lambda x. (\llbracket \D(t) \rrbracket (f_2(x) :: s_D(x)))) \\
            &\Vdash \text{(Induction hypothesis \ref{eqn:subst_ih_app1})}
        \end{align*} \qed

      \item (\<rval>)

      Prove: $S_{R}(\llbracket rval\ n \rrbracket \circ s, \llbracket \D(rval\ n) \rrbracket \circ s_D)$

      This is proven by noting that the corresponding denotations of \<rval> are constant functions, which are both smooth and whose derivatives are equal to $0$.

      \begin{align*}
        S&_R(\llbracket rval\ n \rrbracket \circ s, \llbracket \D(rval\ n) \rrbracket \circ s_D) \\
        &\Vdash \text{(Definition of $\D$)}\\
        &S_R(\llbracket rval\ n \rrbracket \circ s, \llbracket tuple\ (rval\ n)\ (rval\ 0) \rrbracket \circ s_D) \\
        &\Vdash \text{(Definition of $\llbracket\rrbracket$)}\\
        &S_R(const\ n, (const\ n, const\ 0)) \\
        &\Vdash \text{(Definition of $S_R$)}\\
        &smooth\ (const\ n) \wedge
          const\ 0 = \sfrac{\partial{const\ n}}{\partial{x}} \\
        &\Vdash \text{(split goals: goal 1)}\\
        &\;\;\;smooth\ (const\ n) \\
        &\;\;\;\Vdash \text{($f(x) = n$ is continuously differentiable)}\\
        &\Vdash \text{(split goals: goal 2)}\\
        &\;\;\;const\ 0 = \sfrac{\partial{const\ n}}{\partial{x}} \\
        &\;\;\;\Vdash \text{(if $f(x) = n$, then $\sfrac{\partial{f}}{\partial{x}} = 0$)}
      \end{align*} \qed
      \item (\<add>)

      Prove: $S_R(\llbracket add\ t_1\ t_2 \rrbracket \circ s, \llbracket \D(add\ t_1\ t_2) \rrbracket \circ s_D)$

      Induction hypotheses:
      \begin{enumerate}
        \item \label{eqn:subst_ih_add1}$S_R(\llbracket t_1 \rrbracket \circ s, \llbracket \D(t_1) \rrbracket \circ s_D)$
        \item \label{eqn:subst_ih_add2}$S_R(\llbracket t_2 \rrbracket \circ s, \llbracket \D(t_2) \rrbracket \circ s_D)$
      \end{enumerate}

      The proof proceeds by simplifying the denotations and proving the smoothness and derivative requirements for $S_R$.

      \begin{align*}
        S&_R(\llbracket add\ t_1\ t_2 \rrbracket \circ s, \llbracket \D(add\ t_1\ t_2) \rrbracket \circ s_D) \\
        &\Vdash \text{(Definition of $\D$)}\\
        &S_R(\llbracket add\ t_1\ t_2 \rrbracket \circ s, \llbracket tuple\ \\
        & \;\;\;(add\ (first\ \D(t_1)) (first\ \D(t_2)))\ \\
        & \;\;\;(add\ (second \D(t_1)) (second \D(t_2)))) \rrbracket \circ s_D) \\
        &\Vdash \text{(Definition of $\llbracket\rrbracket$, using} \\
        & \;\;\;\;\;\;\;\;\; \text{$(d_1, d_1') = \llbracket \D(t_1) \rrbracket s(x)$ and $(d_2, d_2') = \llbracket \D(t_2) \rrbracket s_D(x)$)}\\
        &S_R(\lambda x. d_1(x) + d_2(x), \lambda x. (d_1(x) + d_2(x), d_1'(x) + d_2'(x))) \\
        &\Vdash \text{(Definition of $S_R$)}\\
        & smooth\ (\lambda x. d_1(x) + d_2(x)) \wedge \\
        & \;\;\; \lambda x. d_1'(x) + d_2'(x) = \sfrac{\partial{(\lambda x. d_1'(x) + d_2'(x))}}{\partial{x}} \\
        &\Vdash \text{(split goals: goal 1)}\\
        &\;\;\;smooth\ (\lambda x. d_1(x) + d_2(x)) \\
        &\;\;\;\Vdash
          \text{(Addition is smooth, if subterms are smooth)}\\
        &\;\;\;smooth\ d_1 \wedge smooth\ d_2 \\
        &\;\;\;\Vdash \text{(Induction hypothesis \ref{eqn:subst_ih_add1} for $d_1$ and \ref{eqn:subst_ih_add2} for $d_2$)}\\
        &\Vdash \text{(split goals: goal 2)}\\
        &\;\;\;\lambda x. d_1'(x) + d_2'(x) = \sfrac{\partial{(\lambda x. d_1'(x) + d_2'(x))}}{\partial{x}} \\
        &\;\;\;\Vdash \text{(By definition of taking the derivative of addition)} \\
        &\;\;\; d_1' = \sfrac{\partial{d_1}}{\partial{x}} \wedge d_2' = \sfrac{\partial{d_2}}{\partial{x}} \\
        &\;\;\;\Vdash \text{(Induction hypothesis \ref{eqn:subst_ih_add1} for $d_1$ and \ref{eqn:subst_ih_add2} for $d_2$)}\\
      \end{align*} \qed

      \item (\<mul>)

      Prove: $S_R(\llbracket mul\ t_1\ t_2 \rrbracket \circ s, \llbracket \D(mul\ t_1\ t_2) \rrbracket \circ s_D)$

      Induction hypotheses:
      \begin{enumerate}
        \item \label{eqn:subst_ih_mul1}$S_R(\llbracket t_1 \rrbracket \circ s, \llbracket \D(t_1) \rrbracket \circ s_D)$
        \item \label{eqn:subst_ih_mul2}$S_R(\llbracket t_2 \rrbracket \circ s, \llbracket \D(t_2) \rrbracket \circ s_D)$
      \end{enumerate}

      Proof goes through almost identically as for the case for \<add>.

      \begin{align*}
        S&_R(\llbracket mul\ t_1\ t_2 \rrbracket \circ s, \llbracket \D(mul\ t_1\ t_2) \rrbracket \circ s_D) \\
        &\Vdash \text{(Definition of $\D$)}\\
        &S_R(\llbracket mul\ t_1\ t_2 \rrbracket \circ s, \llbracket tuple\ \\
        & \;\;\;(mul\ (first\ \D(t_1)) (first\ \D(t_2)))\ \\
        & \;\;\;(add\ \\
        & \;\;\;\;\;(mul\ (first \D(t_1)) (second \D(t_2))) \\
        & \;\;\;\;\;(mul\ (first \D(t_2)) (second \D(t_1)))) \rrbracket \circ s_D) \\
        &\Vdash \text{(Definition of $\llbracket\rrbracket$, using} \\
        & \;\;\;\;\;\;\;\;\; \text{$(d_1, d_1') = \llbracket \D(t_1) \rrbracket s(x)$ and $(d_2, d_2') = \llbracket \D(t_2) \rrbracket s_D(x)$)}\\
        &S_R(\lambda x. d_1(x) * d_2(x), \\
        & \;\;\; \lambda x. (d_1(x) * d_2(x), d_1(x) * d_2'(x) + (d_2(x) * d_1'(x)))) \\
        &\Vdash \text{(Definition of $S_R$)}\\
        &smooth\ (\lambda x. d_1(x) * d_2(x)) \wedge \\
        & \;\;\; \lambda x. d_1(x) * d_2'(x) + d_2(x) * d_1'(x) = \sfrac{\partial{(\lambda x. (d_1(x) * d_2(x))}}{\partial{x}} \\
        &\Vdash \text{(split goals: goal 1)}\\
        &\;\;\;smooth\ (\lambda x. d_1(x) * d_2(x)) \\
        &\;\;\;\Vdash
          \text{(Multiplication is smooth, if subterms are smooth)}\\
        &\;\;\;smooth\ d_1 \wedge smooth\ d_2 \\
        &\;\;\;\Vdash \text{(Induction hypothesis \ref{eqn:subst_ih_mul1} for $d_1$ and \ref{eqn:subst_ih_mul2} for $d_2$)}\\
        &\Vdash \text{(split goals: goal 2)}\\
        &\;\;\;\lambda x. d_1(x) * d_2'(x) + d_2(x) * d_1'(x) = \sfrac{\partial{(\lambda x. (d_1(x) * d_2(x))}}{\partial{x}} \\
        &\;\;\;\Vdash \text{(By definition of taking the derivative of multiplications)} \\
        &\;\;\; d_1' = \sfrac{\partial{d_1}}{\partial{x}} \wedge d_2' = \sfrac{\partial{d_2}}{\partial{x}} \\
        &\;\;\;\Vdash \text{(Induction hypothesis \ref{eqn:subst_ih_mul1} for $d_1$ and \ref{eqn:subst_ih_mul2} for $d_2$)}\\
      \end{align*} \qed

      \item (\<tuple>)

      Prove: $S_(\tau \times \sigma)(\llbracket tuple\ t_1\ t_2 \rrbracket \circ s, \llbracket \D(tuple\ t_1\ t_2) \rrbracket \circ s_D)$

      Induction hypotheses:
      \begin{enumerate}
        \item \label{eqn:subst_ih_tuple1}$S_\tau(\llbracket t_1 \rrbracket \circ s, \llbracket \D(t_1) \rrbracket \circ s_D)$
        \item \label{eqn:subst_ih_tuple2}$S_\sigma(\llbracket t_2 \rrbracket \circ s, \llbracket \D(t_2) \rrbracket \circ s_D)$
      \end{enumerate}

      A recurring pattern will become apparent in later sections when continuing to prove the substitution lemma \ref{thm:substitution_lemma} for types consisting of other types.
      In this case, due to the carefull attention spent on the logical relation, only the witnesses of the subterms of the tuple need to be supplied to finish the proof.

      Note that the witnesses of $S_\tau$ and $S_\sigma$ that need to be given here are supplied by the induction hypotheses.
      While these witnesses are not exactly relevant to finish this proof for \<tuple>, they are needed in the proofs for projections.

      \begin{align*}
        S&_(\tau \times \sigma)(\llbracket tuple\ t_1\ t_2 \rrbracket \circ s, \llbracket \D(tuple\ t_1\ t_2) \rrbracket \circ s_D) \\
        & \Vdash \text{(Definition of $\D$)} \\
        & S_(\tau \times \sigma)(\llbracket tuple\ t_1\ t_2 \rrbracket \circ s, \llbracket tuple\ \D(t_1)\ \D(t_2)) \rrbracket \circ s_D) \\
        & \Vdash \text{(Definition of $\llbracket\rrbracket$)} \\
        & S_(\tau \times \sigma)(\lambda x. (\llbracket t_1 \rrbracket(s(x)), \llbracket t_2 \rrbracket(s(x))), \\
        & \;\;\;\;\;\;\lambda x. (\llbracket \D(t_1) \rrbracket(s'(x)), \llbracket \D(t_2) \rrbracket(s'(x)))) \\
        & \Vdash \text{(Definition of $S_{\tau\times\sigma}$)} \\
        & \exists f_1, f_2, g_1, g_2, \\
            & \;\;\;\;S_\tau(f_1, f_2), S_\sigma(g_1, g_2). \\
            & \;\;\;\;\lambda x. (\llbracket t_1 \rrbracket(s(x)), \llbracket t_2 \rrbracket(s(x))) = \lambda x. (f_1(x), g_1(x)) \wedge \\
            & \;\;\;\;\lambda x. (\llbracket \D(t_1) \rrbracket(s'(x)), \llbracket \D(t_2) \rrbracket(s'(x))) = \lambda x. (f_2(x), g_2(x)) \\
        & \Vdash \text{(Give witnesses: $f_1 := \llbracket t_1 \rrbracket \circ s$, $f_2 := \llbracket t_2 \rrbracket \circ s$,} \\
        & \;\;\;\;\;\; \text{$g_1 := \llbracket \D(t_1) \rrbracket \circ s'$, $g_2 := \llbracket \D(t_2) \rrbracket \circ s'$)} \\
        & \exists S_\tau(f_1, f_2), S_\sigma(g_1, g_2). \\
          & \;\;\;\;\lambda x. (\llbracket t_1 \rrbracket(s(x)), \llbracket t_2 \rrbracket(s(x))) \\
          & \;\;\;\;\;\;\; = \lambda x. (\llbracket t_1 \rrbracket(s(x)), \llbracket t_2 \rrbracket(s(x))) \wedge \\
          & \;\;\;\;\lambda x. (\llbracket \D(t_1) \rrbracket(s'(x)), \llbracket \D(t_2) \rrbracket(s'(x))) \\
          & \;\;\;\;\;\;\; = \lambda x. (\llbracket \D(t_1) \rrbracket(s'(x)), \llbracket \D(t_2) \rrbracket(s'(x))) \\
        & \Vdash \text{(Give witnesses of $S_\tau$ and $S_\sigma$ using respective IHs \ref{eqn:subst_ih_tuple1} and \ref{eqn:subst_ih_tuple2})} \\
        & \;\;\;\;\lambda x. (\llbracket t_1 \rrbracket(s(x)), \llbracket t_2 \rrbracket(s(x))) \\
        & \;\;\;\;\;\;\; = \lambda x. (\llbracket t_1 \rrbracket(s(x)), \llbracket t_2 \rrbracket(s(x))) \wedge \\
        & \;\;\;\;\lambda x. (\llbracket \D(t_1) \rrbracket(s'(x)), \llbracket \D(t_2) \rrbracket(s'(x))) \\
        & \;\;\;\;\;\;\; = \lambda x. (\llbracket \D(t_1) \rrbracket(s'(x)), \llbracket \D(t_2) \rrbracket(s'(x))) \\
        & \Vdash \text{(Reflexivity)} \\
      \end{align*}\qed
      \item (\<first>)

      Prove: $S_(\tau)(\llbracket first\ t \rrbracket \circ s, \llbracket \D(first\ t) \rrbracket \circ s_D)$

      Induction hypotheses:
      \begin{enumerate}
        \item \label{eqn:subst_ih_first}$S_{\tau\times\sigma}(\llbracket t \rrbracket \circ s, \llbracket \D(t) \rrbracket \circ s_D)$
      \end{enumerate}

      Simplifying the induction hypothesis \ref{eqn:subst_ih_first} using the definition of $S_{\tau\times\sigma}$ gives rise to a number of useful assumptions containing:
      $f_1 : R \rightarrow \llbracket \tau \rrbracket$
      , $f_2 : R \rightarrow \llbracket \D(\tau) \rrbracket$
      , $g_1 : R \rightarrow \llbracket \sigma \rrbracket$
      and $g_2 : R \rightarrow \llbracket \D(\sigma) \rrbracket$.

      Assumptions:
      \begin{enumerate}
        \item \label{eqn:subst_ass_proj1_4} $S_\tau(f_1, f_2)$
        \item \label{eqn:subst_ass_proj1_5} $S_\sigma(g_1, g_2)$
        \item \label{eqn:subst_ass_proj1_6} $\llbracket t \rrbracket \circ s = \lambda x. (f_1(x), g_1(x))$
        \item \label{eqn:subst_ass_proj1_7} $\llbracket \D(t) \rrbracket \circ s = \lambda x. (f_2(x), g_2(x))$
      \end{enumerate}

      \begin{align*}
        S&_{\tau}(\llbracket first\ t \rrbracket \circ s, \llbracket \D(first\ t) \rrbracket \circ s_D) \\
        & \Vdash \text{(Rewrite using definition of $\D$)} \\
        & S_{\tau}(\llbracket first\ t \rrbracket \circ s, \llbracket first\ \D(t) \rrbracket \circ s_D) \\
        & \Vdash \text{(Rewrite using definition of $\llbracket\rrbracket$)} \\
        & S_{\tau}(\lambda x. fst(\llbracket t \rrbracket(s(x))), \lambda x. fst(\llbracket \D(t) \rrbracket(s_D(x)))) \\
        & \Vdash \text{(Rewrite using \ref{eqn:subst_ass_proj1_6} and \ref{eqn:subst_ass_proj1_7})} \\
        & S_{\tau}(\lambda x. fst(f_1(x), g_1(x)), \lambda x. fst(f_2(x), g_2(x))) \\
        & \Vdash \text{($\beta\eta$-equality)} \\
        & S_{\tau}(f_1, f_2) \\
        & \Vdash \text{(Assumption \ref{eqn:subst_ass_proj1_4})} \\
      \end{align*} \qed

      \item (\<second>)

      Prove: $S_(\tau)(\llbracket first\ t \rrbracket \circ s, \llbracket \D(first\ t) \rrbracket \circ s_D)$

      Induction hypotheses:
      \begin{enumerate}
        \item \label{eqn:subst_ih_first}$S_{\tau\times\sigma}(\llbracket t \rrbracket \circ s, \llbracket \D(t) \rrbracket \circ s_D)$
      \end{enumerate}

      Proof goes through the same as the case for \<first> with the same assumptions following from the induction hypothesis, where
      $f_1 : R \rightarrow \llbracket \tau \rrbracket$
      , $f_2 : R \rightarrow \llbracket \D(\tau) \rrbracket$
      , $g_1 : R \rightarrow \llbracket \sigma \rrbracket$
      and $g_2 : R \rightarrow \llbracket \D(\sigma) \rrbracket$.

      Assumptions:
      \begin{enumerate}
        \item \label{eqn:subst_ass_proj2_4} $S_\tau(f_1, f_2)$
        \item \label{eqn:subst_ass_proj2_5} $S_\sigma(g_1, g_2)$
        \item \label{eqn:subst_ass_proj2_6} $\llbracket t \rrbracket \circ s = \lambda x. (f_1(x), g_1(x))$
        \item \label{eqn:subst_ass_proj2_7} $\llbracket \D(t) \rrbracket \circ s = \lambda x. (f_2(x), g_2(x))$
      \end{enumerate}

      \begin{align*}
        S&_{\sigma}(\llbracket second\ t \rrbracket \circ s, \llbracket \D(second\ t) \rrbracket \circ s_D) \\
        & \Vdash \text{(Rewrite using definition of $\D$)} \\
        & S_{\sigma}(\llbracket second\ t \rrbracket \circ s, \llbracket second\ \D(t) \rrbracket \circ s_D) \\
        & \Vdash \text{(Rewrite using definition of $\llbracket\rrbracket$)} \\
        & S_{\sigma}(\lambda x. snd(\llbracket t \rrbracket(s(x))), \lambda x. snd(\llbracket \D(t) \rrbracket(s_D(x)))) \\
        & \Vdash \text{(Rewrite using \ref{eqn:subst_ass_proj2_6} and \ref{eqn:subst_ass_proj2_7})} \\
        & S_{\sigma}(\lambda x. snd(f_1(x), g_1(x)), \lambda x. snd(f_2(x), g_2(x))) \\
        & \Vdash \text{($\beta\eta$-equality)} \\
        & S_{\sigma}(f_1, f_2) \\
        & \Vdash \text{(Assumption \ref{eqn:subst_ass_proj2_4})} \\
      \end{align*} \qed
    \end{enumerate}
  \end{proof}

  The proof of the fundamental property of the logical relation now follows from the substitution lemma.

  \begin{lemma}[Fundamental property]\label{thm:fundamental_property}
    For any well-typed term $x_1 : R, \dots, x_n : R \vdash t : \tau$, and argument function $f : R \rightarrow \llbracket R^n \rrbracket$, such that each argument is continuously derivable, then $S_\tau(\llbracket t\rrbracket \circ f, \llbracket \D(t)\rrbracket \circ \D_n \circ f)$.
  \end{lemma}

  \begin{proof}
    This is proven by instantiating the substitution lemma \ref{thm:substitution_lemma} with the proper variables and proving the resulting judgement of $inst_{R^n}$ by induction on $n$.

    \begin{align*}
      S&_{\tau}(\llbracket t \rrbracket \circ f, \llbracket \D(t) \rrbracket \circ \D_n \circ f) \\
      & \Vdash \text{(Apply substitution lemma with $s := f$, $s_D := \D_n \circ f$ and $\Gamma := R^n$)} \\
      & inst_{R^n}(f, \D_n \circ f) \\
    \end{align*}

    Proceed by induction on $n$, intuitively building up the environment with denotation of terms such that they follow $S$.

    \begin{itemize}
      \item Base case: $n = 0$

      \begin{align*}
        inst&_{R^0}(f, \D_0 \circ f) \\
        & \Vdash \text{(Induction on n, base case $n = 0$)} \\
        & \;\;\; inst_{[]}(f, \D_0 \circ f) \\
        & \;\;\; \Vdash \text{(Singleton instance of $R \rightarrow R^0$, $f = const([])$)} \\
        & \;\;\; inst_{[]}(const([]), \D_0 \circ const([])) \\
        & \;\;\; \Vdash \text{(Definition of $\D_0$)} \\
        & \;\;\; inst_{[]}(const([]), const([])) \\
        & \;\;\; \Vdash \text{(Definition of $inst_{[]}$)}
      \end{align*}

      \item Induction case: $n = S(n')$

      Induction hypothesis: $inst_{R^{n'}}(f', D \circ f')$, where
        $f' : R \rightarrow \llbracket R^{n'}\rrbracket$.

      \begin{align*}
        inst&_{R :: R^{n'}}(f, \D_{R :: R^{n'}} \circ f) \\
        & \;\;\; \Vdash \text{(Unfold $\circ$)} \\
        & \;\;\; inst_{R :: R^{n'}}(\lambda x. f(x), \lambda x. \D_{R :: R^{n'}}( f, x)) \\
        & \;\;\; \Vdash \text{(Rewrite using $f = \lambda x. hd(f(x)) :: tl(f(x))$} \\
        & \;\;\;\;\;\;\;\;\;\;\;\; \text{and definition of $\D_{R :: R^{n'}}$)} \\
        & \;\;\; inst_{R :: R^{n'}}(\lambda x. hd(f(x)) :: tl(f(x)), \\
        & \;\;\;\;\;\;
          \lambda x. (hd(f(x)), \sfrac{\partial{(hd \circ f)}}{\partial{x}}(x)) :: \D_{R^{n'}}(tl \circ f, x)) \\
        & \;\;\; \Vdash \text{(By definition of $inst_{R :: R^{n'}}$, rest proven by IH)}
      \end{align*}
    \end{itemize}
  \end{proof}