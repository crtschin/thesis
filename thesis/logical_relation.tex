\subsection{Logical relations}

Logical relations is a technique often employed when proving programming language properties of statically typed languages\cite{skorstengaard2019introduction}. There are two main ways they are used, namely as unary and binary relations.
Unary logical relations, also known as logical predicates, are predicates over single terms and are typically used to prove language characteristics such as type safety or strong normalization.
Binary logical relations on the other hand are used to prove program equivalences, usually in the context of denotational semantics as we will do.
There have been many variations on the versatile technique from syntactic step-indexed relations which have been used to solve recursive types\cite{Ahmed2006}, to open relations which enable working with terms of non-ground type\cite{barthe2020versatility}\cite{huot2020correctness}.
Logical relations in essence are relations between terms defined by induction on their types.
A logical relations proof consists of 2 main steps.
The first states the terms for which the property is expected to hold are in the relation, while the second states that the property of interest follows from the relation.
The second step is easier to prove as it usually follows from the definition of the relation. The first on the other hand, will often require proving a generalized variant, called the fundamental property of the logical relation.
In most cases this requires that the relation is correct with respect to applying substitutions.

A well-known logical relations proof is the proof of strong normalization of well-typed terms, which states that all terms eventually terminate.
An example of a logical relation used in such a proof using the intrinsic strongly-typed formulation is given in Snippet~\ref{lst:sn_logical_relation}.
Noteworthy is the case for function types, where one needs to prove that applying a function preserves the strong normalization property.
If one were to attempt the proof of strong normalization without using logical relations, the proof would get stuck in the cases dealing with function types.
More specifically when applying a function term to an argument term which terminates, the induction hypothesis is not strong enough to prove that substituting the argument into the body of the abstraction results in a terminating term.

% TODO: Remove/Move this
% The proof given in the paper this thesis is based on, is a logical relations proof on the denotational semantics using diffeological spaces as its domains\cite{huot2020correctness}.
% A similar, independent proof of correctness was given by Barthe, et. al.\cite{barthe2020versatility} using a syntactic relation on the operational semantics.

\begin{listing}
  \begin{minted}{coq}
    Equations SN {~\Gamma~} ~\tau~ (t : ~tm \Gamma \tau~): Prop :=
    SN unit t := halts t;
    SN ~(\tau \Rightarrow \sigma)~ t := halts t ~$\wedge$~
      ~(\forall (s : tm \Gamma \tau), SN \tau s \rightarrow SN \sigma (app \Gamma \sigma \tau t s))~;
  \end{minted}
  \caption{Example of a logical predicate used in a strong normalizations proof in the intrinsic strongly-typed representation}
  \label{lst:sn_logical_relation}
\end{listing}
