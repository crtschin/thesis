\subsection{Program Transformations}
The transformation rules consist of several algebraic identities, along with compile time optimization techniques such as partial evaluation and deforestation or loop fusion.

\begin{figure}[]
  \centering
  \begin{subfigure}{0.48\textwidth}
    \centering
    \begin{align*}
      \add{t_1}{t_2} &\leadsto \add{t_2}{t_1} \\
      \add{0}{t} &\leadsto t \\
      \add{t}{(-t)} &\leadsto 0 \\
      \mul{t_1}{t_2} &\leadsto \mul{t_2}{t_1} \\
      \add{(\mul{t}{t_1})}{(&\mul{t}{t_2})} \\
        \leadsto \mul{t}{&(\add{t_1}{t_2})} \\
      \mul{0}{t} &\leadsto 0 \\
      \mul{1}{t} &\leadsto t
    \end{align*}
    \caption{Algebraic laws}
  \end{subfigure}
  \begin{subfigure}{0.48\textwidth}
    \begin{subfigure}{1\textwidth}
      \begin{align*}
        t \evals t' \to t \leadsto t'
      \end{align*}
      \caption{Reuse rewriting rules from operational semantics}
    \end{subfigure}
    \begin{subfigure}{1\textwidth}
      \begin{align*}
        t \leadsto t' \to \abs{t} \leadsto \abs{t'}
      \end{align*}
      \caption{Partial evaluation on functions}
    \end{subfigure}
    \begin{subfigure}{1\textwidth}
      \begin{align*}
        \get{i}{(\build{n}{f})}\leadsto f\ i
      \end{align*}
      \caption{Loop fusion}
    \end{subfigure}
  \end{subfigure}
  \begin{subfigure}{1\textwidth}
    \begin{align*}
      \ifold{
        &(\abs{(\abs{
          (\tuple
          {\\&\hspace{-0.5cm}(\app{(\app{f}{(\var{(\Pop{\Top}))}})}
              {(\first{(\var{\Top})})})}
          {\\&\hspace{-0.5cm}(\app{(\app{f}{(\var{(\Pop{\Top}))}})}
              {(\second{(\var{\Top})})})})})})}
        {i}{(\tuple{z_1}{z_2})} \\
        & \leadsto \tuple{(\ifold{f}{i}{z_1})}{(\ifold{f}{i}{z_2})}
    \end{align*}
    \caption{Loop fission}
  \end{subfigure}
  \caption{Included rewrite rules for the simply-typed lambda calculus extended with sum, product, number and array types}
  \label{fig:rewrite_rules}
\end{figure}

As a small deviation from the rules given by Shaikhha, et. al.\cite{Shaikha2019}, we explicitly included some simplification rules in the style of a natural semantics. Figures \ref{fig:rewrite_rules} and \ref{fig:natural_infer} show, respectively, the rewrite rules we included in our language and the inference rules we used for our simplification rules.

\begin{figure}
  \begin{mathpar}
    \inferrule*[Right=\textsc{EVAppAbs}]
      {t_1 \evals \abs{t_1'} \\
        t_2 \evals t_2'}
      {\app{t_1}{t_2} \evals \substitute{t_2'}{t_1'}} \and
    \inferrule*[Right=\textsc{EVSucc}]
      {t \evals \nval{n}}
      {\nsucc{t} \evals \nval{(n + 1)}} \\ \and
    \inferrule*[Right=\textsc{EVNRec0}]
      {t_1 \evals t_1' \\
        t_2 \evals \nval{0} \\
        t_3 \evals t_3'}
      {\nrec{t_1}{t_2}{t_3} \evals t_3} \\ \and
    \inferrule*[Right=\textsc{EVNRecS}]
      {t_1 \evals t_1' \\
        t_2 \evals \nval{(n + 1)} \\
        t_3 \evals t_3'}
      {\nrec{t_1}{t_2}{t_3} \evals \app{t_1'}{(\nrec{t_1'}{(\nval{n})}{t_3'})}} \\ \and
    \inferrule*[Right=\textsc{EVAdd}]
      {t_1 \evals \rval{r_1} \\
        t_2 \evals \rval{r_2}}
      {\add{t_1}{t_2} \evals \rval{r_1 + r_2}} \and
    \inferrule*[Right=\textsc{EVMult}]
      {t_1 \evals \rval{r_1} \\
        t_2 \evals \rval{r_2}}
      {\mul{t_1}{t_2} \evals \rval{r_1 * r_2}} \\ \and
      \inferrule*[Right=\textsc{EVTuple}]
        {t_1 \evals t_1' \\
          t_2 \evals t_2'}
        {\tuple{t_1}{t_2} \evals \tuple{t_1'}{t_2'}} \\ \and
      \inferrule*[Right=\textsc{EVFst}]
        {t \evals \tuple{t_1}{t_2}}
        {\first{t} \evals t_1} \and
      \inferrule*[Right=\textsc{EVSnd}]
        {t \evals \tuple{t_1}{t_2}}
        {\second{t} \evals t_2} \\ \and
      \inferrule*[Right=\textsc{EVInl}]
        {t \evals t'}
        {\inl{t} \evals \inl{t'}} \and
      \inferrule*[Right=\textsc{EVInr}]
        {t \evals t'}
        {\inr{t} \evals \inr{t'}} \\ \and
      \inferrule*[Right=\textsc{EVCaseInl}]
        {t \evals \inl{t'} \\
          t_1 \evals t_1' \\
          t_2 \evals t_2'}
        {\case{t}{t_1}{t_2} \evals \app{t_1'}{t'}} \\ \and
      \inferrule*[Right=\textsc{EVCaseInr}]
        {t \evals \inr{t'} \\
          t_1 \evals t_1' \\
          t_2 \evals t_2'}
        {\case{t}{t_1}{t_2} \evals \app{t_2'}{t'}}
  \end{mathpar}
  \caption{Inference rules for the evaluation relation}
  \label{fig:natural_infer}
\end{figure}

Before we can prove soundness of our rewrite rules, we have to prove soundness of our natural semantics.
We reuse the denotational semantics from Section~\ref{sec:forward} we used to prove the forward-mode algorithm correct.

\begin{lemma}[Soundness of natural semantics]\label{lem:natural_soundness}
  For any well-typed terms $t, t'$ such that $t \Downarrow t'$ holds, we have $\llbracket t \rrbracket = \llbracket t' \rrbracket$.
\end{lemma}
\begin{proof}
  This is proven by induction on the evaluation relation $\Downarrow$.
  All of the cases follow from the induction hypotheses after simplification.
\end{proof}

\begin{theorem}[Soundness of program transformations]
  For any well-typed terms $t, t'$ such that $t \leadsto t'$ holds, we have $\llbracket t \rrbracket = \llbracket t' \rrbracket$.
\end{theorem}
\begin{proof}
  This is proven by induction on the rewriting relation $\Downarrow$.
  Most of the cases follow from the induction hypothesis.
  The rewrite rule where we incorporated the evaluation relation is proven by Lemma~\ref{lem:natural_soundness}.
  The rewrite rules associated with algebraic identities are proven by exactly those identities after simplifying using the denotational semantics.

  For the loop fusion rule, we first do induction on $i$, the index being accessed.
  For $i = 0$, we use case-analysis on $n$ along with simple rewriting to prove the goal. The induction step is proven by the induction hypothesis.
  Similarly for the loop fission rule, we have to do induction on the denotation of the term used to encode the number of iterations.
  The base case is trivial and the induction step is proven by the induction hypothesis.
\end{proof}