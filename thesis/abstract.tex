\begin{abstract}
  Automatic differentiation is a well-known concept, which has been gaining traction is recent years due to its heavy usage in many versatile applications.
  Exposing the decades-worth of programming language history to this technique may bear fruit with improving both performance, correctness and modularity of such code.
  While pen-and-paper proofs of correctness do exist for these techniques in the context of functional languages, a formal proof has been absent up till now.
  In this research, we give a formalized proof of correctness of both a ubiquitous forward-mode and a continuation-based pseudo-reverse-mode automatic differentiation algorithms.
  We repeatedly do this using a logical relations argument accompanied with simple but effective language representations and denotations.
  We also discuss and prove sound various program transformations, which lets forward-mode automatic differentiation approach the performance of reverse-mode.
  We also make preliminary steps towards a formalized proof of correctness of a real combinator-based reverse-mode algorithm.
\end{abstract}
