\documentclass[11pt, final]{article}
\usepackage{mystyle}

\addbibresource{./references.bib}

\declaretheorem{definition}
\declaretheorem{example}
\declaretheorem{theorem}
\declaretheorem{lemma}
\declaretheorem{corrollary}
\renewcommand{\headrulewidth}{0.5pt}
\renewcommand{\footrulewidth}{0.5pt}
\renewcommand{\listingscaption}{Code snippet}
\newcommand{\comment}[1]{}

\def\Vakar{V\'{a}k\'{a}r}
\def\D{\overrightarrow{\mathcal{D}}}
\def\lambdaBase{\Lambda_{\delta}^{\times, \rightarrow, \mathds{R}}}
\def\<#1>{\csname keyword@@#1\endcsname}
\begingroup
\makeatletter
\def\do#1{\expandafter\doaux\expandafter{\keyword@style{#1}}{#1}}
\def\doaux#1#2{\global\@namedef{keyword@@#2}{#1}}
\def\keyword@style#1{\textbf{#1}}
\do{Coq}
\do{Agda}
\def\keyword@style#1{\mathds{#1}}
\do{R}
\def\keyword@style#1{\mathcal{#1}}
\do{D}
\def\keyword@style#1{\texttt{#1}}
\do{bottom}
\do{Equations}
\do{Coquelicot}
\do{Program}
\do{Set}
\do{Prop}
\do{Type}
\do{return}
\do{simpl}
\do{Either}
\do{sb}
\do{Dsb}
\do{Reals}
\do{var}
\do{abs}
\do{app}
\do{tuple}
\do{first}
\do{second}
\do{rval}
\do{add}
\do{mul}
\do{Top}
\do{Pop}
\endgroup


\setlength{\headheight}{15pt}
\pagestyle{fancy}
\lhead{Utrecht University}
\rfoot{\thepage}
\cfoot{ }
\allowdisplaybreaks

\begin{document}

\begin{titlepage}
  \pagenumbering{gobble}

  \center
  \textsc{\Huge Utrecht University}

  \vspace{0.3in}
  \begin{figure}[H]
    \centering
    \includegraphics[scale=0.09]{assets/UU_logo.png}
  \end{figure}

  \thispagestyle{fancy}

  \center
  \textsc{\Large Master's Thesis}

  \vspace{0.5in}

  \noindent\makebox[\linewidth]{\rule{\linewidth}{1.2pt}}
  \vspace{0.03in}

  \begin{doublespace}
  \textsc{\textbf{\huge Formalized Correctness Proofs of Automatic Differentiation in Coq}}
  \end{doublespace}

  \vspace{0.03in}
  \noindent\makebox[\linewidth]{\rule{\linewidth}{1.2pt}}

  \vspace{0.5in}

  \begin{minipage}{0.48\textwidth}
    \begin{flushleft}
      \textit{Author:} \\
      Curtis Chin Jen Sem \\
      \textit{5601118}
      % crtschin@gmail.com
    \end{flushleft}
  \end{minipage}
  \begin{minipage}{0.48\textwidth}
    \begin{flushright}
    \textit{Supervisors:} \\
    Matthijs \Vakar \\
    Wouter Swierstra \\
    \end{flushright}
  \end{minipage}

  \vspace{0.8in}

  \textit{A thesis submitted in partial fulfillment of the\\ requirements for the degree of}\\
  \vspace{0.1in}
  \textit{\Large Master of Science in Computing Science}\\
  \vspace{1.2in}
  \text{\large Department of Information and Computing Science} \\
  \textit{\today}
  \thispagestyle{empty}
\end{titlepage}

\newpage

\begin{abstract}
  In this thesis, we give a formalized proof of correctness of both a ubiquitous forward-mode and a continuation-based pseudo-reverse-mode automatic differentiation algorithm.
  We repeatedly do this using logical relations arguments accompanied by simple but effective language representations and denotational semantics.
  We also discuss and prove sound various program transformations, which in the context of efficient code generation for automatic differentiation, let forward-mode approach the performance of reverse-mode algorithms.
  Finally, we make preliminary steps towards a formalized proof of correctness of a real combinator-based reverse-mode algorithm.
\end{abstract}

\newpage

\pagenumbering{arabic}
\setcounter{page}{3}
\tableofcontents
\newpage

% \section{Introduction}
% TODO: Work through MV feedback
In machine learning and neural network research, problems are automatically almost magically solved using just input and output data.
This happens through an iterative optimization process of carefully designed functions.
One of the algorithms facilitating this process is called backpropagation.
Backpropagation directly corresponds to reverse-mode automatic differentiation, which, in most cases, is the most efficient method to compute the derivatives of a function, critical in optimization problems.
But programming in an environment that allows for automatic differentiation can be limiting.

Frameworks such as Tangent\fancyfootnote{https://github.com/google/tangent} or autograd\fancyfootnote{https://github.com/HIPS/autograd} are define-by-run algorithms, whose main tactic is to build up the derivative calculation dynamically during runtime.
This process can restrict which high-level optimizations one can apply to generated code.
Support for higher-order derivatives is also limited.

Programming language research has a rich history, with many well-known both high and low-level optimization techniques such as partial evaluation and deforestation.
Exposing these optimization techniques to the world of automatic differentiation can be very fruitful as these calculations are very expensive and often require significant computing power to run.
Through other concepts such as higher-order functions and type systems, we would also get additional benefits such as code-reusability and correctness.

In this thesis, we aim to formalize an extensible correctness proof of an implementation of automatic differentiation on a simply-typed lambda calculus in the \<Coq> proof assistant, opening up further possibilities for formally proving the correctness of more complex language features in the future.
Our formalization is based on a recent proof by Huot, Staton, and \Vakar{} \cite{huot2020correctness}.
They proved, using a denotational model of diffeological spaces, that their forward-mode emulating macro is correct when applied to a simply-typed lambda calculus with products, co-products and inductive types.

With this thesis we make the following core contributions:
\begin{itemize}
  \item Formalize the proofs of both the forward-mode and continuation-based automatic differentiation algorithms specified by Huot, Staton, and \Vakar{} \cite{huot2020correctness} in \<Coq>.
  \item Prove the semantic correctness of various useful compile-time optimizations techniques in the context of generating performant code for automatic differentiation.
  \item Extend the proofs with the array types and compile-time optimization rules by Shaikhha, et al.\cite{Shaikha2019}.
  \item Analyze both the requirements of and issues involved with giving a formal proof of correctness for the combinator-based reverse-mode automatic differentiation algorithm by \Vakar{}\cite{vkr2020reverse}.
\end{itemize}

% TODO: Add references
\Cref{sec:bg} includes a background section explaining many of the topics and techniques used in this thesis. The formalization of the ubiquitous forward-mode automatic differentiation is given in \cref{sec:forward}, starting from a base simply-typed lambda calculus extended with product types and incrementally adding new types and language constructs. \Cref{sec:opt,sec:continuation-base} give formalizations of optimization avenues through, respectively, program transformations and a continuation-based automatic differentiation algorithms.
Finally, \cref{sec:rev} gives our attempt at a formal proof of the combinator-based reverse-mode automatic differentiation algorithm.

% TODO: Add a more elaborate section for notations
% As a notational convention, we will use specialized notation in the definitions themselves.
% \<Coq> normally requires that pretty printed notations be defined separately from the definitions they reference.
% The letter $\Gamma$ is used for typing contexts while lowercase Greek letters are usually used for types.


% \section{Background}

% \subsection{Automatic differentiation}

Automatic differentiation (\textit{AD}) has a long and rich history, where its driving motivation is to efficiently calculate the derivatives of functions in a manner that is both correct and fast\cite{Baydin2015AutomaticDI}.
There are several different methods of implementing AD algorithms, such as source-code transformations or operator overloading.
These algorithms usually transform any program which implements some function to one that calculates its derivative.

There are two main variants of AD, namely forward mode and reverse-mode AD.
In forward mode AD, every term in the function trace is annotated with the corresponding derivative of that term.
These are also known as the respectively the primal and tangent traces.
So calculating the partial derivatives of sub-terms is structure preserving with respect to the normal calculation of terms.

This approach to forward mode AD can be explained by dual numbers as these are, mathematically seen, what we are calculating with\cite{Baydin2015AutomaticDI}. Dual numbers are numbers of the form of
$$
  x + x' \epsilon
$$
where $x, x' \in \denR$ and $\epsilon$ is a nilpotent number, such that $\epsilon^2 = 0$ and $\epsilon \neq 0$.
Notably, both primal and tangent values are tracked in this representation, namely the tangent value is present in the coefficients of $\epsilon$.
As an example, we can see that this is true for both addition and multiplication:
\begin{align*}
  (x + x' \epsilon) + (y + y' \epsilon) &= (x + y) + (x' + y')\epsilon \\
  (x + x' \epsilon)(y + y' \epsilon) &= (xy) + (xy' + yx')\epsilon
\end{align*}
Using the following scheme for function application:
\begin{align*}
  f(x + x' \epsilon) &= f(x) + f'(x)x'\epsilon
\end{align*}
We can also see that it follows the chain rule for function composition.
\begin{align*}
  f(g(x + x' \epsilon)) &= f(g(x) + g'(x)x'\epsilon)) \\
    &= f(g(x)) + f'(g(x))g'(x)x'\epsilon
\end{align*}
Using this, we can essentially calculate the derivative of any derivable function by interpreting the non-dual number input $x$ as its dual number counterpart of $x + 0\epsilon$.

To give a more elaborate example of how this works in forward mode AD, take the function $f(x, y) = x^2 + (x - y)$ as an example.
The dependencies between the terms and operations of the function is visible in the computational graph in Figure~\ref{fig:func_trace}.
The corresponding traces are filled in Table~\ref{table:func_trace} for the input values $x = 2, y = 1$.
We can calculate the partial derivative $\frac{\delta f}{\delta x}$ at this point by setting $x' = 1$ and $y' = 0$.
In this paper we will prove the correctness of a simple forward mode automatic differentiation algorithm with respect to the semantics of a simply-typed lambda calculus.

Reverse mode automatic differentiation takes a different approach.
It works backwards from the output by annotating each intermediate variable $v_i$ with an adjoint $v'_i=\frac{\delta y_i}{\delta v_i}$.
To do this, two passes are necessary.
Like the forward mode variant, a primal trace is needed to determine the intermediate variables and their dependencies.
The second pass calculates the derivatives by working backwards from the output using the adjoints, also called the adjoint trace.

The optimal choice between automatic differentiation variant is heavily dependent on the specific function being differentiated.
Preference is given for forward mode AD when the number of output variables exceeds the number of input variables, as it has to be rerun for each partial derivative of the function.
On the other hand, as reverse mode AD works backwards, the reverse-pass needs to be redone for each output variable.
In machine learning research, reverse mode AD is generally preferred as the objective functions generally contain a small number of output variables.

% TODO: Mention something about Eliott's categorical approach and its possible extension to a macro on combinators.

\begin{figure}
  \centering
  \includegraphics[scale=0.6]{./assets/function_trace.png}
  \caption{Computational graph of $f(x, y) = x^2 + (x - y)$}
  \label{fig:func_trace}
\end{figure}

\begin{table}
  \begin{center}
    \begin{tabular}{ l l l l l | l l l l l }
      \hline
      \multicolumn{5}{l}{Primal trace} & \multicolumn{5}{l}{Tangent trace} \\
      \hline
$v_{-1} $&$=$&$x$&$=$&$2$             &$v'_{-1}$&$=$&$x'$&$=$&$1$ \\
$v_0    $&$=$&$y$&$=$&$1$             &$v'_{0}$&$=$&$y'$&$=$&$0$ \\
      \hline
$v_1    $&$=$&$v_{-1}^2$&$=$&$4$      &$v'_{1}$&$=$&$2*v_{-1}$&$=$&$4$ \\
$v_2    $&$=$&$v_{-1} - v_{0}$&$=$&$1$&$v'_{2}$&$=$&$v'_{-1}-v'_{0}$&$=$&$1$ \\
$v_3    $&$=$&$v_1 + v_2$&$=$&$5$     &$v'_{3}$&$=$&$v'_1 + v'_2$&$=$&$5$ \\
      \hline
$f      $&$=$&$v_3$&$=$&$5$           &$f'$&$=$&$v'_3$&$=$&$5$ \\
      \hline
    \end{tabular}
  \end{center}
  \caption{Primal and tangent traces of $f(x, y) = x^2 + (x - y)$}
  \label{table:func_trace}
\end{table}

% \subsection{Denotational semantics}
% A formal semantics of programming language: An introduction
% TODO: Work through MV feedback
Denotational semantics enables reasoning about programs using formal mathematics.
It also functions as a hotbed for new and innovative language designs and algorithms.
The most well-known example is the domain theory model given by Dana Scott and Christopher Strachey\cite{Scott1977} for lambda calculi.
To be able to formalize non-termination and partiality, they thought to use concepts such as partial orderings and least fixed points\cite{aaby2020}.
In this model, programs are regularly interpreted as partial functions, and recursive computations as taking the fixpoint of such functions.
Non-termination, on the other hand, is formalized as a value $\bot$ that is lower in the ordering relation than any other element.

Automatic differentiation introduces a challenge in constructing a denotational semantics as the notion of differentiability needs to be included.
If the language under consideration were to be restricted to real-typed terms, Cartesian spaces would have been sufficient as any well-typed term $x_1 : \synR, \dots, x_n : \synR \vdash t : \synR$ could be interpreted as the corresponding smooth function $\llbracket t \rrbracket : \denR^n \to \denR$.
Note that we use $\synR$ as the syntactic type for real numbers, while $\denR$ is its denotational counterpart.
Using Cartesian spaces, however, does not work when function types are added as their denotational equivalent, function spaces, are not supported\cite{huot2020correctness}.
The original pen and paper proof of automatic differentiation this thesis is based on by Huot, Staton and \Vakar{}\cite{huot2020correctness}, remedies this issue by using diffeological spaces as the underlying mathematical model.

For the purpose of this thesis, however, we were able to avoid using diffeological spaces by directly encoding the property of differentiability in the logical relation itself.
We were also able to avoid domain theoretical models such as $\omega$-cpos by excluding language constructs such as recursion and iteration where non-termination and partiality come into play.
As a part of its type system, \<Coq> contains a set-theoretical model available under the sort \<Set> in its type system, which is satisfactory as the denotational semantics for our language.

Because we use real numbers as the ground type in our language, we also needed an encoding of the real numbers in \<Coq>.
While support for real numbers in the standard library in \<Coq> has improved in recent times from one based on an axiomatic definition to one involving Cauchy sequences\fancyfootnote{https://coq.inria.fr/library/Coq.Reals.ConstructiveCauchyReals.html}, it is still insufficient for our purposes.
We also need a notion of differentiability which we will use to among others, phrase correctness of our automatic differentiation macros.
Instead of attempting to encode this by hand, we opted for the more comprehensive library \<Coquelicot>\cite{Boldo2015CoquelicotAU}, which contains many useful user-friendly definitions for doing calculus.

% \subsection{Coq}
% TODO: Work through MV feedback
% TODO: Recheck these sections
\<Coq> is a proof assistant based on the calculus of constructions type theory created by Thierry Coquand and G\'{e}rard Huet\cite{Coquand1988}.
In the past 30 years since it has been released, research has contributed to extending the proof assistant with additional features such as inductive and co-inductive data types\cite{Coquand1990}, dependent pattern matching\cite{Sozeau2010} and advanced modular constructions for organizing large mathematical proofs\cite{Sozeau2008}\cite{Mahboubi2013}.

The core of this type theory is based on constructive logic and so many of the laws known in classical logic are not provable.
An example includes the law of the excluded middle, $\forall A, A \vee \neg A$.
In some cases they can, however, be safely added to \<Coq> without making its logic inconsistent. These are readily available in the standard library.
Due to its usefulness in proving propositions over functions, we will make use of the functional extensionality axiom in \<Coq>, which states that functions are equal if they are extensionally equivalent.

\subsubsection{Language representation}
\label{sec:language_repr}
% TODO: Work through MV feedback

\begin{figure}
  \begin{mathpar}
    \inferrule*[Right=\textsc{TVar}]
      {elem\ n\ \Gamma = \tau}
      {\Gamma \vdash var\ n : \tau} \and
    \inferrule*[Right=\textsc{TAbs}]
      {(\sigma, \Gamma) \vdash t : \tau}
      {\Gamma \vdash t : \sigma \rightarrow \tau} \\ \and
    \inferrule*[Right=\textsc{TApp}]
      {\Gamma \vdash t1 : \sigma \rightarrow \tau \\
        \Gamma \vdash t2 : \sigma}
      {\Gamma \vdash t1\ t2 : \tau}
  \end{mathpar}
  \label{fig:stlc_infer}
  \caption{Type-inferrence rules for a simply-typed lambda calculus using De-Bruijn indices}
\end{figure}

When defining a simply-typed lambda calculus, there are two main possibilities\cite{plfa2019}.
The arguably simpler variant, known as an extrinsic representation, is traditionally the one introduced to new students learning \<Coq>.
In the extrinsic representation, the terms themselves are untyped and typing judgments are defined separately as relations between the types and terms. A basic example of working with this is given by Pierce, et. al.\cite{Pierce:SF2}.
This, however, required many additional lemmas and machinery to be proved to be able to work with both substitutions and contexts as these are defined separate from the terms.
As an example, the preservation property which states that reduction does not change the type of a term, needs to be proven explicitly.
The other approach, also called an intrinsic representation, makes use of just a single well-typed definition.
Ill-typed terms are made impossible by the type checker.
This representation, while beneficial in the proof load, however complicates much of the normal machinery involved in programming language theory.
One example is how one would define operations such as substitutions or weakening.

But even when choosing an intrinsic representation, the problem of variable binding persists.
Much meta-theoretical research has been done on possible approaches to this problem each with their own advantages and disadvantages.
The POPLmark challenge gives a comprehensive overview of each of the possibilities in various proof assistants\cite{Aydemir2005}.
An example of an approach is the nominal representation where every variable is named.
While this does follow the standard format used in regular mathematics, problems such as alpha-conversion and capture-avoidance appears.

\begin{listing}[h]
  \begin{minted}{coq}
  Inductive ty : Type :=
    | ~unit~ : ty
    | ~\Rightarrow~ : ty ~\rightarrow~ ty ~\rightarrow~ ty.

  Inductive tm : Type :=
    | var : string ~\rightarrow~ tm
    | abs : string ~\rightarrow~ ty ~\rightarrow~ tm ~\rightarrow~ tm
    | app : tm ~\rightarrow~ tm ~\rightarrow~ tm.
  \end{minted}
  \caption{Simply typed \lambda-calculus using an extrinsic nominal representation.}
  \label{lst:nominal_stlc}
\end{listing}

The approach used in the rest of this thesis is an extension of the De-Bruijn representation which numbers variables relative to the binding lambda term.
In this representation the variables are referred to as well-typed De-Bruijn indices.
A significant benefit of this representation is that the problems of capture avoidance and alpha equivalence are avoided.
As an alternative, instead of using numbers to represent the distance, indices within the typing context can be used to ensure that a variable is always well-typed and well-scoped.
While the idea of using type indexed terms has been both described and used by many authors\cite{Altenkirch99}\cite{McBride04}\cite{Adams06}, the specific formulation used in this thesis using separate substitutions and rename operations was fleshed out in Coq by Nick Benton, et. al.\cite{Benton2011}, and was also used as one of the examples in the second POPLmark challenge which deals with logical relations\cite{poplmark_reloaded}.
While this does avoid the problems present in the nominal representation, it unfortunately does have some problems of its own.
Variable substitutions have to be defined using two separate renaming and substitution operations.
Renaming is formulated as extending the typing context of variables, while substitution actually swaps the variables for terms.
Due to using indices from the context as variables, some lifting boilerplate is also needed to manipulate contexts.

\begin{listing}[h]
  \begin{minted}{coq}
  Inductive ~\tau \in \Gamma~ : Type :=
    | Top : ~\forall \Gamma \tau, \tau \in (\tau::\Gamma)~
    | Pop : ~\forall \Gamma \tau \sigma, \tau \in \Gamma \rightarrow \tau \in (\sigma::\Gamma)~.

  Inductive tm ~\Gamma \tau~ : Type :=
    | var : ~\forall \Gamma \tau, \tau \in \Gamma \rightarrow tm \Gamma \tau~
    | abs : ~\forall \Gamma \tau \sigma, tm (\sigma::\Gamma) \tau \rightarrow tm \Gamma (\sigma \Rightarrow \tau)~
    | app : ~\forall \Gamma \tau \sigma, tm \Gamma (\sigma \Rightarrow \tau) \rightarrow tm \Gamma \sigma \rightarrow tm \Gamma \tau~.
  \end{minted}
  \caption{Basis of a simply-typed \lambda-calculus using a strongly typed intrinsic formulation.}
  \label{lst:strong_stlc}
\end{listing}

% TODO: Work out how substitutions work

\subsubsection{Dependently-typed programming in Coq}

In \<Coq>, one can normally write function definitions using either case-analysis as is done in other functional languages, or using \<Coq>'s tactics language.
Using the standard case-analysis functionality can cause the code to be complicated and verbose, even more so when proof terms are present in the function signature.
This has been caused by the previously poor support in Coq for dependent pattern matching.
Using the return keyword, one is able to vary the result type of a match expression. But due to requirement Coq used to have that case expressions be syntactically total, this could be very annoying to work with.
One other possibility would be to write the function as a relation between its input and output.
This also has its limitations as you then lose computability as Coq treats these definitions opaquely. In this case the standard \<simpl> tactic which invokes \<Coq>'s reduction mechanism is not able to reduce instances of the term.
This often requires the user to write many more proofs to be able to work with the definitions.

As an example, we will work through defining a length indexed list and a corresponding head function limited to lists of length at least one in Snippet~\ref{lst:dt_ilist}.
Using the \<Coq> keyword return, it is possible to let the return type of a match expressions depend on the result of one of the type arguments.
This makes it possible to define an auxiliary function which, while total on the length of the list, has an incorrect return type. It namely returns the type unit if the input list had the length zero.
We can then use this auxiliary function in the actual head function by specifying that the list has length at least one.
It should be noted that more recent versions of Coq do not require that case expressions be syntactically total, so specifying that the input list has a length of at least zero is enough to eliminate the requirement for the zero-case.

\begin{listing}
  \begin{minted}{coq}
  Inductive ilist : ~Type \rightarrow nat \rightarrow Type~ :=
    | nil : ~\forall A, ilist A 0~
    | cons : ~\forall A n, A \rightarrow ilist A n \rightarrow ilist A (S n)~

  Definition hd' {A} n (ls : ilist A n) :=
    match ls in (ilist A n) return
      (match n with
      | O => unit
      | S _ => A end) with
    | nil => tt
    | cons h _ => h
  end.

  Definition hd {A} n (ls : ilist A (S n)) : A := hd' n ls.
  \end{minted}
  \caption{Definition of a length indexed list and hd using the return keyword, adapted from Certified Programming with Dependent Types\cite{ChlipalaCPDT}.}
  \label{lst:dt_ilist}
\end{listing}

Mathieu Sozeau introduces an extension to \<Coq> via a new keyword \<Program> which allows the use of case-analysis in more complex definitions\cite{Sozeau2006}\cite{Sozeau2007}.
To be more specific, it allows definitions to be specified separately from their accompanying proofs, possibly filling them in automatically if possible.
While this does improve on the previous situation, using the definitions in proofs can often be unwieldy due to the amount of boilerplate introduced.
This makes debugging error messages even harder than it already is in a proof assistant. This approach was used by Benton in his formulation of strongly typed terms.

Sozeau further improves on this introducing a method for user-friendlier dependently-typed pattern matching in \<Coq> in the form of the \<Equations> library\cite{Sozeau2010}\cite{Sozeau2019}.
This introduces \<Agda>-like dependent pattern matching with with-clauses.
It does this by using a notion called coverings, where a covering is a set of equations such that the pattern matchings of the type signature are exhaustive.
There are two main ways to integrate this in a dependently typed environment, externally where it is integrated as high-level constructs in the pattern matching core as \<Agda> does it, or internally by using the existing type theory and finding witnesses of the covering to prove the definition correct, which is the approach used by Sozeau.
Due to the intrinsic typeful representation this paper uses, much of this was invaluable when defining the substitution operators as the regular type checker in Coq often had difficulty unifying dependently typed terms in certain cases.

\begin{listing}
  \begin{minted}{coq}
  Equations hd {A n} (ls : ilist A n) (pf : n <> 0) : A :=
  hd nil pf with pf eq_refl := {};
  hd (cons h n) _ := h.
  \end{minted}
  \caption{Definition of hd using \<Equations>}
  \label{lst:dt_ilist_hd_equations}
\end{listing}


% \subsection{Logical relations}
% TODO: Work through MV feedback
Logical relations arguments are a proof technique often employed when proving programming language properties of statically typed languages\cite{skorstengaard2019introduction}. There are two main ways they are used, namely as unary and binary relations.
Unary logical relations, also known as logical predicates, are predicates over single terms and are typically used to prove language characteristics such as type safety or strong normalization.
Binary logical relations on the other hand are used to prove program equivalences, usually in the context of denotational semantics as we will do.
There have been many variations on the versatile technique from syntactic step-indexed relations which have been used to reason about recursive types and general references\cite{Ahmed2006}, to open relations which enable working with terms of non-ground type\cite{barthe2020versatility}\cite{huot2020correctness}.
Logical relations in essence are relations between terms defined by induction on their types.
A logical relations proof consists of 2 main steps.
The first states the terms for which the property is expected to hold are in the relation, while the second states that the property of interest follows from the relation.
The second step is easier to prove as it usually follows from the definition of the relation. The first on the other hand, will often require proving a generalized variant, called the fundamental property of the logical relation.
In most cases this requires that the relation is correct with respect to applying substitutions.

A well-known logical relations proof is the proof of strong normalization of well-typed terms, which states that all terms eventually terminate.
An example of a logical relation used in such a proof using the intrinsic strongly-typed formulation is given in \cref{lst:sn_logical_relation}.
Noteworthy is the case for function types, where one needs to prove that applying a function preserves the strong normalization property.
If one were to attempt the proof of strong normalization without using logical relations, the proof would get stuck in the cases dealing with function types.
More specifically when applying a function term to an argument term which terminates, the induction hypothesis is not strong enough to prove that substituting the argument into the body of the abstraction results in a terminating term.

% TODO: Remove/Move this
% The proof given in the paper this thesis is based on, is a logical relations proof on the denotational semantics using diffeological spaces as its domains\cite{huot2020correctness}.
% A similar, independent proof of correctness was given by Barthe, et al.\cite{barthe2020versatility} using a syntactic relation on the operational semantics.

\begin{listing}
  \begin{minted}{coq}
    Equations SN {~\Gamma~} ~\tau~ (t : ~tm \Gamma \tau~): Prop :=
    SN unit t := halts t;
    SN ~(\tau \Rightarrow \sigma)~ t := halts t ~$\wedge$~
      ~(\forall (s : tm \Gamma \tau), SN \tau s \rightarrow SN \sigma (app \Gamma \sigma \tau t s))~;
  \end{minted}
  \caption{Example of a logical predicate used in a strong normalization proof in the strongly-typed intrinsic representation}
  \label{lst:sn_logical_relation}
\end{listing}

% \subsection{Related work}\label{sec:related-work}
\textbf{AD Formalizations.} While there exists proofs of forward-mode AD algorithms\cite{huot2020correctness}\cite{barthe2020versatility}\cite{10.1145/3371106} and many more implementations\cite{Shaikha2019}\cite{Margossian2019ARO} in a functional setting, there have been relative few attempts at formalized proofs in proof assistant.
In 2002, Mayero did a formalized correctness proof of an AD framework implemented in \<Fortran> in \<Coq>\cite{Mayero:CorrectnessProofAD}.
Their minimal core language included assignments and sequences as language constructs, and excluded all forms of non-sequential control flow.
They also restricted the terms in their language to first-order types.

\textbf{Programming Language Metatheory.} Much meta-theoretical research has been done on encoding programming languages in proof assistants\cite{Aydemir2005}.
Examples include the weak higher-order abstract syntax approach worked out in Coq by Despeyroux, et al.\cite{10.1007/BFb0014049}, which shallowly embeds abstractions as functions $abs : (var \to tm) \to tm$.
The parametric HOAS variant by Chlipala\cite{10.1145/1411204.1411226}, is an interesting polymorphic generalization of this technique.
PHOAS, like the strongly-typed terms representation used in this thesis, avoids the problems of alpha-conversion and capture avoidance while still being somewhat user-friendly.
The locally nameless approach introduced by many various authors\cite{McKinna_Pollack_1997}\cite{10.1007/3-540-57826-9_152}\cite{10.1145/1017472.1017477} takes a hybrid approach and preserves names for free variables while using the De-Bruijn representation for bound variables.

With regards to denotational semantics, both Benton, et al.\cite{Benton2009} and Dockins\cite{Dockins2014} present domain-theoretical libraries in \<Coq>.
% Proof-wise, Huot, Staton and \Vakar{} use diffeological spaces in their correctness proof of automatic differentiation on a simply-typed lambda calculus.
% Contrastingly, Abadi and Plotkin\cite{10.1145/3371106} use the more conventional \omega-cpos to be able to support partiality.

\textbf{Forward-Mode AD.}
The earliest found description of an approach for forward-mode AD on functional languages is by Karczmarczuk\cite{Karczmarczuk98functionaldifferentiation}, on first-order terms.
Siskind and Pearlmutter presented a "nestable" variant of a forward-mode AD algorithm using the dual numbers representation.
This same algorithm is used in the \FS{} library, DiffSharp\cite{Baydin2015AutomaticDI}.
A nearly identical variation, implemented in \<Haskell>, is given by Elliott\cite{Elliott2009-beautiful-differentiation}.
This uncontroversial implementation of forward-mode AD is also discussed in the survey by Baydin et al.\cite{Baydin2015AutomaticDI}.

\textbf{Reverse-Mode AD.} There are many interpretations of reverse-mode AD on functional languages.
Most well-known is the one by Pearlmutter and Siskind\cite{PearlmutterSiskind2008}, which is one of the first attempts at reverse-mode AD in a functional context and introduces the practice of using various first-class operations to calculate derivatives.
These operations very often involve maintaining some notion of state to keep track of adjoints.
The specific approach by Pearlmutter and Siskind uses non-local program transformations as their primitive construct of choice.
In the trend of define-by-run algorithms, whose main strategy involves building up the reverse pass of the algorithm during runtime, their primitive $\Jrev$ uses reflection to perform reverse-mode AD at runtime.

Define-by-run algorithms, however, lose much of the optimization opportunities provided by the explicit compilation process involved with programming languages.
As an improvement on the approach by Pearlmutter and Siskind, Wang, et al.\cite{ShiftReset:Backprop} negate some of the issues associated with define-by-run algorithms by using multi-stage programming to reclaim some optimization opportunities.
Their algorithm makes heavy usage of both delimited continuations as well as state by way of references.
% TODO: Technique using linear negation is an extension of \cite{10.1145/3371106}, so discuss that next

Abadi and Plotkin\cite{10.1145/3371106} also make use of reverse-mode AD primitives, but do so in the context of a define-by-run trace-based algorithm.
Their reverse-mode primitive is given special treatment in their operational semantics, essentially symbolically evaluating terms into so-called trace terms, which are devoid of control-flow constructs.

One significant issue with defining define-then-run reverse-mode algorithms is how to treat many of the various control-flow constructs such as conditionals, loops or higher-order types.
Elliott\cite{Elliott-2018-ad-icfp} gave an interesting principled approach to reverse-mode AD from the perspective of category theory by formulating the algorithm as a functor.
Their method, though, is still limited to first-order programs.
An extension to higher-order types by \Vakar{}\cite{vkr2020reverse} is further discussed in \cref{sec:rev}.

% \cite{ShiftReset:Backprop}\cite{Brunel2020BackpropagationIT}\cite{PearlmutterSiskind2008}
% Advancements in the understanding of both forward-mode AD algorithms and meta-theoretical proof techniques have supplied enough resources to be able to do a formal proof.


\section{Formalizing Forward-Mode AD}
  The formalization of the forward-mode automatic differentiation macro will be explained in the following sections.
  The formal proof will start from a base simply-typed lambda calculus enriched with product types and incrementally add both sum and array types.
  Also included in the final language is an implementation of primitive recursion using integer types.
  Many of the theorems and lemmas will stay consistent between sections as the overarching correctness statement does not change.

  % \subsection{Simply Typed Lambda Calculus}\label{sec:formal_stlc}
  % TODO: Work through MV feedback
  % Talk about simply typed lambda calculus,
  % Something about Λ_δ^{+, *, R}
  % Give examples of functions
  % talk about denotations and
  As mentioned in the background section~\ref{sec:language_repr}, we will make use of De-Bruijn indices in an intrinsic representation to formulate our language.
  We include both addition and multiplication as example operations on the real numbers, but the proofs are easily extensible to other primitive operations.
  Our base language consists of the classic simply-typed lambda calculus with product types and real numbers.

  Both the language constructs and the typing rules for this language are common for a simply-typed lambda calculus, as shown in figure~\ref{fig:base_infer}.
  As expected, we include variables, applications, and abstractions in the language using, respectively, the \<var>, \<app>, and \<abs> terms.
  We work with projection products, whose elimination rules are encoded in the  \<first> and \<second> terms. The \<tuple> term is used to represent the introduction rule.
  For real numbers, \<rval> is used to introduce real numbered constants and \<add> and \<mul> will be used to respectively encode addition and multiplication.

  \begin{figure}
    \begin{mathpar}
      \inferrule*[Right=\textsc{TVar}]
        {elem\ n\ \Gamma = \tau}
        {\Gamma \vdash \var{n} : \tau} \and
      \inferrule*[Right=\textsc{TAbs}]
        {(\sigma, \Gamma) \vdash t : \tau}
        {\Gamma \vdash \abs{t} : \sigma \rightarrow \tau} \\ \and
      \inferrule*[Right=\textsc{TApp}]
        {\Gamma \vdash t1 : \sigma \rightarrow \tau \\
          \Gamma \vdash t2 : \sigma}
        {\Gamma \vdash \app{t1}{t2} : \tau} \\ \and
      \inferrule*[Right=\textsc{TTuple}]
        {\Gamma \vdash t1 : \tau \\
          \Gamma \vdash t2 : \sigma}
        {\Gamma \vdash \tuple{t1}{t2} : \tau \times \sigma} \\ \and
      \inferrule*[Right=\textsc{TFst}]
        {\Gamma \vdash t : \tau \times \sigma}
        {\Gamma \vdash \first{t} : \tau} \and
      \inferrule*[Right=\textsc{TSnd}]
        {\Gamma \vdash t : \tau \times \sigma}
        {\Gamma \vdash \second{t} : \sigma} \\ \and
      \inferrule*[Right=\textsc{TRVal}]
        {r \in \denR}
        {\Gamma \vdash \rval{r} : \synR} \\ \and
      \inferrule*[Right=\textsc{TAdd}]
        {\Gamma \vdash r1 : \synR \\
          \Gamma \vdash r2 : \synR \\ }
        {\Gamma \vdash \add{r1}{r2} : \synR} \and
      \inferrule*[Right=\textsc{TMull}]
        {\Gamma \vdash r1 : \synR \\
        \Gamma \vdash r2 : \synR \\ }
      {\Gamma \vdash \mul{r1}{r2} : \synR} \and
    \end{mathpar}
    \caption{Type-inference rules for the base simply-typed lambda calculus}
    \label{fig:base_infer}
  \end{figure}

  % How we translated this into the well-typed intrinsic representation
  These can be translated into \<Coq> definitions in a reasonably straightforward manner, with each case keeping track of both how the typing context and types change.
  In the \<var> case, we need some way to determine what type the variable is referencing.
  Like many others previously\cite{Benton2011}\cite{Coquand1994}, instead of using indices into the list accompanied by a proof that the index does not exceed the length of the list, we make use of an inductively defined type evidence to type our variables as shown in code snippet~\ref{lst:strong_stlc}.
  The cases for \<app> and \<abs> are as expected, where variables in the body of abstractions can reference their respective arguments.

  Note that in the original proof by Huot, Staton, and \Vakar{} \cite{huot2020correctness}, they made use of n-ary products accompanied by pattern matching expressions.
  We opted to implement binary projection products, as these are conceptually simpler while still retaining much of the same functionality expected of product types.

  \begin{listing}
    \begin{minted}{coq}
      Inductive tm ~(\Gamma : Ctx) : ty \rightarrow Type~ :=
        ...
        (* Binary projection products *)
        | tuple : ~forall {\tau \sigma},
          tm \Gamma \tau \rightarrow
          tm \Gamma \sigma \rightarrow
          tm \Gamma (\tau \synStar \sigma)~
        | first : ~forall {\tau \sigma}, tm \Gamma (\tau \synStar \sigma) \rightarrow tm \Gamma \tau~
        | second : ~forall {\tau \sigma}, tm \Gamma (\tau \synStar \sigma) \rightarrow tm \Gamma \sigma~
        (* Operations on reals *)
        | rval : ~forall r, tm \Gamma $\synR$~
        | add : ~tm \Gamma $\synR$ \rightarrow tm \Gamma $\synR$ \rightarrow tm \Gamma $\synR$~
        | mul : ~tm \Gamma $\synR$ \rightarrow tm \Gamma $\synR$ \rightarrow tm \Gamma $\synR$~
    \end{minted}
    \caption{Terms in our language related to product and real types.}
    \label{lst:stlc_prod_r}
  \end{listing}

  % \begin{listing}
  %   \begin{minted}{coq}
  % Definition Ctx : Type := list ty.

  % Inductive tm ~(\Gamma : Ctx) : ty \rightarrow Type~ :=
  %   (* Base STLC *)
  %   | var : ~forall \tau,
  %     \tau ∈ \Gamma \rightarrow tm \Gamma \tau~
  %   | app : ~forall \tau \sigma,
  %     tm \Gamma (\sigma \Rightarrow \tau) \rightarrow
  %     tm \Gamma \sigma \rightarrow
  %     tm \Gamma \tau~
  %   | abs : ~forall \tau \sigma,
  %     tm (\sigma::\Gamma) \tau \rightarrow tm \Gamma (\sigma \Rightarrow \tau)~

  %   (* Operations on real numbers *)
  %   | const : ~R \rightarrow tm \Gamma Real~
  %   | add : ~tm \Gamma Real \rightarrow tm \Gamma Real \rightarrow tm \Gamma Real~
  %   | mul : ~tm \Gamma Real \rightarrow tm \Gamma Real \rightarrow tm \Gamma Real~

  %   (* Binary projection products *)
  %   | tuple : ~forall {\tau \sigma},
  %     tm \Gamma \tau \rightarrow
  %     tm \Gamma \sigma \rightarrow
  %     tm \Gamma (\tau \times \sigma)~
  %   | first : ~forall {\tau \sigma}, tm \Gamma (\tau \times \sigma) \rightarrow tm \Gamma \tau~
  %   | second : ~forall {\tau \sigma}, tm \Gamma (\tau \times \sigma) \rightarrow tm \Gamma \sigma~
  %   \end{minted}
  %   \caption{\<Coq> definition of the base lambda calculus}
  %   \label{lst:stlc_base}
  % \end{listing}

  % TODO: Find better origins of this macro
  We use the same inductively defined macro on types and terms used by many previous authors to implement the forward-mode automatic differentiation macro\cite{huot2020correctness}\cite{barthe2020versatility}\cite{Shaikha2019}.
  The forward-mode macro, $\D$, keeps track of both primal and tangent traces using tuples as respectively its first and second elements.
  In most cases, the macro simply preserves the structure of the language.
  The cases for real numbers such as addition and multiplication are the exception.
  Here, the element encoding the tangent trace needs to contain the proper syntactic translation of the derivative of the operation.

  Due to the intrinsic nature of our language representation, the macro also needs to be applied to both the types and typing context to ensure that the terms remain well-typed.
  In other words, for any well-typed term $\Gamma \vdash t : \tau$, applying the forward-mode macro results in a well-typed term in the macro-expanded context, $\D(\Gamma) \vdash \D(t) : \D(\tau)$.

  \begin{figure}[H]
    \centering
    \begin{equation*}
      \begin{split}
        \D(\synR) &= \synR \synStar \synR \\
        \D(\tau \synStar \sigma) &= \D(\tau) \synStar \D(\sigma) \\
        \D(\tau \synFunc \sigma) &= \D(\tau) \synFunc \D(\sigma)
      \end{split}
      \;\;\;\;\;\;
      \begin{split}
        \D(\rval{n}) &= \tuple{(\rval{n})}{(\rval{0})} \\
        \D(\add{n}{m}) &= \tuple{(\add{n}{m})}{(\add{n'}{m'})} \\
        \D(\mul{n}{m}) &= \tuple{(\mul{n}{m})} \\
          &{(\add{(\mul{n'}{m})}{(\mul{m'}{n})})})
      \end{split}
    \end{equation*}
    \caption{Macro on base simply-typed lambda calculus}
    \label{eqn:macro_base}
  \end{figure}

  Applying the macro to a term gives the syntactic counterparts of both their primal and tangent denotations as a tuple.
  These terms can be accessed with projections to implement the various derivative implementations of the operations on real terms included in the language.
  Note that applying the macro to the case for variables does nothing as the macro is also applied to the typing context, so variables implicitly already reference macro-applied terms.

  As we restrict our language to total constructions and excluding concepts such as general recursion and iteration, it suffices to give our language a set-theoretic denotational semantics.
  In this case the types $\synR, \synFunc, \synStar$ directly correspond to their \<Coq> equivalent, respectively $\denR, \denFunc, \denStar$.
  Well-typed terms of type $\tau$, given typing context $\Gamma$, will denotate to functions $\llbracket \Gamma \rrbracket \to \llbracket \tau \rrbracket$.

  \begin{figure}
    \centering
    \begin{gather*}
      \begin{aligned}
        \llbracket \synR \rrbracket &= \denR \\
        \llbracket \tau \synStar \sigma \rrbracket &=
          \llbracket \tau \rrbracket \denStar \llbracket \sigma \rrbracket \\
        \llbracket \tau \synFunc \sigma \rrbracket &= \llbracket \tau \rrbracket \denFunc \llbracket \sigma \rrbracket \\
        \\
        \llbracket \<Top> \rrbracket &= \<hd> \\
        \llbracket \<Pop>\ v \rrbracket &= \llbracket v \rrbracket \circ \<tl> \\
      \end{aligned}
      \;\;\;\;\;\;
      \begin{aligned}
        \llbracket \var{v} \rrbracket &=
          \lambda x. lookup\ \llbracket v \rrbracket\ x \\
        \llbracket \app{t_1}{t_2} \rrbracket &=
          \lambda x. (\llbracket t_1 \rrbracket(x)) (\llbracket t_2 \rrbracket(x)) \\
        \llbracket \abs{t} \rrbracket &=
          \lambda x\ y. \llbracket t \rrbracket(y :: x) \\
        \llbracket \add{t_1}{t_2} \rrbracket &=
          \lambda x. \llbracket t_1 \rrbracket(x) + \llbracket t_2 \rrbracket(x) \\
        \llbracket \mul{t_1}{t_2} \rrbracket &=
          \lambda x. \llbracket t_1 \rrbracket(x) * \llbracket t_2 \rrbracket(x) \\
        \llbracket \tuple{t_1}{t_2} \rrbracket &=
          \lambda x. (\llbracket t_1 \rrbracket(x), \llbracket t_2 \rrbracket(x)) \\
        \llbracket \first{t} \rrbracket &=
          \lambda x. fst(\llbracket t \rrbracket(x)) \\
        \llbracket \second{t} \rrbracket &=
          \lambda x. snd(\llbracket t \rrbracket(x)) \\
      \end{aligned} \\ \\
      \begin{aligned}
        fst &=
          \lambda x. let\ (x, y) \coloneqq \llbracket t \rrbracket(x)\ in\ x \\
        snd &=
          \lambda x. let\ (x, y) \coloneqq \llbracket t \rrbracket(x)\ in\ y \\
      \end{aligned}
    \end{gather*}
    \caption{Denotations of the base simply-typed lambda calculus}
    \label{eqn:denotation_base}
  \end{figure}

  Denotating the terms in our language now corresponds to finding the appropriate inhabitants in the denotated types.
  As typing contexts, $\Gamma$, are represented by lists of types.
  The appropriate way to denotate these would be to map the denotation function over the list.
  The resulting heterogeneous list contains the denotations of each type in the list in the correct order.
  The specific implementation of heterogeneous lists used in the proof corresponds to the one given by Adam Chlipala\cite{ChlipalaCPDT}.
  In this implementation, heterogeneous lists consist of an underlying list of some type $A$ and an accompanying function $A \to Set$, which in our use case are, respectively, the typing context and the denotation function.

  When giving the constructs in our language their proper denotations, most of the cases are straightforward.
  Notable is the case for variables, where we made use of the inductively defined type evidence to type our terms.
  Remember that to type variables in our term language, we have to also give the exact position of the type we are referencing in the typing context.
  Similarly as denotations, we are able to transform this positional information to generate a specialized lookup function, which given a valid typing context, gives a term denotation with the correct type.
  Essentially, we do a lookup into the heterogeneous list of denotations corresponding to the typing context.

  \begin{minted}{coq}
    Equations denote_v ~$\Gamma$~ ~$\tau$~ (v: ~$\tau \in \Gamma$~) : ~$\llbracket \Gamma \rrbracket \rightarrow \llbracket \tau \rrbracket$~ :=
    denote_v (Top ~$\Gamma$~ ~$\tau$~) := hd;
    denote_v (Pop ~$\Gamma$~ ~$\tau$~ ~$\sigma$~ v) := denote_v v ~$\circ$~ tl.
  \end{minted}

  % \begin{listing}
  %   \begin{minted}{coq}
  %   \end{minted}
  %   \caption{Denotatonal semantics for types, typing contexts and lookups.}
  %   \label{lst:denotation_types}
  % \end{listing}

  % \begin{listing}
  %   \begin{minted}{coq}
  %   \end{minted}
  %   \caption{Denotatonal semantics for the base simply-typed lambda calculus.}
  %   \label{lst:denotation_base}
  % \end{listing}

  % In the section denotation
  % Explain expressiveness of base language
  % Work out examples

  % As mentioned by by Barthe, et al.\cite{barthe2020versatility}, this small calculus, $\lambdaBase$, accompanied with the arguably simple set-theoretic denotational semantics is expressive enough to encode the higher-order polynomials containing the addition and multiplication operators.

  \begin{example}[Square]
    $abs\ (mul\ (var\ Top)\ (var\ Top))$ denotates to the square function $\lambda x. x * x$.
    \begin{proof}
      This follows from the definition of our denotation functions.
      \begin{align*}
        \llbracket abs\ &(mul\ (var\ Top)\ (var\ Top)) \rrbracket\ [] \\
          &\equiv \lambda x.
            \llbracket mul\ (var\ Top)\ (var\ Top) \rrbracket\ [x] \\
          &\equiv \lambda x.
            \llbracket var\ Top \rrbracket\ [x] *
              \llbracket var\ Top \rrbracket\ [x] \\
          &\equiv \lambda x. x * x \qedhere
      \end{align*}
    \end{proof}
  \end{example}

  % TODO: give reasonable examples

  Using the denotation rules in Figure~\ref{eqn:denotation_base}, syntactically well-typed terms in our language of the form $x_1 : \synR, \dots, x_n : \synR \vdash t : \synR$ can be interpreted as their corresponding smooth functions $f : \denR^n \to \denR$.
  Intuitively, the free variables in the syntactic term $t$ correspond to the parameters of the denotation function $f$.

  Although Barthe, et al.\cite{barthe2020versatility} gave a syntactic proof of correctness of the macro, our formal proof follows the more denotational style of proof given by Huot, Staton and \Vakar{}\cite{huot2020correctness}.
  Likewise, our proof of correctness will follow a similar logical relations argument.
  While both approaches have their merits, the proof using the denotational semantics requires less technical bookkeeping of open and closed terms.

  Informally, the correctness statement of the forward-mode macro will consists of the assertion that the denotation of any macro-applied term of type $x_1 : \synR, \dots, x_n : \synR \vdash t : \synR$ will result in a pair of both the denotation of the original term and the derivative of that denotation.
  Note that while both the free variables and result type of the term $t$ are restricted to type $\synR$, $t$ itself can contain subterms of higher-order types.

  The logical relation will ensure that both the smoothness property and the derivatives are preserved over higher-order types.
  We define the logical relation as a type-indexed relation between denotations of both terms and their macro-applied variants, so for any type $\tau$, $S_\tau$ is the relation between functions $R \rightarrow \llbracket \tau \rrbracket$ and $R \rightarrow \llbracket \D(\tau) \rrbracket$.

  When $\tau = \synR$, the denotation of the macro-applied term should give both the original denotation and its derivative.
  With function types, as long as the relation is valid for the argument, applying these argument functions to the tracked denotations should preserve the relation.
  Some care has to be taken in the case for products.
  Notably, the denotations of the subterms, $R \rightarrow \llbracket \tau \rrbracket$ and $R \rightarrow \llbracket \sigma \rrbracket$, should be existentially quantified as these are dependent on the original denotation $R \rightarrow \llbracket \tau \times \sigma \rrbracket$.

  \begin{definition}(Logical relation)
    Denotation functions $f$ and their corresponding derivatives $g$ are inductively defined on the structure of our types such that they follow the relation
    \begin{equation}
      S_\tau(f, g) =
        \left\{
          \begin{array}{ll}
            smooth\ f \wedge
              g = \lambda x. (f(x), \frac{\partial f}{\partial x}(x))
              & : \tau = R \\
            \exists f_1, f_2, g_1, g_2,
              & : \tau = \sigma \times \rho \\
              \;\;\;\;S_\sigma(f_1, f_2), S_\sigma(g_1, g_2). \\
              \;\;\;\;f = \lambda x. (f_1(x), g_1(x)) \wedge \\
              \;\;\;\;g = \lambda x. (f_2(x), g_2(x)) \\
            \forall f_1, f_2.
              & : \tau = \sigma \rightarrow \rho \\
              \;\;\;\;S_\sigma(f_1, f_2) \Rightarrow \\
              \;\;\;\;S_\rho(\lambda x. f(x)(f_1(x)),\lambda x. f(x)(f_2(x)))
          \end{array}
        \right.
    \label{eqn:lr_base}
    \end{equation}
  \end{definition}

  The next step involves proving that syntactically well-typed terms are semantically correct.
  In other words, the relation needs to be proven valid for any term $x_1 : \synR, \dots, x_n : \synR \vdash t : \tau$ and argument function $f : \denR \rightarrow \denR^n$ such that $S_\tau(\llbracket t \rrbracket \circ f, \llbracket \D(t) \rrbracket \circ \D_n \circ f)$.
  To properly instantiate the arguments to the denotation of the macro-applied term, an auxiliary function is needed that pairs each constant with their derivative $0$. So it transforms the argument function $f : \denR \rightarrow \llbracket \synR^n \rrbracket$ into one that supplies both the original input value and its accompanying derivative.
  The full type signature of this auxiliary function is $\D_n : (\denR \rightarrow \llbracket \synR^n \rrbracket) \rightarrow \denR \rightarrow \llbracket \D(\synR)^n \rrbracket$. Note that $\tau^n$ is used as syntactic sugar for $\synRepeat{\tau}{n}$ and is simply a typing context consisting of the type $\tau$ repeated $n$ times.

  \begin{equation}
    \D_n(f, x) =
      \left\{
        \begin{array}{ll}
          f(x) & : n = 0 \\
          ((hd \circ f)(x), \frac{\partial{(hd \circ f)}}{\partial{x}}(x)) :: \D_{n'}(tl \circ f, x) & : n = n' + 1 \\
        \end{array}
      \right.
  \label{eqn:argument_df}
  \end{equation}

  Proving this statement directly by induction on the typing derivation, however, does not work.
  As expected in a logical relations proof, the indicative issue lies in both the case for applications and abstractions.
  To make this work, the correctness statement needs to be generalized to arbitrary contexts and implicitly, substitutions.
  If this were a syntactic proof, one would need to show that the relation is preserved when applying substitutions consisting of arbitrary terms, possibly containing higher-order constructs.
  In this style of proof, however, the same concept needs to be formulated in a denotational manner.

  The key in formulating these denotationally lies in the argument function $f : \denR \to \denR^n$.
  Previously the function was used to indicate the open variables or function arguments.
  Generalized to $\Gamma = x_1 : \tau_1, \dots, x_n : \tau_n$, this function can be interpreted as supplying for each open variable $x_1, \dots, x_n$ a corresponding denotated term with type $\llbracket \tau_1 \rrbracket, \dots, \llbracket \tau_n \rrbracket$.
  So the argument function now becomes the pair of functions $s : R \rightarrow \llbracket \Gamma \rrbracket$ and $s_D : R \rightarrow \llbracket \D(\Gamma) \rrbracket$, which intuitively speaking, form the denotational counterparts of syntactic substitutions.
  Notably, for the functions $s$ and $s_D$ to be valid with respect to the logical relation, they are required to be built from the denotations of terms such that these denotations also follow the logical relation.
  We phrase this requirement as a definition.

  \begin{definition}(Instantiation)
    Instantiation functions $s : R \rightarrow \llbracket \Gamma \rrbracket$ and $s_D : R \rightarrow \llbracket \D(\Gamma) \rrbracket$ are inductively defined such that they follow
    \begin{equation}
      inst_\Gamma(f, g) =
        \left\{
          \begin{array}{ll}
            f = (\lambda x. \denNil) \wedge g = (\lambda x. \denNil)
              & : \Gamma = \denNil \\
            \forall f_1, f_2, g_1, g_2.
              & : \Gamma = (\tau :: \Gamma') \\
              \;\;inst_{\Gamma'}(f_1, g_1) \wedge S_\tau(f_2, g_2) \\
              \;\;\;\; \to f = (\lambda x. f_2(x) :: f_1(x)) \wedge \\
              \;\;\;\;\;\; g = (\lambda x. g_2(x) :: g_1(x))
          \end{array}
        \right.
    \label{eqn:inst_base}
    \end{equation}
  \end{definition}

  Using this notion of instantiations we can now formulate our fundamental lemma. Informally, this states that given correct instantiation functions any well-typed term is semantically correct with respect to the logical relation.

  \begin{lemma}[Fundamental]\label{thm:fundamental_lemma}
    For any well-typed term $\Gamma \vdash t : \tau$, and instantiation functions $s : R \rightarrow \llbracket \Gamma \rrbracket$ and $s_D : R \rightarrow \llbracket \D(\Gamma) \rrbracket$ such that they follow $inst_\Gamma(s, s_D)$, we have that $S_\tau(\llbracket t\rrbracket \circ s, \llbracket \D(t)\rrbracket \circ s_D)$.
  \end{lemma}

  % 
  \begin{proof}

    This is proven by induction on the typing derivation of $t$.
    Unless otherwise specified, the type of $s$ and $s_D$ are respectively $R \rightarrow \llbracket \Gamma \rrbracket$ and $R \rightarrow \llbracket \D(\Gamma) \rrbracket$.
    \begin{enumerate}
      \item (\<var>)

        Prove: $S_\tau(\llbracket var\ v \rrbracket \circ s, \llbracket \D(var\ v) \rrbracket \circ s_D)$.

        Proceed by induction on the type evidence $v$.
        \begin{itemize}
          \item(\<Top>) Base case

          Prove: $S_\tau(\llbracket var\ Top \rrbracket \circ s, \llbracket \D(var\ Top) \rrbracket \circ s_D)$, where $s : R \rightarrow \llbracket \tau :: \Gamma \rrbracket$ and $s_D : R \rightarrow \llbracket \tau :: \Gamma \rrbracket$

          In this case the referenced $\tau$ exists at the top of the list.
          So both $\llbracket var\ Top \rrbracket$ and $\llbracket \D(var\ Top) \rrbracket$ denotate to fetching the top term.
          This is now proven by definition of $inst$, which states that the the term is semantically well-typed.

          \begin{align*}
            S&_\tau(\llbracket var\ Top \rrbracket \circ s, \llbracket \D(var\ Top) \rrbracket \circ s_D) \\
            &\Vdash \text{(Definition of $\D$)} \\
            & S_\tau(\llbracket var\ Top \rrbracket \circ s, \llbracket var\ Top \rrbracket \circ s_D) \\
            &\Vdash \text{(Definition of $\circ$)} \\
            & S_\tau(\lambda x. \llbracket var\ Top \rrbracket (s(x)), \lambda x. \llbracket var\ Top \rrbracket (s_D(x))) \\
            &\Vdash \text{(Definition of $\llbracket\rrbracket$)} \\
            & S_\tau(\lambda x. lookup\ \llbracket Top \rrbracket (s(x)), \lambda x. lookup\ \llbracket Top \rrbracket (s_D(x))) \\
            &\Vdash \text{(Rewrite using $s = \lambda x. hd(s(x))::tl(s(x))$)} \\
            & S_\tau(\lambda x. lookup\ \llbracket Top \rrbracket (hd(s(x))::tl(s(x))), \\
              & \;\;\; \lambda x. lookup\ \llbracket Top \rrbracket (hd(s_D(x))::tl(s_D(x)))) \\
            & \Vdash \text{(Simplify with lookup and $\llbracket Top \rrbracket$)} \\
            & S_\tau(\lambda x. hd(s(x)), \lambda x. hd(s_D(x))) \\
            & \Vdash \text{(By definition of $inst_{\tau::\Gamma}$)} \\
          \end{align*} \qed

          \item(\<Pop>) Induction step

          Prove: $S_\tau(\llbracket var\ (Pop\ v) \rrbracket \circ s, \llbracket \D(var\ (Pop\ v)) \rrbracket \circ s_D)$, where $s : R \rightarrow \llbracket \sigma :: \Gamma \rrbracket$ and $s_D : R \rightarrow \llbracket \D(\sigma :: \Gamma) \rrbracket$.

          Induction hypothesis:
          \begin{enumerate}\label{eqn:subst_ih_var_Pop}
            \item $\forall (f : R \rightarrow \llbracket \Gamma \rrbracket), (g : R \rightarrow \llbracket \D(\Gamma) \rrbracket). \\
            \;\;\;S_\tau(\llbracket var\ v \rrbracket \circ f, \llbracket \D(var\ v) \rrbracket \circ g)$
          \end{enumerate}

          Note that the \<var> term now denotates to ignoring the arbitrary unrelated type $\sigma$ and looking up $v$ in the rest of the list $\Gamma$.
          So $S_\tau(\llbracket var\ v \rrbracket \circ tl \circ s, \llbracket \D(var\ v) \rrbracket \circ tl \circ s_D)$, which is proven using the induction hypothesis by respectively instantiating $f$ and $g$ as $tl \circ s$ and $tl \circ s_D$.

          \begin{align*}
            S&_\tau(\llbracket var\ (Pop\ v) \rrbracket \circ s, \llbracket \D(var\ (Pop\ v)) \rrbracket \circ s_D) \\
            &\Vdash \text{(Definition of $\D$)} \\
            & S_\tau(\llbracket var\ (Pop\ v) \rrbracket \circ s, \llbracket var\ (Pop\ v) \rrbracket \circ s_D) \\
            &\Vdash \text{(Definition of $\circ$)} \\
            & S_\tau(\lambda x. \llbracket var\ (Pop\ v) \rrbracket (s(x)), \lambda x. \llbracket var\ (Pop\ v) \rrbracket (s_D(x))) \\
            &\Vdash \text{(Definition of $\llbracket\rrbracket$)} \\
            & S_\tau(\lambda x. lookup\ \llbracket Pop\ v \rrbracket (s(x)), \lambda x. lookup\ \llbracket Pop\ v \rrbracket (s_D(x))) \\
            &\Vdash \text{(Rewrite using $s = \lambda x. hd(s(x))::tl(s(x))$)} \\
            & S_\tau(\lambda x. lookup\ \llbracket Pop\ v \rrbracket (hd(s(x))::tl(s(x))), \\
              & \;\;\; \lambda x. lookup\ \llbracket Pop\ v \rrbracket (hd(s_D(x))::tl(s_D(x)))) \\
            & \Vdash \text{(Simplify with lookup and $\llbracket Pop\ v \rrbracket$)} \\
            & S_\tau(\lambda x. lookup\ \llbracket v \rrbracket (tl(s(x))), \lambda x. lookup\ \llbracket v \rrbracket (tl(s_D(x)))) \\
            & \Vdash \text{(Use IH. \ref{eqn:subst_ih_var_Pop} with $f = tl(s(x))$ and $g = tl(s_D(x))$)}
          \end{align*} \qed
        \end{itemize}
      \item (\<app>)

        Prove: $S_\tau(\llbracket app\ t_1\ t_2 \rrbracket \circ s, \llbracket \D(app\ t_1\ t_2) \rrbracket \circ s_D)$

        Induction hypotheses:
        \begin{enumerate}
          \item \label{eqn:subst_ih_app1}$S_{\sigma\rightarrow\tau}(\llbracket t_1 \rrbracket \circ s, \llbracket \D(t_1) \rrbracket \circ s_D)$
          \item \label{eqn:subst_ih_app2}$S_{\sigma}(\llbracket t_2 \rrbracket \circ s, \llbracket \D(t_2) \rrbracket \circ s_D)$
        \end{enumerate}

        First it is useful to rewrite the induction hypothesis \ref{eqn:subst_ih_app1} in a more usable format. Rewrite the statement using the definition of $S$ at function types.

        \begin{align*}
          S&_{\sigma\rightarrow\tau}(\llbracket t_1 \rrbracket \circ s, \llbracket \D(t_1) \rrbracket \circ s_D) \\
            & \Vdash \text{(Definition of \circ)} \\
            & S_{\sigma\rightarrow\tau}(\lambda x. \llbracket t_1 \rrbracket(s(x)), \lambda x. \llbracket \D(t_1) \rrbracket(s_D(x))) \\
            & \Vdash \text{(Definition of $S_{\rightarrow}$)} \\
            & \forall f_1, f_2.
              S_{\sigma}(f1, f2) \rightarrow \\
            &S_\tau(\lambda x. (\llbracket t_1 \rrbracket(s(x)))(f_1(x)), \lambda x. (\llbracket \D(t_1) \rrbracket(s_D(x)))(f_2(x)))
        \end{align*}

        The case for \<app> is now proven by applying the induction hypothesis \ref{eqn:subst_ih_app1} for the function term using the induction hypothesis \ref{eqn:subst_ih_app2} for the argument term to satify its premise.

        \begin{align*}
          S&_\tau(\llbracket app\ t_1\ t_2 \rrbracket \circ s, \llbracket \D(app\ t_1\ t_2) \rrbracket \circ s_D) \\
            &\Vdash \text{(Definition of $\D$)}\\
            & S_\tau(\llbracket app\ t_1\ t_2 \rrbracket \circ s, \llbracket app\ \D(t_1)\ \D(t_2) \rrbracket \circ s_D) \\
            &\Vdash \text{(Definition of \circ)}\\
            & S_\tau(\lambda x. \llbracket app\ t_1\ t_2 \rrbracket (s (x)), \lambda x. \llbracket app\ \D(t_1)\ \D(t_2) \rrbracket (s_D (x))) \\
            &\Vdash \text{(Definition of $\llbracket \rrbracket$)}\\
            & S_\tau(\lambda x. (\llbracket t_1\ \rrbracket(s(x))) (\llbracket t_2 \rrbracket(s(x))),\lambda x. (\llbracket \D(t_1)\ \rrbracket(s_D(x))) (\llbracket \D(t_2) \rrbracket(s_D(x))) \\
            &\Vdash \text{(Induction hypothesis \ref{eqn:subst_ih_app1})}\\
            & S_{\sigma}(\lambda x. \llbracket t_2 \rrbracket (s(x)), \lambda x. \llbracket \D(t_2) \rrbracket \circ (s_D(x))) \\
            &\Vdash \text{(Induction hypothesis \ref{eqn:subst_ih_app2})}
        \end{align*} \qed
      \item (\<abs>)

        Prove: $S_{\sigma\rightarrow\tau}(\llbracket abs\ t \rrbracket \circ s, \llbracket \D(abs\ t) \rrbracket \circ s_D)$

        Induction hypothesis:
        \begin{enumerate}
          \item \label{eqn:subst_ih_abs} $S_\sigma(\llbracket t \rrbracket \circ s, \llbracket \D(t) \rrbracket \circ s_D)$, where $s : R \rightarrow \llbracket \sigma::\Gamma \rrbracket$ and $s_D : R \rightarrow \llbracket \sigma::\Gamma \rrbracket$
        \end{enumerate}

        As is the case for \ref{eqn:subst_ih_app1}, simplify the goal statement using the definition of $S_\rightarrow$. So the proof obligation now becomes.

        Prove: $S_{\tau}(\lambda x. (\llbracket abs\ t \rrbracket (s(x)))(f_1(x)), \lambda x. (\llbracket \D(abs\ t) \rrbracket (s_D(x)))(f_2(x)))$

        Assume:
        \begin{enumerate}
          \item $f_1 : R \rightarrow \llbracket \sigma \rrbracket$
          \item $f_2 : R \rightarrow \llbracket \D(\sigma) \rrbracket$
          \item \label{eqn:subst_ass_abs3} $S_\sigma(f_1, f_2)$
        \end{enumerate}

        The proof proceeds by rewriting the goal until we can apply the induction hypothesis.
        Note that the assumption \ref{eqn:subst_ass_abs3}: $S_\sigma(f_1, f_2)$ ensures that the requirement of $inst_{\sigma::\Gamma}$ in the induction hypothesis \ref{eqn:subst_ih_abs} is satisfied.

        \begin{align*}
          S&_{\tau}(\lambda x. (\llbracket abs\ t \rrbracket (s(x)))(f_1(x)), \lambda x. (\llbracket \D(abs\ t) \rrbracket (s_D(x)))(f_2(x))) \\
            &\Vdash \text{(Definition of $\D$)}\\
            & S_{\tau}(\lambda x. (\llbracket abs\ t \rrbracket (s(x)))(f_1(x)), \lambda x. (\llbracket abs\ \D(t) \rrbracket (s_D(x)))(f_2(x))) \\
            &\Vdash \text{(Definition of $\llbracket \rrbracket$)}\\
            & S_{\tau}(\lambda x. (\llbracket t \rrbracket (f_1(x) :: s(x))), \lambda x. (\llbracket \D(t) \rrbracket (f_2(x) :: s_D(x)))) \\
            &\Vdash \text{(Induction hypothesis \ref{eqn:subst_ih_app1})}
        \end{align*} \qed

      \item (\<rval>)

      Prove: $S_{R}(\llbracket rval\ n \rrbracket \circ s, \llbracket \D(rval\ n) \rrbracket \circ s_D)$

      This is proven by noting that the corresponding denotations of \<rval> are constant functions, which are both smooth and whose derivatives are equal to $0$.

      \begin{align*}
        S&_R(\llbracket rval\ n \rrbracket \circ s, \llbracket \D(rval\ n) \rrbracket \circ s_D) \\
        &\Vdash \text{(Definition of $\D$)}\\
        &S_R(\llbracket rval\ n \rrbracket \circ s, \llbracket tuple\ (rval\ n)\ (rval\ 0) \rrbracket \circ s_D) \\
        &\Vdash \text{(Definition of $\llbracket\rrbracket$)}\\
        &S_R(const\ n, (const\ n, const\ 0)) \\
        &\Vdash \text{(Definition of $S_R$)}\\
        &smooth\ (const\ n) \wedge
          const\ 0 = \sfrac{\partial{const\ n}}{\partial{x}} \\
        &\Vdash \text{(split goals: goal 1)}\\
        &\;\;\;smooth\ (const\ n) \\
        &\;\;\;\Vdash \text{($f(x) = n$ is continuously differentiable)}\\
        &\Vdash \text{(split goals: goal 2)}\\
        &\;\;\;const\ 0 = \sfrac{\partial{const\ n}}{\partial{x}} \\
        &\;\;\;\Vdash \text{(if $f(x) = n$, then $\sfrac{\partial{f}}{\partial{x}} = 0$)}
      \end{align*} \qed
      \item (\<add>)

      Prove: $S_R(\llbracket add\ t_1\ t_2 \rrbracket \circ s, \llbracket \D(add\ t_1\ t_2) \rrbracket \circ s_D)$

      Induction hypotheses:
      \begin{enumerate}
        \item \label{eqn:subst_ih_add1}$S_R(\llbracket t_1 \rrbracket \circ s, \llbracket \D(t_1) \rrbracket \circ s_D)$
        \item \label{eqn:subst_ih_add2}$S_R(\llbracket t_2 \rrbracket \circ s, \llbracket \D(t_2) \rrbracket \circ s_D)$
      \end{enumerate}

      The proof proceeds by simplifying the denotations and proving the smoothness and derivative requirements for $S_R$.

      \begin{align*}
        S&_R(\llbracket add\ t_1\ t_2 \rrbracket \circ s, \llbracket \D(add\ t_1\ t_2) \rrbracket \circ s_D) \\
        &\Vdash \text{(Definition of $\D$)}\\
        &S_R(\llbracket add\ t_1\ t_2 \rrbracket \circ s, \llbracket tuple\ \\
        & \;\;\;(add\ (first\ \D(t_1)) (first\ \D(t_2)))\ \\
        & \;\;\;(add\ (second \D(t_1)) (second \D(t_2)))) \rrbracket \circ s_D) \\
        &\Vdash \text{(Definition of $\llbracket\rrbracket$, using} \\
        & \;\;\;\;\;\;\;\;\; \text{$(d_1, d_1') = \llbracket \D(t_1) \rrbracket s(x)$ and $(d_2, d_2') = \llbracket \D(t_2) \rrbracket s_D(x)$)}\\
        &S_R(\lambda x. d_1(x) + d_2(x), \lambda x. (d_1(x) + d_2(x), d_1'(x) + d_2'(x))) \\
        &\Vdash \text{(Definition of $S_R$)}\\
        & smooth\ (\lambda x. d_1(x) + d_2(x)) \wedge \\
        & \;\;\; \lambda x. d_1'(x) + d_2'(x) = \sfrac{\partial{(\lambda x. d_1'(x) + d_2'(x))}}{\partial{x}} \\
        &\Vdash \text{(split goals: goal 1)}\\
        &\;\;\;smooth\ (\lambda x. d_1(x) + d_2(x)) \\
        &\;\;\;\Vdash
          \text{(Addition is smooth, if subterms are smooth)}\\
        &\;\;\;smooth\ d_1 \wedge smooth\ d_2 \\
        &\;\;\;\Vdash \text{(Induction hypothesis \ref{eqn:subst_ih_add1} for $d_1$ and \ref{eqn:subst_ih_add2} for $d_2$)}\\
        &\Vdash \text{(split goals: goal 2)}\\
        &\;\;\;\lambda x. d_1'(x) + d_2'(x) = \sfrac{\partial{(\lambda x. d_1'(x) + d_2'(x))}}{\partial{x}} \\
        &\;\;\;\Vdash \text{(By definition of taking the derivative of addition)} \\
        &\;\;\; d_1' = \sfrac{\partial{d_1}}{\partial{x}} \wedge d_2' = \sfrac{\partial{d_2}}{\partial{x}} \\
        &\;\;\;\Vdash \text{(Induction hypothesis \ref{eqn:subst_ih_add1} for $d_1$ and \ref{eqn:subst_ih_add2} for $d_2$)}\\
      \end{align*} \qed

      \item (\<mul>)

      Prove: $S_R(\llbracket mul\ t_1\ t_2 \rrbracket \circ s, \llbracket \D(mul\ t_1\ t_2) \rrbracket \circ s_D)$

      Induction hypotheses:
      \begin{enumerate}
        \item \label{eqn:subst_ih_mul1}$S_R(\llbracket t_1 \rrbracket \circ s, \llbracket \D(t_1) \rrbracket \circ s_D)$
        \item \label{eqn:subst_ih_mul2}$S_R(\llbracket t_2 \rrbracket \circ s, \llbracket \D(t_2) \rrbracket \circ s_D)$
      \end{enumerate}

      Proof goes through almost identically as for the case for \<add>.

      \begin{align*}
        S&_R(\llbracket mul\ t_1\ t_2 \rrbracket \circ s, \llbracket \D(mul\ t_1\ t_2) \rrbracket \circ s_D) \\
        &\Vdash \text{(Definition of $\D$)}\\
        &S_R(\llbracket mul\ t_1\ t_2 \rrbracket \circ s, \llbracket tuple\ \\
        & \;\;\;(mul\ (first\ \D(t_1)) (first\ \D(t_2)))\ \\
        & \;\;\;(add\ \\
        & \;\;\;\;\;(mul\ (first \D(t_1)) (second \D(t_2))) \\
        & \;\;\;\;\;(mul\ (first \D(t_2)) (second \D(t_1)))) \rrbracket \circ s_D) \\
        &\Vdash \text{(Definition of $\llbracket\rrbracket$, using} \\
        & \;\;\;\;\;\;\;\;\; \text{$(d_1, d_1') = \llbracket \D(t_1) \rrbracket s(x)$ and $(d_2, d_2') = \llbracket \D(t_2) \rrbracket s_D(x)$)}\\
        &S_R(\lambda x. d_1(x) * d_2(x), \\
        & \;\;\; \lambda x. (d_1(x) * d_2(x), d_1(x) * d_2'(x) + (d_2(x) * d_1'(x)))) \\
        &\Vdash \text{(Definition of $S_R$)}\\
        &smooth\ (\lambda x. d_1(x) * d_2(x)) \wedge \\
        & \;\;\; \lambda x. d_1(x) * d_2'(x) + d_2(x) * d_1'(x) = \sfrac{\partial{(\lambda x. (d_1(x) * d_2(x))}}{\partial{x}} \\
        &\Vdash \text{(split goals: goal 1)}\\
        &\;\;\;smooth\ (\lambda x. d_1(x) * d_2(x)) \\
        &\;\;\;\Vdash
          \text{(Multiplication is smooth, if subterms are smooth)}\\
        &\;\;\;smooth\ d_1 \wedge smooth\ d_2 \\
        &\;\;\;\Vdash \text{(Induction hypothesis \ref{eqn:subst_ih_mul1} for $d_1$ and \ref{eqn:subst_ih_mul2} for $d_2$)}\\
        &\Vdash \text{(split goals: goal 2)}\\
        &\;\;\;\lambda x. d_1(x) * d_2'(x) + d_2(x) * d_1'(x) = \sfrac{\partial{(\lambda x. (d_1(x) * d_2(x))}}{\partial{x}} \\
        &\;\;\;\Vdash \text{(By definition of taking the derivative of multiplications)} \\
        &\;\;\; d_1' = \sfrac{\partial{d_1}}{\partial{x}} \wedge d_2' = \sfrac{\partial{d_2}}{\partial{x}} \\
        &\;\;\;\Vdash \text{(Induction hypothesis \ref{eqn:subst_ih_mul1} for $d_1$ and \ref{eqn:subst_ih_mul2} for $d_2$)}\\
      \end{align*} \qed

      \item (\<tuple>)

      Prove: $S_(\tau \times \sigma)(\llbracket tuple\ t_1\ t_2 \rrbracket \circ s, \llbracket \D(tuple\ t_1\ t_2) \rrbracket \circ s_D)$

      Induction hypotheses:
      \begin{enumerate}
        \item \label{eqn:subst_ih_tuple1}$S_\tau(\llbracket t_1 \rrbracket \circ s, \llbracket \D(t_1) \rrbracket \circ s_D)$
        \item \label{eqn:subst_ih_tuple2}$S_\sigma(\llbracket t_2 \rrbracket \circ s, \llbracket \D(t_2) \rrbracket \circ s_D)$
      \end{enumerate}

      A recurring pattern will become apparent in later sections when continuing to prove the substitution lemma \ref{thm:substitution_lemma} for types consisting of other types.
      In this case, due to the carefull attention spent on the logical relation, only the witnesses of the subterms of the tuple need to be supplied to finish the proof.

      Note that the witnesses of $S_\tau$ and $S_\sigma$ that need to be given here are supplied by the induction hypotheses.
      While these witnesses are not exactly relevant to finish this proof for \<tuple>, they are needed in the proofs for projections.

      \begin{align*}
        S&_(\tau \times \sigma)(\llbracket tuple\ t_1\ t_2 \rrbracket \circ s, \llbracket \D(tuple\ t_1\ t_2) \rrbracket \circ s_D) \\
        & \Vdash \text{(Definition of $\D$)} \\
        & S_(\tau \times \sigma)(\llbracket tuple\ t_1\ t_2 \rrbracket \circ s, \llbracket tuple\ \D(t_1)\ \D(t_2)) \rrbracket \circ s_D) \\
        & \Vdash \text{(Definition of $\llbracket\rrbracket$)} \\
        & S_(\tau \times \sigma)(\lambda x. (\llbracket t_1 \rrbracket(s(x)), \llbracket t_2 \rrbracket(s(x))), \\
        & \;\;\;\;\;\;\lambda x. (\llbracket \D(t_1) \rrbracket(s'(x)), \llbracket \D(t_2) \rrbracket(s'(x)))) \\
        & \Vdash \text{(Definition of $S_{\tau\times\sigma}$)} \\
        & \exists f_1, f_2, g_1, g_2, \\
            & \;\;\;\;S_\tau(f_1, f_2), S_\sigma(g_1, g_2). \\
            & \;\;\;\;\lambda x. (\llbracket t_1 \rrbracket(s(x)), \llbracket t_2 \rrbracket(s(x))) = \lambda x. (f_1(x), g_1(x)) \wedge \\
            & \;\;\;\;\lambda x. (\llbracket \D(t_1) \rrbracket(s'(x)), \llbracket \D(t_2) \rrbracket(s'(x))) = \lambda x. (f_2(x), g_2(x)) \\
        & \Vdash \text{(Give witnesses: $f_1 := \llbracket t_1 \rrbracket \circ s$, $f_2 := \llbracket t_2 \rrbracket \circ s$,} \\
        & \;\;\;\;\;\; \text{$g_1 := \llbracket \D(t_1) \rrbracket \circ s'$, $g_2 := \llbracket \D(t_2) \rrbracket \circ s'$)} \\
        & \exists S_\tau(f_1, f_2), S_\sigma(g_1, g_2). \\
          & \;\;\;\;\lambda x. (\llbracket t_1 \rrbracket(s(x)), \llbracket t_2 \rrbracket(s(x))) \\
          & \;\;\;\;\;\;\; = \lambda x. (\llbracket t_1 \rrbracket(s(x)), \llbracket t_2 \rrbracket(s(x))) \wedge \\
          & \;\;\;\;\lambda x. (\llbracket \D(t_1) \rrbracket(s'(x)), \llbracket \D(t_2) \rrbracket(s'(x))) \\
          & \;\;\;\;\;\;\; = \lambda x. (\llbracket \D(t_1) \rrbracket(s'(x)), \llbracket \D(t_2) \rrbracket(s'(x))) \\
        & \Vdash \text{(Give witnesses of $S_\tau$ and $S_\sigma$ using respective IHs \ref{eqn:subst_ih_tuple1} and \ref{eqn:subst_ih_tuple2})} \\
        & \;\;\;\;\lambda x. (\llbracket t_1 \rrbracket(s(x)), \llbracket t_2 \rrbracket(s(x))) \\
        & \;\;\;\;\;\;\; = \lambda x. (\llbracket t_1 \rrbracket(s(x)), \llbracket t_2 \rrbracket(s(x))) \wedge \\
        & \;\;\;\;\lambda x. (\llbracket \D(t_1) \rrbracket(s'(x)), \llbracket \D(t_2) \rrbracket(s'(x))) \\
        & \;\;\;\;\;\;\; = \lambda x. (\llbracket \D(t_1) \rrbracket(s'(x)), \llbracket \D(t_2) \rrbracket(s'(x))) \\
        & \Vdash \text{(Reflexivity)} \\
      \end{align*}\qed
      \item (\<first>)

      Prove: $S_(\tau)(\llbracket first\ t \rrbracket \circ s, \llbracket \D(first\ t) \rrbracket \circ s_D)$

      Induction hypotheses:
      \begin{enumerate}
        \item \label{eqn:subst_ih_first}$S_{\tau\times\sigma}(\llbracket t \rrbracket \circ s, \llbracket \D(t) \rrbracket \circ s_D)$
      \end{enumerate}

      Simplifying the induction hypothesis \ref{eqn:subst_ih_first} using the definition of $S_{\tau\times\sigma}$ gives rise to a number of useful assumptions containing:
      $f_1 : R \rightarrow \llbracket \tau \rrbracket$
      , $f_2 : R \rightarrow \llbracket \D(\tau) \rrbracket$
      , $g_1 : R \rightarrow \llbracket \sigma \rrbracket$
      and $g_2 : R \rightarrow \llbracket \D(\sigma) \rrbracket$.

      Assumptions:
      \begin{enumerate}
        \item \label{eqn:subst_ass_proj1_4} $S_\tau(f_1, f_2)$
        \item \label{eqn:subst_ass_proj1_5} $S_\sigma(g_1, g_2)$
        \item \label{eqn:subst_ass_proj1_6} $\llbracket t \rrbracket \circ s = \lambda x. (f_1(x), g_1(x))$
        \item \label{eqn:subst_ass_proj1_7} $\llbracket \D(t) \rrbracket \circ s = \lambda x. (f_2(x), g_2(x))$
      \end{enumerate}

      \begin{align*}
        S&_{\tau}(\llbracket first\ t \rrbracket \circ s, \llbracket \D(first\ t) \rrbracket \circ s_D) \\
        & \Vdash \text{(Rewrite using definition of $\D$)} \\
        & S_{\tau}(\llbracket first\ t \rrbracket \circ s, \llbracket first\ \D(t) \rrbracket \circ s_D) \\
        & \Vdash \text{(Rewrite using definition of $\llbracket\rrbracket$)} \\
        & S_{\tau}(\lambda x. fst(\llbracket t \rrbracket(s(x))), \lambda x. fst(\llbracket \D(t) \rrbracket(s_D(x)))) \\
        & \Vdash \text{(Rewrite using \ref{eqn:subst_ass_proj1_6} and \ref{eqn:subst_ass_proj1_7})} \\
        & S_{\tau}(\lambda x. fst(f_1(x), g_1(x)), \lambda x. fst(f_2(x), g_2(x))) \\
        & \Vdash \text{($\beta\eta$-equality)} \\
        & S_{\tau}(f_1, f_2) \\
        & \Vdash \text{(Assumption \ref{eqn:subst_ass_proj1_4})} \\
      \end{align*} \qed

      \item (\<second>)

      Prove: $S_(\tau)(\llbracket first\ t \rrbracket \circ s, \llbracket \D(first\ t) \rrbracket \circ s_D)$

      Induction hypotheses:
      \begin{enumerate}
        \item \label{eqn:subst_ih_first}$S_{\tau\times\sigma}(\llbracket t \rrbracket \circ s, \llbracket \D(t) \rrbracket \circ s_D)$
      \end{enumerate}

      Proof goes through the same as the case for \<first> with the same assumptions following from the induction hypothesis, where
      $f_1 : R \rightarrow \llbracket \tau \rrbracket$
      , $f_2 : R \rightarrow \llbracket \D(\tau) \rrbracket$
      , $g_1 : R \rightarrow \llbracket \sigma \rrbracket$
      and $g_2 : R \rightarrow \llbracket \D(\sigma) \rrbracket$.

      Assumptions:
      \begin{enumerate}
        \item \label{eqn:subst_ass_proj2_4} $S_\tau(f_1, f_2)$
        \item \label{eqn:subst_ass_proj2_5} $S_\sigma(g_1, g_2)$
        \item \label{eqn:subst_ass_proj2_6} $\llbracket t \rrbracket \circ s = \lambda x. (f_1(x), g_1(x))$
        \item \label{eqn:subst_ass_proj2_7} $\llbracket \D(t) \rrbracket \circ s = \lambda x. (f_2(x), g_2(x))$
      \end{enumerate}

      \begin{align*}
        S&_{\sigma}(\llbracket second\ t \rrbracket \circ s, \llbracket \D(second\ t) \rrbracket \circ s_D) \\
        & \Vdash \text{(Rewrite using definition of $\D$)} \\
        & S_{\sigma}(\llbracket second\ t \rrbracket \circ s, \llbracket second\ \D(t) \rrbracket \circ s_D) \\
        & \Vdash \text{(Rewrite using definition of $\llbracket\rrbracket$)} \\
        & S_{\sigma}(\lambda x. snd(\llbracket t \rrbracket(s(x))), \lambda x. snd(\llbracket \D(t) \rrbracket(s_D(x)))) \\
        & \Vdash \text{(Rewrite using \ref{eqn:subst_ass_proj2_6} and \ref{eqn:subst_ass_proj2_7})} \\
        & S_{\sigma}(\lambda x. snd(f_1(x), g_1(x)), \lambda x. snd(f_2(x), g_2(x))) \\
        & \Vdash \text{($\beta\eta$-equality)} \\
        & S_{\sigma}(f_1, f_2) \\
        & \Vdash \text{(Assumption \ref{eqn:subst_ass_proj2_4})} \\
      \end{align*} \qed
    \end{enumerate}
  \end{proof}

  The proof of the fundamental property of the logical relation now follows from the substitution lemma.

  \begin{lemma}[Fundamental property]\label{thm:fundamental_property}
    For any well-typed term $x_1 : R, \dots, x_n : R \vdash t : \tau$, and argument function $f : R \rightarrow \llbracket R^n \rrbracket$, such that each argument is continuously derivable, then $S_\tau(\llbracket t\rrbracket \circ f, \llbracket \D(t)\rrbracket \circ \D_n \circ f)$.
  \end{lemma}

  \begin{proof}
    This is proven by instantiating the substitution lemma \ref{thm:substitution_lemma} with the proper variables and proving the resulting judgement of $inst_{R^n}$ by induction on $n$.

    \begin{align*}
      S&_{\tau}(\llbracket t \rrbracket \circ f, \llbracket \D(t) \rrbracket \circ \D_n \circ f) \\
      & \Vdash \text{(Apply substitution lemma with $s := f$, $s_D := \D_n \circ f$ and $\Gamma := R^n$)} \\
      & inst_{R^n}(f, \D_n \circ f) \\
    \end{align*}

    Proceed by induction on $n$, intuitively building up the environment with denotation of terms such that they follow $S$.

    \begin{itemize}
      \item Base case: $n = 0$

      \begin{align*}
        inst&_{R^0}(f, \D_0 \circ f) \\
        & \Vdash \text{(Induction on n, base case $n = 0$)} \\
        & \;\;\; inst_{[]}(f, \D_0 \circ f) \\
        & \;\;\; \Vdash \text{(Singleton instance of $R \rightarrow R^0$, $f = const([])$)} \\
        & \;\;\; inst_{[]}(const([]), \D_0 \circ const([])) \\
        & \;\;\; \Vdash \text{(Definition of $\D_0$)} \\
        & \;\;\; inst_{[]}(const([]), const([])) \\
        & \;\;\; \Vdash \text{(Definition of $inst_{[]}$)}
      \end{align*}

      \item Induction case: $n = S(n')$

      Induction hypothesis: $inst_{R^{n'}}(f', D \circ f')$, where
        $f' : R \rightarrow \llbracket R^{n'}\rrbracket$.

      \begin{align*}
        inst&_{R :: R^{n'}}(f, \D_{R :: R^{n'}} \circ f) \\
        & \;\;\; \Vdash \text{(Unfold $\circ$)} \\
        & \;\;\; inst_{R :: R^{n'}}(\lambda x. f(x), \lambda x. \D_{R :: R^{n'}}( f, x)) \\
        & \;\;\; \Vdash \text{(Rewrite using $f = \lambda x. hd(f(x)) :: tl(f(x))$} \\
        & \;\;\;\;\;\;\;\;\;\;\;\; \text{and definition of $\D_{R :: R^{n'}}$)} \\
        & \;\;\; inst_{R :: R^{n'}}(\lambda x. hd(f(x)) :: tl(f(x)), \\
        & \;\;\;\;\;\;
          \lambda x. (hd(f(x)), \sfrac{\partial{(hd \circ f)}}{\partial{x}}(x)) :: \D_{R^{n'}}(tl \circ f, x)) \\
        & \;\;\; \Vdash \text{(By definition of $inst_{R :: R^{n'}}$, rest proven by IH)}
      \end{align*}
    \end{itemize}
  \end{proof}

  \begin{proof}
    This is proven by induction on the typing derivation of the well-typed term $t$. The majority of cases follow from the induction hypothesis.
    The case for \<var> follows from $inst$ which ensures that any term referenced is semantically well-typed with respect to the relation.
    Proving the cases used to encode the operators on reals such as \<add> and \<mul> involve proving both smoothness and giving the witness of the derivative.
  \end{proof}

  We can derive the fundamental property of the base logical relation directly from the fundamental lemma.
  This involves proving the prerequisite $inst$ we used previously.
  Note that the correctness of both the macro and the fundamental property is dependent on the requirement that the denotations supplied by the argument function are smooth.

  \begin{corollary}[Fundamental property]\label{thm:fundamental_property}
    For any term $x_1 : \synR, \dots, x_n : \synR \vdash t : \synR$, $\llbracket\D(t)\rrbracket$ gives the dual number representation of $\llbracket t \rrbracket$, such that for any argument function $f : \denR \to \denR^n$, we have that $S_\tau(\llbracket t\rrbracket \circ f, \llbracket \D(t)\rrbracket \circ \D_n \circ f)$.
  \end{corollary}

  \begin{proof}
    This follows from the fundamental lemma. We lastly need to prove $\inst{(\synRepeat{\synR}{n})}$.
    This is proven by induction on $n$.
    If $n = 0$, the goal is trivial due to the argument function $f$ being extensionally equal to $\synConst{[]}$, which directly corresponds to $\inst{[]}$.
    The induction step is proven by both the induction hypothesis and the assumption that the denotations of the arguments supplied are smooth.
  \end{proof}

  \begin{theorem}[Macro correctness]\label{thm:macro_correctness}
    For any term $x_1 : \synR, \dots, x_n : \synR \vdash t : \synR$, $\llbracket\D(t)\rrbracket$ gives the dual number representation of $\llbracket t \rrbracket$, such that for any argument function $f : \denR \rightarrow \denR^n$, we have that $\llbracket \D(t) \rrbracket \circ \D_n \circ f = \lambda x. (\llbracket t \rrbracket \circ f, \sfrac{\partial{(\llbracket t \rrbracket \circ f)}}{\partial{x}})$.
  \end{theorem}

  \begin{proof}
    This is proven by showing that the goal follows from the logical relation which itself is implied by the fundamental property.

    \begin{align*}
      \llbracket \D(t) &\rrbracket \circ \D_n \circ f = \lambda x. (\llbracket t \rrbracket \circ f, \sfrac{\partial{(\llbracket t \rrbracket \circ f)}}{\partial{x}}) \\
      & \Vdash \text{(By definition of $S_R$ with $f := \llbracket t \rrbracket \circ f$ and $g := \llbracket \D(t) \rrbracket \circ \D_n \circ f$)} \\
      & S_R(\llbracket t \rrbracket \circ f, \llbracket \D(t) \rrbracket \circ \D_n \circ f) \\
      & \Vdash \text{(Fundamental property (\ref{thm:fundamental_property}))}
    \end{align*}
  \end{proof}

  \subsection{Adding Sums and Primitive Recursion}
  Now that correctness has been verified for the base simply-typed lambda calculus, the next goal will be to add in both sum and integer types.
  In the interest of testing the flexibility of both the representation and the proofing technique, integer types and primitive recursion were also added.
  The inference rules for the new language constructs added for sum and number types are given in figure~\ref{fig:sum_prim_infer}.

  \begin{figure}[H]
    \begin{mathpar}
      \inferrule*[Right=\textsc{TCase}]
        {\Gamma \vdash e : \tau + \sigma \\
          \Gamma \vdash t1 : \tau \rightarrow \rho \\
          \Gamma \vdash t2 : \sigma \rightarrow \rho }
        {\Gamma \vdash \case{e}{t1}{t2} : \rho} \\ \and
      \inferrule*[Right=\textsc{TInl}]
        {\Gamma \vdash t : \tau}
        {\Gamma \vdash \inl{t} : \tau + \sigma} \and
      \inferrule*[Right=\textsc{TInr}]
        {\Gamma \vdash t : \sigma}
        {\Gamma \vdash \inr{t} : \tau + \sigma} \\ \and
      \inferrule*[Right=\textsc{TNVal}]
        {n \in \denoteN}
        {\Gamma \vdash \nval{n} : \<N>} \and
      \inferrule*[Right=\textsc{TNSucc}]
        {\Gamma \vdash t : \<N>}
        {\Gamma \vdash \nsucc{t} : \<N>} \\ \and
      \inferrule*[Right=\textsc{TPrim}]
        {\Gamma \vdash f : \tau \rightarrow \tau \\
          \Gamma \vdash n : \<N> \\
          \Gamma \vdash t : \tau }
        {\Gamma \vdash \nrec{f}{n}{t} : \tau}
    \end{mathpar}
    \caption{Type-inference rules for language constructs for sum types and primitive recursion}
    \label{fig:sum_prim_infer}
  \end{figure}

  \begin{minted}{coq}
    Inductive tm ~($\Gamma$ : Ctx) : ty $\rightarrow$ Type~ :=
      ...
      (* Sums *)
      | case : ~forall {$\tau$ $\sigma$ $\rho$},
        tm $\Gamma$ ($\tau$ <+> $\sigma$) $\rightarrow$
        tm $\Gamma$ ($\tau \Rightarrow \rho$) $\rightarrow$
        tm $\Gamma$ ($\sigma \Rightarrow \rho$) $\rightarrow$
        tm $\Gamma$ $\rho$~
      | inl : ~forall {$\tau$ $\sigma$},
        tm $\Gamma$ $\tau$ $\rightarrow$ tm $\Gamma$ ($\tau$ <+> $\sigma$)~
      | inr : ~forall {$\tau$ $\sigma$},
        tm $\Gamma$ $\sigma$ $\rightarrow$ tm $\Gamma$ ($\tau$ <+> $\sigma$)~
  \end{minted}

  Binary sum types are included in the language using \<inl> and \<inr> as introducing terms.
  The \<case> term encodes case-analysis given a function term for each possibility.
  Primitive recursion is implemented using simple integers, where a endomorphic function is recursively applied a bounded number of times to a start value.
  For convenience, an additional successor function is added in the form of the \<nsucc> term.

  \begin{minted}{coq}
    Inductive tm ~($\Gamma$ : Ctx) : ty $\rightarrow$ Type~ :=
      ...
      (* Primitive recursion *)
      | nval : ~forall n, tm \Gamma N~
      | nsucc : ~tm $\Gamma$ N $\rightarrow$ tm $\Gamma$ N~
      | nrec : ~forall $\tau$,
        tm $\Gamma$ ($\tau \Rightarrow \tau$) $\rightarrow$ tm $\Gamma$ N $\rightarrow$ tm $\Gamma$ $\tau$ $\rightarrow$ tm $\Gamma$ $\tau$~
  \end{minted}

  In terms of denotations, \<case> expressions will follow the same lines as \<app> as they both involve applying a function to an argument.
  Note that the sum term first needs to be destructed to be able to determine which function branch to apply.
  Both \<inl> and \<inr> will map to their \<Coq> counterparts.
  For \<nrec>, the number of applications should be dependent on the input integer.

  \begin{figure}
    \centering
    \begin{equation*}
      \begin{split}
        \llbracket \tau \text{<+>} \sigma \rrbracket &= \llbracket \tau \rrbracket + \llbracket \sigma \rrbracket \\
        \llbracket N \rrbracket &= \denoteN \\
        \\
        \llbracket \case{e}{t_1}{t_2} \rrbracket &=
          \left\{
            \begin{array}{ll}
              \llbracket t_1 \rrbracket(t)
                & : \llbracket e \rrbracket = inl(t) \\
              \llbracket t_2 \rrbracket(t)
                & : \llbracket e \rrbracket = inr(t) \\
            \end{array}
          \right. \\
        \llbracket \inl{t} \rrbracket &= inl(\llbracket t \rrbracket) \\
        \llbracket \inr{t} \rrbracket &= inr(\llbracket t \rrbracket) \\
        \llbracket \nval{n} \rrbracket &= n \\
        \llbracket \nsucc{t} \rrbracket &= n + 1 \\
        \llbracket \nrec{f}{i}{t} \rrbracket &= fold(\llbracket f \rrbracket, \llbracket i \rrbracket, \llbracket t \rrbracket) \\
        fold(f, i, t) &=
          \left\{
            \begin{array}{ll}
              t &: i = 0 \\
              f(fold(f, i', t))
                &: i = i' + 1 \\
            \end{array}
          \right. \\
      \end{split}
    \end{equation*}
    \caption{Denotations of the sum and integer terms}
    \label{eqn:denotation_base}
  \end{figure}


  \subsection{Arrays}
\section{Optimization}
  \subsection{Program Transformations}
\section{Reverse-Mode AD}
\section{Discussion}
  \subsection{Problems}
  \subsection{Future Work}

% TODO: non-termination and partiality are left out of the scope of this thesis.
% \<Coq> has very limited support for domain theoretical models.
% There are possible libraries which have resulted from experiments trying to encode domain theoretical models\cite{Benton2009}\cite{Dockins2014}, but these are incompatible with recent versions of \<Coq>.
\section{Conclusion}

\appendix
\section{Language Definitions}
\section{Forward-Mode Macro}
\section{Denotations}
\printbibliography
\makeatother
\end{document}