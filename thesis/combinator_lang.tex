\subsection{Core Combinator Language}\label{sec:combinator-core}
  A well-known fact is the connection between CCC and simply-typed lambda calculi\cite{10.1007/3-540-15198-2_10}.
  We can define a simple core combinator language inspired by the various categorical laws related to CCC.
  The requirement for a combinator language to be able to do reverse-mode AD comes from the need to make the contraction and weakening rules usually kept implicit in the typing contexts of typed lambda calculi, explicit.
  Translating from a simply-typed lambda calculus necessitates a translation between the implicit manipulation of the typing context to access variables, to one that is explicit in its usage of specific combinators.

  The core combinator language we will be using is shown in \cref{fig:combinator_core_lang}.
  Note the combinator language contains no typing context and are defined as terms $c : \comb{\tau}{\sigma}$, where $\tau$ and $\sigma$ are, respectively, the input and output types of the combinator $c$.
  As programming in the combinator language can quickly become cumbersome and unreadable, we make repeated usage of some syntactic niceties.
  We interchangeably use $\synSeq{}{}$ as the infix version of \<seq> and $\langle A, B\rangle$ as a shortcut for $\synSeq{\synDupl}{\synCross{A}{B}}$.
  We also use some helper functions for moving elements around in products, as shown below.

  \begin{figure}[]
    \centering
    \begin{minted}{coq}
    Inductive ty : Type :=
      | ~$\synR\texttt{\^{}}$~ : nat ~$\rightarrow$~ ty
      | ~$\synR$~ : ty
      | ~$\synUnit$~ : ty
      | ~$\synFunc$~ : ty ~$\rightarrow$~ ty ~$\rightarrow$~ ty
      | ~$\synStar$~  : ty ~$\rightarrow$~ ty ~$\rightarrow$~ ty
    .

    Inductive comb : ty ~$\rightarrow$~ ty ~$\rightarrow$~ Type :=
      (* Category laws *)
      | id : forall A, comb A A
      | seq : forall A B C,
        comb A B ~$\rightarrow$~ comb B C ~$\rightarrow$~ comb A C
      (* Monoidal *)
      | cross : forall A B C D,
        comb A B ~$\rightarrow$~ comb C D ~$\rightarrow$~ comb (A ~$\synStar$~ C) (B ~$\synStar$~ D)
      (* Terminal *)
      | neg : forall A,
        comb A ~$\synUnit$~
      (* Cartesian *)
      | exl : forall A B,
        comb (A ~$\synStar$~ B) A
      | exr : forall A B,
        comb (A ~$\synStar$~ B) B
      | dupl : forall A,
        comb A (A ~$\synStar$~ A)
      (* Closed *)
      | ev : forall A B,
        comb ((A ~$\synFunc$~ B) ~$\synStar$~ A) B
      | curry : forall A B C,
        comb (A ~$\synStar$~ B) C ~$\rightarrow$~ comb A (B ~$\synFunc$~ C)
      (* Reals *)
      | cplus : comb (~$\synR$~ ~$\synStar$~ ~$\synR$~) ~$\synR$~
      | crval : forall (r : ~$\denR$~), comb ~$\synUnit$~ ~$\synR$~
      | cmplus : forall n, comb (~$\synRn$~ ~$\synStar$~ ~$\synRn$~) (~$\synRn$~)
      | cmrval : forall n (a : vector ~$\denR$~ n), comb ~$\synUnit$~ (~$\synRn$~)
    .
    \end{minted}
    \caption{Core combinator language inspired by Cartesian closed categories}
    \label{fig:combinator_core_lang}
  \end{figure}

  \begin{minted}{coq}
    Definition assoc1 {A B C} : comb ((A ~$\synStar$~ B) ~$\synStar$~ C) (A ~$\synStar$~ (B ~$\synStar$~ C)) :=
      ~$\langle$~ exl;;exl, ~$\langle$~ exl;;exr, exr ~$\rangle$~ ~$\rangle$~.
    Definition assoc2 {A B C} : comb (A ~$\synStar$~ (B ~$\synStar$~ C)) ((A ~$\synStar$~ B) ~$\synStar$~ C) :=
      ~$\langle$~ ~$\langle$~ exl, exr;;exl ~$\rangle$~, exr;;exr ~$\rangle$~.
    Definition sym {A B} : comb (A ~$\synStar$~ B) (B ~$\synStar$~ A) :=
      ~$\langle$~ exr, exl ~$\rangle$~.
  \end{minted}

  We will next describe a translation from a simply-typed lambda calculus to this combinator language.
  Like the combinator language, the simply-typed lambda calculus will be restricted to function types, product types, and types for real numbers, vectors specialized to real numbers and unit, which is shown in \cref{fig:stlc_combinator}.
  Note that we omitted the terms related to typing contexts, function types and product types, as these were identical to the ones used in \cref{sec:forward}.

  \begin{figure}
    \centering
    \begin{minted}{coq}
      Inductive tm ~($\Gamma$ : Ctx) : ty $\to$ Type~ :=
        ...
        (* Operations on reals *)
        | rval : ~forall r, \rightarrow tm \Gamma \synR~
        | plus :
          ~tm \Gamma \synR \rightarrow tm \Gamma \synR \rightarrow tm \Gamma \synR~
        (* Operations on real vectors *)
        | mrval : ~forall n, vector $\denR$ n \rightarrow tm \Gamma ($\synRn$)~
        | mplus : forall n,
          ~tm \Gamma ($\synRn$) \rightarrow tm \Gamma ($\synRn$) \rightarrow tm \Gamma ($\synRn$)~
        (* $\synUnit$ *)
        | it : ~tm \Gamma \synUnit~
      .
    \end{minted}
    \caption{Simply-typed lambda calculus with unit and specialized real arrays}
    \label{fig:stlc_combinator}
  \end{figure}

  Using the same technique as Curien\cite{10.1007/3-540-15198-2_10}, we make use of an auxiliary language to smoothen the process.
  This auxiliary language, the typed categorical combinatory logic \textit{(CCL)}, will contain terms related to both the combinator language and the simply-typed lambda calculus.
  So it will also include usage of a typing context and related constructs such as variable access and function abstraction.
  While this CCL can be used to facilitate both back and forth translations, we will restrict our focus to only translating from the combinator language to the simply-typed lambda calculus.

  The \cclenv{} combinator in CCL is most notable.
  In essence we transition each type in the typing context to become an additional argument in the resulting function type.
  Repeated usage of this specific combinator turns any previously open term into a closed one.

  \begin{figure}[]
    \centering
    \begin{minted}{coq}
    Inductive ccl (~$\Gamma$~ : Ctx) : ty ~$\rightarrow$~ Type :=
      (* Variables *)
      | ccl_var : forall ~$\tau$~,
        ~$\tau$~ ~$\in$~ ~$\Gamma$~ ~$\rightarrow$~ ccl ~$\tau$~
      (* Reals *)
      | ccl_plus : ccl (~$\synR$~ ~$\synStar$~ ~$\synR$~ ~$\synFunc$~ ~$\synR$~)
      | ccl_rval : ~$\denR$~ ~$\rightarrow$~ ccl ~$\synR$~
      | ccl_mplus : forall n,
        ccl (~$\synRn$~ ~$\synStar$~ ~$\synRn$~ ~$\synFunc$~ ~$\synRn$~)
      | ccl_mrval : forall n, vector ~$\denR$~ n ~$\rightarrow$~ ccl (~$\synRn$~)
      (* Category laws *)
      | ccl_id : forall A, ccl (A ~$\synFunc$~ A)
      | ccl_seq : forall A B C,
        ccl (A ~$\synFunc$~ B) ~$\rightarrow$~ ccl (B ~$\synFunc$~ C) ~$\rightarrow$~ ccl (A ~$\synFunc$~ C)
      (* Cartesian *)
      | ccl_exl : forall A B,
        ccl (A ~$\synStar$~ B ~$\synFunc$~ A)
      | ccl_exr : forall A B,
        ccl ((A ~$\synStar$~ B) ~$\synFunc$~ B)
      (* Monoidal *)
      | ccl_cross : forall A B C,
        ccl (A ~$\synFunc$~ B) ~$\rightarrow$~ ccl (A ~$\synFunc$~ C) ~$\rightarrow$~ ccl (A ~$\synFunc$~ B ~$\synStar$~ C)
      (* Closed *)
      | ccl_ev : forall A B,
        ccl ((A ~$\synFunc$~ B) ~$\synStar$~ A ~$\synFunc$~ B)
      | ccl_env : forall A B C,
        @ccl (A::~$\Gamma$~) (B ~$\synFunc$~ C) ~$\rightarrow$~ @ccl ~$\Gamma$~ ((B ~$\synFunc$~ A ~$\synFunc$~ C))
      (* Const *)
      | ccl_const : forall A B,
        ccl A ~$\rightarrow$~ ccl (B ~$\synFunc$~ A)
    .
    \end{minted}
    \caption{Auxilliary categorical combinatory logic language used in the translations}
    \label{fig:combinator_ccl_lang}
  \end{figure}

  The intrinsic representation we use in our definitions makes defining the exact translations a breeze as it then becomes an exercise in type-directed programming.
  The simply-typed lambda calculus to CCL translation specifically is straightforward, as the CCL language still has access to a typing context.
  We translate abstractions, where the argument type is added onto the typing context, using the \cclenv{} construct.
  In cases where we introduce new values, we make use of the \cclconst{} construct to ensure the terms fit the type signature.
  Note that an additional domain type of $\synUnit$ is added in the type signature of the translation function to accommodate the combinator-heavy auxiliary language.
  This translation is shown in \cref{lst:combinator_stlc_to_ccl}.

  \begin{listing}
    \begin{minted}{coq}
      Fixpoint stlc_ccl {~$\Gamma$~ ~$\tau$~} (t : tm ~$\Gamma$~ ~$\tau$~) : ccl ~$\Gamma$~ (~$\synUnit$~ ~$\synFunc$~ ~$\tau$~) :=
        match t with
        (* Base *)
        | var v => ccl_const (ccl_var v)
        | app t1 t2 => ~$\langle$~ stlc_ccl t1, stlc_ccl t2 ~$\rangle$~ ;; ccl_ev
        | abs t' => ccl_env (stlc_ccl t')
        (* Products *)
        | tuple t1 t2 => ~$\langle$~ stlc_ccl t1, stlc_ccl t2~$\rangle$~
        | first t' => stlc_ccl t' ;; ccl_exl
        | second t => stlc_ccl t ;; ccl_exr
        (* Reals *)
        | plus t1 t2 => ~$\langle$~ stlc_ccl t1, stlc_ccl t2~$\rangle$~ ;; ccl_plus
        | rval r => ccl_const (ccl_rval r)
        | mplus t1 t2 => ~$\langle$~ stlc_ccl t1, stlc_ccl t2~$\rangle$~ ;; ccl_mplus
        | mrval r => ccl_const (ccl_mrval r)
        (* Unit *)
        | it => ccl_id
        end.
    \end{minted}
    \caption{Simply-typed lambda calculus to CCL translation}
    \label{lst:combinator_stlc_to_ccl}
  \end{listing}

  As with defining the denotations of variables, we also need to define a mechanism to ensure the correct variable is still referenced when translating from the typed lambda calculi to CCL.
  Consider that the typing context, previously a list, is isomorphic to a single nested product type.
  Empty lists correspond to the built-in unit type while concatenation becomes nested tupling.
  This isomorphism is what we use to translate the typing contexts to the input and output type format used by the combinator language.

  \begin{minted}{coq}
    Fixpoint translate_context (~$\Gamma$~ : Ctx) : ty :=
      match ~$\Gamma$~ with
      | nil => ~$\synUnit$~
      | ~$\tau$~ :: ~$\Gamma'$~ => ~$\tau$~ ~$\synStar$~ translate_context ~$\Gamma'$~
      end.
  \end{minted}

  Doing a lookup in such a nested product type then reduces to applying the correct projection combinator.

  \begin{minted}{coq}
    Fixpoint fetch ~$\Gamma$~ ~$\tau$~ (v : ~$\tau$~ ~$\in$~ ~$\Gamma$~) : comb (translate_context ~$\Gamma$~) ~$\tau$~ :=
      match v with
      | Top => exl
      | Pop v' => exr ;; fetch v'
      end.
  \end{minted}

  We also define additional functions to correctly model weakening in the combinator language.

  \begin{minted}{coq}
    Definition weaken ~$\tau$~ ~$\rho$~ ~$\sigma$~ (c : comb ~$\tau$~ ~$\rho$~): comb (~$\sigma$~ ~$\synStar$~ ~$\tau$~) ~$\rho$~ := exr ;; c.
    Fixpoint weaken_ctx ~$\Gamma$~ ~$\tau$~ (c : comb ~$\synUnit$~ ~$\tau$~): comb (translate_context ~$\Gamma$~) ~$\tau$~ :=
      match ~$\Gamma$~ with
      | nil => c
      | ~$\tau'$~ :: ~$\Gamma'$~ => weaken ~$\tau'$~ (weaken_ctx ~$\Gamma'$~ c)
      end.
  \end{minted}

  \begin{listing}
    \begin{minted}{coq}
      Fixpoint ccl_ccc ~$\Gamma$~ ~$\tau$~ (c : @ccl ~$\Gamma$~ ~$\tau$~) : comb (translate_context ~$\Gamma$~) ~$\tau$~ :=
        match c with
        (* Base *)
        | ccl_var v => fetch v

        (* Reals *)
        | ccl_plus => curry (exr ;; cplus)
        | ccl_rval r => weaken_ctx ~$\Gamma$~ (crval r)
        | ccl_mplus => curry (exr ;; cmplus)
        | ccl_mrval r => weaken_ctx ~$\Gamma$~ (cmrval r)

        (* Category laws *)
        | ccl_id => curry exr
        | ccl_seq t1 t2 =>
          ~$\langle$~ ccl_ccc t2, ccl_ccc t1~$\rangle$~ ;; curry (assoc1 ;; cross id ev ;; ev)

        (* Cartesian *)
        | ccl_exl => curry (exr ;; exl)
        | ccl_exr => curry (exr ;; exr)

        (* Monoidal *)
        | ccl_cross t1 t2 =>
          ~$\langle$~ ccl_ccc t1, ccl_ccc t2 ~$\rangle$~ ;;
            curry (~$\langle$~ ~$\langle$~ exl;;exl, exr~$\rangle$~, ~$\langle$~ exl;;exr, exr ~$\rangle$~ ~$\rangle$~;; cross ev ev)

        (* Closed *)
        | ccl_ev => curry (exr ;; ev)
        | ccl_env t' =>
          curry (curry (sym ;; assoc2 ;; cross (ccl_ccc t') id ;; ev))

        (* Const *)
        | ccl_const t' => ccl_ccc t';; curry exl
        end.
    \end{minted}
    \caption{CCL to CCC translation}
    \label{lst:combinator_ccl_to_ccc}
  \end{listing}

  The final translation function is the composition of both the translation functions in \cref{lst:combinator_stlc_to_ccl,lst:combinator_ccl_to_ccc}.
  Note that we have to remove the extra $\synUnit$ type we added in the simply-typed lambda calculus to CCL translation shown \cref{lst:combinator_stlc_to_ccl}.

  \begin{minted}{coq}
    Definition stlc_ccc ~$\Gamma$~ ~$\tau$~ : tm ~$\Gamma$~ ~$\tau$~ ~$\rightarrow$~ comb (translate_context ~$\Gamma$~) ~$\tau$~ :=
      fun t => ~$\langle$~ ccl_ccc (stlc_ccl t), neg ~$\rangle$~ ;; ev.
  \end{minted}
