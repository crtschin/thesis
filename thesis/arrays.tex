\subsection{Arrays}\label{sec:arrays}
  % TODO: Mention something about Shaikhha, et al.'s system of optimizations and how this is able to let the performance of forward-mode AD approach reverse-mode levels.
  % TODO: Mention that we implemented what are known as pull arrays
  Automatic differentiation is rarely done on mono-valued real numbers, due to the massive computational power available on GPUs in the form of array operations.
  So the next extension worth considering in our language are array types.
  To be more specific, we will be considering the pull array formulation used by Shaikhha, et al.\cite{Shaikha2019}.

  Pull-arrays are a well-known formulation which lends itself well to deforestation\cite{10.1145/165180.165214}.
  The most significant element of this concept is to encode arrays as functions, which in some cases can be composed with other similarly encoded arrays.
  The two operations we will be focussing on are \<build> and \<get>.
  Note that $Ix$ represents some arity type which should correspond to the arity of the arrays the operations are used on.
  \begin{gather*}
    \begin{aligned}
      \text{build} &: (Ix \rightarrow A) \rightarrow [A] \\
      \text{get} &: Ix \rightarrow [A] \rightarrow A
    \end{aligned} \\
    \text{get}\ i\ (\text{build}\ f) \equiv f(i)
  \end{gather*}

  This formulation is straightforward to encode in the intrinsic representation, as shown in \cref{lst:tm_array}.
  We are able to omit much of their cardinality and indexing type-level considerations due to our shallow embedding of pull-arrays into the existing types in \<Coq>.
  The well-typed nature of our representation allows us to avoid much of the hairy details associated with bounds checking for both indexing and array creation.
  To accomplish this, we use the \<Fin> inductive datatype which is indexed by an upper-bound and represents for some $n$, the range $[1..n]$.

  The final iteration of our simply-typed lambda calculus including the array types is an expansion of the $\Fsmooth$ language by Shaikha, et al. as ours also includes sum types.
  We can define the missing \<ifold> and \<let> constructs as syntactic sugar using the primitive recursion terms we added in \cref{sec:sum_prim}.
  Note that due to our nameless representation, \<let> terms lose much of their original usefulness in writing programs.
  \begin{gather*}
    \letin{e}{t} \equiv \app{(\abs{t})}{e} \\
    \ifold{f}{n}{d} \equiv
    \nrec{(\app{(\abs{(\app{f}{(\nsucc{(\var{\Top})})})})}{(\nval{0})})}{n}{d}
  \end{gather*}

  % TODO: decide if the let rules should be included
  % Nonetheless, we included it to help formulate some of the program optimizations used in the final system.

  \begin{figure}
    \begin{mathpar}
      \inferrule*[Right=\textsc{TBuild}]
        {\Gamma \vdash f : \Fin{n} \rightarrow \tm{\Gamma}{\tau} }
        {\Gamma \vdash \build{n}{f} : \Array{n}{\tau}} \\ \and
      \inferrule*[Right=\textsc{TGet}]
        {\Gamma \vdash t : \Array{n}{\tau} \\
          \Gamma \vdash i : \Fin{n}}
        {\Gamma \vdash \get{i}{t} : \tau}
    \end{mathpar}
    \caption{Type-inference rules for array construction and indexing}
    \label{fig:array_infer}
  \end{figure}

  \begin{listing}
    \begin{minted}{coq}
      Inductive tm ~($\Gamma$ : Ctx) : ty $\to$ Type~ :=
        ...
        (* Arrays *)
        | build : ~forall n \tau (f : $\Fin{n}$ $\rightarrow$ $\tm{\Gamma}{\tau}$), tm $\Gamma$ $(\Array{n}{\tau})$~
        | get : ~forall n \tau (i : $\Fin{n}$), tm $\Gamma$ $(\Array{n}{\tau})$ $\to$ tm $\Gamma$ $\tau$~
      .
    \end{minted}
    \caption{The terms related to array types included in our language}
    \label{lst:tm_array}
  \end{listing}

  Both the macro and denotation functions deviate slightly from how the previous terms were defined.
  When looking at the macro, while it previously sufficed to recursively call the macro on subterms, this is not possible as the subterm of interest is now a function.
  Instead, we substitute the function by a composition of itself combined with the forward-mode macro, essentially applying the macro to every possible result of the function.

  \begin{figure}
    \centering
    \begin{align*}
        &\cdots \\
        \D(\Array{n}{R}) &= \Array{n}{\D(R)} \\
        \D(\build{n}{f}) &= \build{n}{(\D \circ f)} \\
        \D(\get{i}{t}) &= \get{i}{\D(t)}
    \end{align*}
    \caption{Macro on array construction and indexing terms}
    \label{eqn:macro_array}
  \end{figure}

  Similarly for denotations, the denotation function, instantiated to the correct type, has to be passed along to an auxiliary function that builds up a vector of denotation terms.
  Appropriately, array types will denotate to vectors indexed by length.
  In \cref{fig:denotation_array} we formulate this as $\text{vector}(n, T)$, where $n$ indicates the length of the vector and $T$ is the contained type.
  There is some additional boilerplate necessary to circumvent the structurally recursive requirement imposed by the \<Coq> type checker.
  Note that in the definition of $vect$, the $Fin$ and $nat$ types are treated interchangeably, where $\Fin{n}$ is treated as the type level integer corresponding to a $n : \denN$.
  So every $n : \denN$ is transformed to $1 : \Fin{n}$, and any $i : \Fin{n}$ is transformed to $n : \denN$.

  \begin{figure}
    \begin{gather*}
      \begin{aligned}
        &\cdots \\
        \llbracket \Array{n}{\tau} \rrbracket &=
          \text{vector}(n, \llbracket \tau \rrbracket) \\
        \llbracket \build{n}{f} \rrbracket &=
          \lambda x.
          \vect(n, \llbracket \rrbracket \circ f, x) \\
        \llbracket \get{i}{t} \rrbracket &=
          \lambda x. \denGet{\llbracket t \rrbracket(x)}{i} \\
      \end{aligned} \\
      \begin{aligned}
      \vect(i, f, x) &=
        \begin{cases}
          \denNil
            &: i = 0 \\
          \denCons{f(i)(x)}{\vect(i', \lambda j. f(j + 1), x)}
            &: i = i' + 1
        \end{cases}
      \end{aligned}
    \end{gather*}
    \caption{Denotations of the array construction and indexing terms}
    \label{fig:denotation_array}
  \end{figure}

  The logical relation for array types needs to exhibit the same behavior for both construction and indexing in how it preserves the relation on subterms.
  This is accomplished by quantifying over the indices of the vector denotation.
  Next, the denotation of each subterm needs to both preserve the relation and be extensionally equal to the appropriate projection of the term.
  \begin{equation}
    S_\tau(f, g) =
      \left\{
        \begin{array}{ll}
          \dots \\
          \forall i. \exists f_1, g_1.
            & : \tau = \Array{n}{\sigma} \\
          \;\;S_\sigma(f_1, g_1) \wedge \\
          \;\;\;\;f_1 = \lambda x. f(x)!i \wedge \\
          \;\;\;\;f_1 = \lambda x. g(x)!i \\
        \end{array}
      \right.
  \label{eqn:lr_array}
  \end{equation}

  As was the case for sum types, only the proof of the fundamental lemma needs to be extended.
  The proof for the array terms proceeds as follows.
  For \<build>, we first do induction on $n$, the length of the array.
  The base case is trivial, as $\Fin{0}$ contains $0$ inhabitants.
  For the induction step, we first do case-analysis on the indices, $i$, where the $(+1)$ case follows from the induction hypothesis.
  For $i=1$ it suffices to give the proper inhabitants using the induction hypothesis derived from the function used for construction.
