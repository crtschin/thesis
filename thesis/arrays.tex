\subsection{Arrays}
  Rarely is automatic differentiation done on mono-valued real numbers, due to the massive computational power available in GPUs in the form of array operations.
  So the next extension worth considering in our language are the array types.
  To be more specific we will be considering the array types and terms presented by Shaikhha, et. al.\cite{Shaikha2019}.
  The well-typed nature of our presentation allows us to avoid much of the hairy details associated with bounds checking both indexing and array creation constructions.
  This is possible using the finite datatype, \<Fin>, which is indexed by the upper-bound and represents for some $n$, the range of $[1..n]$.

  \begin{figure}
    \begin{mathpar}
      \inferrule*[Right=\textsc{TBuild}]
        {\Gamma \vdash f : \Fin{n} \rightarrow tm \Gamma \tau }
        {\Gamma \vdash \build{n}{f} : \Array{n}{\tau}} \\ \and
      \inferrule*[Right=\textsc{TGet}]
        {\Gamma \vdash t : \Array{n}{\tau} \\
          \Gamma \vdash i : \Fin{n}}
        {\Gamma \vdash \get{i}{t} : \tau}
    \end{mathpar}
    \caption{Type-inference rules for array construction and indexing}
    \label{fig:array_infer}
  \end{figure}

  Both the macro and denotation functions deviate slightly from how the previous terms were defined.
  % While it sufficed to recurse down the datatype and apply the appropriate functions at each level previously, this is not possible as with the \<build> term as it contains a function.
  When looking at the macro, while it previously sufficed to recursively call the macro on subterms, this is not possible as the subterm of interest is now a function.
  This can be solved by substituting the function by a composition of itself combined with the forward-mode macro, essentially applying the macro to every possible result of the function.

  \begin{figure}
    \centering
    \begin{equation*}
      \begin{split}
        \D(\Array{n}{R}) &= \Array{n}{\D(R)} \\
      \end{split}
      \begin{split}
        \D(\build{n}{f}) &= \build{n}{(\D \circ f)} \\
        \D(\get{i}{t}) &= \get{i}{\D(t)}
      \end{split}
    \end{equation*}
    \caption{Macro on array construction and indexing terms}
    \label{eqn:macro_array}
  \end{figure}

  Similarly for denotations, the denotation function, instantiated to the correct type, has to be passed along to an auxiliary function that builds up a vector of denotation terms.
  Appropriately, array types will denotate to vectors indexed by length.
  There is some additional boilerplate necessary to circumvent the structurally recursive requirement imposed by the \<Coq> type checker.
  Note that in the $S$ case of $vectorize$, the $Fin$ and $nat$ types are treated interchangeably, where $\Fin{n}$ is interpreted as $n$.
  The recursive call to $vectorize$ also transforms the function by incrementing the bounded integer given as its argument.

  \begin{figure}
    \centering
    \begin{gather}
      \begin{align*}
      \llbracket \Array{n}{\tau} \rrbracket &=
        vector(n, \llbracket \tau \rrbracket) \\
      \llbracket \build{n}{f} \rrbracket &=
        \lambda x.
        vect(n, \llbracket \rrbracket \circ f, x) \\
      \llbracket \get{i}{t} \rrbracket &=
        \lambda x. \llbracket t \rrbracket(x) ! i \\
      \end{align*} \\
      \begin{align*}
      vect(i, f, x) &=
        \begin{cases}
          []
            &: i = 0 \\
          f(i, x) :: vect(i', \lambda j. f(j + 1), x)
            &: i = S(i')
        \end{cases}
      \end{align*}
    \end{gather}
    \caption{Denotations of the array construction and indexing terms}
    \label{eqn:denotation_array}
  \end{figure}

  The logical relation for array types needs to exhibit the same behavior with respect to both construction and indexing in how it preserves the relation on subterms.
  This is accomplished by first and foremost, quantifying over the indices possible for the vector denotation.
  Next, each subdenotation needs to both preserve the relation and be extensionally equal to the appropriate projection of the term.

  \begin{equation}
    S_\tau(f, g) =
      \left\{
        \begin{array}{ll}
          \forall i. \exists f_1, g_1.
            & : \tau = \Array{n}{\sigma} \\
          \;\;S_\sigma(f_1, g_1) \wedge \\
          \;\;\;\;f_1 = \lambda x. f(x)!i \wedge \\
          \;\;\;\;f_1 = \lambda x. g(x)!i \\
        \end{array}
      \right.
  \label{eqn:lr_array}
  \end{equation}

  The proof for the array terms proceeds as follows.
  For \<build>, we first do induction on $n$, the length of the array.
  The base case is trivial, as $\Fin{0}$ contains $0$ inhabitants.
  For the induction step, we first do case-analysis on the indices, $i$, where the $(+1)$ case follows from the induction hypothesis.
  For $i=1$ it suffices to give the proper inhabitants using the induction hypothesis derived from the function used for construction.
