\subsection{Formalizing Continuation-Based AD}\label{sec:continuation-base}
  The guiding principle in this algorithm is to build up the reverse pass as a continuation using a structure preserving macro on the program syntax.
  Like the forward-mode macro $\D$, the continuation-based macro $\Dc$ is structure preserving and works with tuples at ground type $\synR$.
  But unlike the forward-mode macro, which uses these tuples to represent dual-numbers, the continuation-based macro pairs the primal values with a continuation calculating an array containing all of the perturbation variables.
  Notably, because the continuation-based macro, shown in \cref{fig:continuation_macro}, calculates all of the perturbation values with respect to each of the input variables, it is additionally indexed by the number of input variables.

  \begin{figure}[H]
    \centering
    \begin{gather*}
      \begin{aligned}
        \Dc_n(\synR) &= \synR \synStar (\synR \synFunc \Array{n}{\synR}) \\
        \Dc_n(\tau \synStar \sigma) &= \D(\tau) \synStar \D(\sigma) \\
        \Dc_n(\tau \synPlus \sigma) &= \D(\tau) \synPlus \D(\sigma) \\
        \Dc_n(\tau \synFunc \sigma) &= \D(\tau) \synFunc \D(\sigma) \\
        \Dc_n(\Array{m}{\tau}) &= \Array{m}{\Dc_n(\tau)} \\
      \\
        \Dc_n(\rval{r}&) = \tuple{(\rval{r})} \\
          &{(\build{(\synConst{(\rval{0})})})} \\
        \Dc_n(\add{r_1}{r_2}&) = \tuple{(\add{r_1}{r_2})}{(\abs{(\vecadd{ \\
        &(\app{r_1'}{(\var{\Top})})}{(\app{r_2'}{(\var{\Top})})})})} \\
        \Dc_n(\mul{r_1}{r_2}&) = \tuple{(\mul{r_1}{r_2})}{(\abs{(\vecadd{ \\
        &(\app{r_1'}{(\mul{r_2}{(\var{\Top})})})}{ \\
        &(\app{r_2'}{(\mul{r_1}{(\var{\Top})})})})})} \\
      \end{aligned}
    \end{gather*}
    \caption{Continuation-based macro on the simply-typed lambda calculus extended with array types}
    \label{fig:continuation_macro}
  \end{figure}

  As we will phrase correctness in terms of the forward-mode macro, the forward-mode macro used in \cref{sec:formal_stlc} has to be slightly altered to calculate the partial derivatives with respect to all of the input variables.
  This change is slight, and merely swaps what was previously the tangent value for a vector of the tangent values corresponding to the input variables.
  The operations of reals are then swapped out for their vector element-wise counterparts.
  The modified forward-mode macro is shown in \cref{fig:forward_mode_cont}.

  \begin{figure}
    \centering
    \begin{subfigure}{1\textwidth}
      \begin{gather*}
        \begin{aligned}
          \D_n(\synR) &= \synR \synStar \Array{n}{\synR} \\
          \D_n(\tau \synStar \sigma) &= \D(\tau) \synStar \D(\sigma) \\
          \D_n(\tau \synPlus \sigma) &= \D(\tau) \synPlus \D(\sigma) \\
          \D_n(\tau \synFunc \sigma) &= \D(\tau) \synFunc \D(\sigma) \\
          \D_n(\Array{m}{\tau}) &= \Array{m}{\Dc_n(\tau)}
        \end{aligned} \\ \\
        \begin{aligned}
          \D_n(\rval{r}) &= \tuple{(\rval{r})}{(\build{n}{(\texttt{fun \_ => 0})})} \\
          \D_n(\add{r_1}{r_2}) &= \tuple{(\add{r_1}{r_2})}{(\vecadd{r_1'}{r_2'})} \\
          \D_n(\mul{r_1}{r_2}) &= \tuple{(\mul{r_1}{r_2})}
            {\\&(\vecadd{(\vecscale{r_2}{r_1'})}{(\vecscale{r_1}{r_2'})})}
        \end{aligned}
      \end{gather*}
      \caption{Forward-mode macro calculating all partial derivatives}
    \end{subfigure}
    \begin{subfigure}{1\textwidth}
      \begin{minted}{coq}
        Definition vector_map ~$\Gamma$~ ~$\tau$~ ~$\sigma$~ n ( f : tm ~$\Gamma$~ (~$\tau \synFunc \sigma$~) )
        ( a : tm ~$\Gamma$~ (Array n ~$\tau$~) ) : tm ~$\Gamma$~ (Array n ~$\sigma$~) :=
          build n (fun i => app f (get i a)).
        Definition vector_map2 ~$\Gamma$~ ~$\tau$~ ~$\sigma$~ ~$\rho$~ n
          ( a1 : tm ~$\Gamma$~ (Array n ~$\tau$~) ) ( a2 : tm ~$\Gamma$~ (Array n ~$\sigma$~) )
          ( f : tm ~$\Gamma$~ (~$\tau \synFunc \sigma \synFunc \rho$~) ) : tm ~$\Gamma$~ (Array n ~$\rho$~) :=
          build n (fun i => app (app f (get i a1)) (get i a2)).
        Definition vector_add ~$\Gamma$~ n
          ( a1 a2 : tm ~$\Gamma$~ (Array n ~$\synR$~) ) : tm ~$\Gamma$~ (Array n ~$\synR$~) :=
          vector_map2 a1 a2 (abs (abs (add (var Top) (var (Pop Top))))).
        Definition vector_scale ~$\Gamma$~ n ( s : tm ~$\Gamma$~ ~$\synR$~ )
          ( a : tm ~$\Gamma$~ (Array n ~$\synR$~) ) : tm ~$\Gamma$~ (Array n ~$\synR$~) :=
          vector_map (abs (mul s (var Top))) a.
      \end{minted}
      \caption{Helper functions on array types}
    \end{subfigure}
    \caption{Consolidated forward-mode macro on the simply-typed lambda calculus extended with array types}
    \label{fig:forward_mode_cont}
  \end{figure}

  As we work with the same denotational semantics as in our proof in \cref{sec:formal_stlc}, the usage of an argument supplying function $f : \denR \denFunc \denR^n$ makes a reappearance.
  Unlike last time, however, the arguments have to be massaged to fit both the forward and the reverse-mode macro.
  Note that for the forward-mode macro, input arguments we take the partial derivative of are supplied in a dual number format where the tangent value is equal to $1$.
  Likewise, the tangent value of any other input argument should be set to $0$.
  So each of the input arguments for the forward-mode macro are essentially coupled with a one-hot encoded vector.
  \begin{equation*}
    \D_n\left(
    \begin{bmatrix}
      x_1 & x_2 & x_3 \\
    \end{bmatrix}\right)
    \to
    \begin{bmatrix}
      (x_1,
        \begin{bmatrix}
          1 & 0 & 0 \\
        \end{bmatrix}) \\
      (x_2,
        \begin{bmatrix}
          0 & 1 & 0 \\
        \end{bmatrix}) \\
      (x_3,
        \begin{bmatrix}
          0 & 0 & 1 \\
        \end{bmatrix}) \\
    \end{bmatrix}^T
  \end{equation*}
  Each of the one-hot encoded vectors function as indicating which partial derivative to calculate.
  The continuation-based macro requires a similar format, but instead of a one-hot encoded vector containing the value $1$, the identity function is used.
  \begin{equation*}
    \Dc_n\left(
    \begin{bmatrix}
      x_1 & x_2 & x_3 \\
    \end{bmatrix}\right)
    \to
    \begin{bmatrix}
      (x_1,
        \lambda x.
        \begin{bmatrix}
          x & 0 & 0 \\
        \end{bmatrix}) \\
      (x_2,
        \lambda x.
        \begin{bmatrix}
          0 & x & 0 \\
        \end{bmatrix}) \\
      (x_3,
        \lambda x.
        \begin{bmatrix}
          0 & 0 & x \\
        \end{bmatrix}) \\
    \end{bmatrix}^T
  \end{equation*}

  Using these two input massaging functions, we can state correctness of our continuation-based macro with respect to our forward-mode macro.

  \begin{proposition}[Correctness of continuation-based AD]
    For any well-typed term $x_1 : \synR, \dots, x_n : \synR \vdash t : \synR$ and function arguments $x : \llbracket \synRepeat{\synR}{n} \rrbracket$ we will take the partial derivatives relative to, we have that $snd(\llbracket\D_n(t)\rrbracket (\D_n(x))) = snd(\llbracket\Dc_n(t)\rrbracket (\Dc_n(x)))(1)$.
  \end{proposition}

  This statement cannot be proven directly, however, we need a more general correctness statement.
  To be be specific, we need to establish equivalence between the macros regardless of the number we apply to the continuation-based macro, which corresponds to the coefficient of the forward-mode macro.
  This equivalence needs to be baked into the logical relation in our proof.
  The logical relation we will use in this proof for the most part stays the same as the one we used in the proof in \cref{sec:forward}.
  The only case to differ significantly is the one for our ground type, the type of real numbers, $\synR$.
  Also note that, like the macro functions, the logical relation is also indexed by the number of input variables.

  \begin{equation}
    S_{n, \tau}(f, g) =
      \left\{
        \begin{array}{ll}
          fst \circ f = fst \circ g \wedge
            & : \tau = \synR \\
          \;\;\;\forall x. \lambda r. x * snd(f(r)) = \lambda r. (snd(g(r)))(x)
        \end{array}
      \right.
  \label{eqn:lr_continuous}
  \end{equation}

  As a recurring theme, many of the definitions and lemmas we used in the proof in \cref{sec:forward} reappear in this proof, differing only superficially in that they are additionally indexed by the number of partial derivatives.
  So we still make use of an instantiation relation like the one given in \cref{eqn:inst_base} to establish that arbitrary typing context instantiations preserve the logical relation.

  The proof of the fundamental lemma proceeds in much the same manner as the one given for \cref{thm:fundamental_lemma}.
  As expected, the cases for the \synR-typed are exceptional.
  Each of these cases require a generalized version of the case to be proven.
  The problem here is that the number of partial derivatives is tightly coupled to the number of terms the operation is over.
  So induction on the number of terms involved in the operation necessary leads to issues with our relations, as these are indexed by the number of partial derivatives.

  Proving this along with the corresponding fundamental property of the logical relation results in the following theorem.
  \begin{theorem}[Perturbation correctness]
    For any well-typed term $x_1 : \synR, \dots, x_n : \synR \vdash t : \synR$, function arguments $x : \llbracket \synRepeat{\synR}{n} \rrbracket$, and real number $r$, we have $r * snd(\D_n(t)(\D_n(x))) = snd(\Dc_n(t)(\Dc_n(x)))(r)$
  \end{theorem}
  \begin{proof}
    This follows from the fundamental property of the logical relation in (\cref{eqn:lr_continuous}).
  \end{proof}

  This is easily specialized to the original correctness statement we set out to prove.

  \begin{corollary}[Correctness of continuation-based AD]
    For any well-typed term $x_1 : \synR, \dots, \linebreak x_n : \synR \vdash t : \synR$ and function arguments $x : \llbracket \synRepeat{\synR}{n} \rrbracket$ we will take the partial derivatives relative to, we have that $snd(\llbracket\D_n(t)\rrbracket (\D_n(x))) = snd(\llbracket\Dc_n(t)\rrbracket (\Dc_n(x)))(1)$.
  \end{corollary}
