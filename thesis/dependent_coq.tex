\subsubsection{Dependently-typed programming in Coq}
% TODO: Work through MV feedback
In \<Coq>, one can normally write function definitions using either case-analysis as is done in other functional languages, or using \<Coq>'s tactics language.
Using the standard case-analysis functionality can cause the code to be complicated and verbose, even more so when proof terms are present in the function signature.
This has been caused by the previously poor support in \<Coq> for dependent pattern matching.
Using the return keyword, one is able to vary the result type of a match expression. But due to requirement \<Coq> used to have that case expressions be syntactically total, this could be very difficult to work with.
One other possibility would be to write the function as a relation between its input and output.
This also has its limitations as you then lose computability as \<Coq> treats these definitions opaquely. In this case the standard \<simpl> tactic which invokes \<Coq>'s reduction mechanism is not able to reduce instances of the term.
This often requires the user to write many more proofs to be able to work with the definitions.

As an example, we will work through defining a length indexed list and a corresponding head function limited to lists of length at least one in \cref{lst:dt_ilist}.
Using the \<Coq> keyword return, it is possible to let the return type of a match expressions depend on the result of one of the type arguments.
This makes it possible to define an auxiliary function which, while total on the length of the list, has an incorrect return type. It namely returns the type unit if the input list had the length zero.
We can then use this auxiliary function in the actual head function by specifying that the list has length at least one.
It should be noted that more recent versions of \<Coq> do not require that case expressions be syntactically total, so specifying that the input list has a length of at least zero is enough to eliminate the requirement for the zero-case.

\begin{listing}
  \begin{minted}{coq}
  Inductive ilist : ~Type \rightarrow nat \rightarrow Type~ :=
    | nil : ~\forall A, ilist A 0~
    | cons : ~\forall A n, A \rightarrow ilist A n \rightarrow ilist A (S n)~

  Definition hd' {A} n (ls : ilist A n) :=
    match ls in (ilist A n) return
      (match n with
      | O => unit
      | S _ => A end) with
    | nil => tt
    | cons h _ => h
  end.

  Definition hd {A} n (ls : ilist A (S n)) : A := hd' n ls.
  \end{minted}
  \caption{Definition of a length indexed list and hd using the return keyword, adapted from Certified Programming with Dependent Types\cite{ChlipalaCPDT}.}
  \label{lst:dt_ilist}
\end{listing}

Mathieu Sozeau introduces an extension to \<Coq> via a new keyword \<Program> which allows the use of case-analysis in more complex definitions\cite{10.5555/1789277.1789293}\cite{Sozeau2007}.
To be more specific, it allows definitions to be specified separately from their accompanying proofs, possibly filling them in automatically if possible.
While this does improve on the previous situation, using the definitions in proofs can often be unwieldy due to the amount of boilerplate introduced.
This makes debugging error messages even harder than it already is in a proof assistant. This approach was used by Benton in his formulation of strongly typed terms.

Sozeau further improves on this introducing a method for user-friendlier dependently-typed pattern matching in \<Coq> in the form of the \<Equations> library\cite{Sozeau2010}\cite{Sozeau2019}.
This introduces \<Agda>-like dependent pattern matching with with-clauses.
It does this by using a notion called coverings, where a covering is a set of equations such that the pattern matchings of the type signature are exhaustive.
There are two main ways to integrate this in a dependently typed environment, externally where it is integrated as high-level constructs in the pattern matching core as \<Agda> does it, or internally by using the existing type theory and finding witnesses of the covering to prove the definition correct, which is the approach used by Sozeau.
Due to the intrinsic typeful representation this paper uses, much of this was invaluable when defining the substitution operators as the regular type checker in \<Coq> often had difficulty unifying dependently typed terms in certain cases.

\begin{listing}
  \begin{minted}{coq}
  Equations hd {A n} (ls : ilist A n) (pf : n <> 0) : A :=
  hd nil pf with pf eq_refl := {};
  hd (cons h n) _ := h.
  \end{minted}
  \caption{Definition of hd using \<Equations>}
  \label{lst:dt_ilist_hd_equations}
\end{listing}
