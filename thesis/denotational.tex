\subsection{Denotational semantics}
% A formal semantics of programming language: An introduction

The notion of denotational semantics tries to find underlying mathematical models able to underpin the concepts known in programming languages. The most well-known example is the solution given by Dana Scott and Christopher Strachey\cite{Scott1977} for lambda calculi, also called domain theory.
To be able to formalize non-termination and partiality, they thought to use concepts such as partial orderings and least fixed points\cite{aaby2020}.
In this model, programs are interpreted as partial functions, and recursive computations as taking the fixpoint of such functions.
Non-termination, on the other hand, is formalized as a value \<bottom> that is lower in the ordering relation than any other element.

In our specific case, we try to find a satisfactory model we can use to show that our implementation of forward mode automatic differentiation is correct when applied to a simply-typed lambda calculus.
In the original pen and paper proof of automatic differentiation this thesis is based on, the mathematical models used were diffeological spaces, which are a generalization of smooth manifolds.
For the purpose of this thesis, however, we were able to avoid using diffeological spaces as recursion, iteration and concepts dealing with non-termination and partiality are left out of the scope of this thesis.
\<Coq> has very limited support for domain theoretical models.
There are possible libraries which have resulted from experiments trying to encode domain theoretical models\cite{Benton2009}\cite{Dockins2014}, but these are incompatible with recent versions of \<Coq>.
As a part of its type system, \<Coq> contains a set-theoretical model available under the sort \<Set>, which is satisfactory as the denotational semantics for our language.

Because we use the real numbers as the ground type in our language, we also needed an encoding of the real numbers in Coq. The library for real numbers in \<Coq> has improved in recent times from one based on a completely axiomatic definition to one involving Cauchy sequences\fancyfootnote{https://coq.inria.fr/library/Coq.Reals.ConstructiveCauchyReals.html}. For the purposes of this thesis, however, we needed differentiability as the denotational result of applying the macro operation. Instead of encoding this by hand, we opted for the more comprehensive library \<Coquelicot>\cite{Boldo2015CoquelicotAU}, which contains many general definitions for differentiating functions.
