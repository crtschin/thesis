\subsection{Simply Typed Lambda Calculus}\label{sec:formal_stlc}
  As mentioned in the background \cref{sec:language_repr}, we will make use of De-Bruijn indices in an intrinsic representation to formulate our language.
  We include both addition and multiplication as example operations on the real numbers, but the proofs are easily extensible to other primitive operations.
  Our base language consists of the classic simply-typed lambda calculus with product types and real numbers.

  Both the language constructs and the typing rules for this language are common for a simply-typed lambda calculus.
  These are shown as typing judgments in \cref{fig:base_infer}.
  As expected, we include variables, applications, and abstractions in the language using, respectively, the \<var>, \<app>, and \<abs> terms.
  We work with projection products, whose elimination rules are encoded in the  \<first> and \<second> terms. The \<tuple> term is used to represent the introduction rule.
  For real numbers, \<rval> is used to introduce real numbered constants and \<add> and \<mul> will be used to respectively encode addition and multiplication.
  The included operations are chosen for their simplicity, but the proof is able to accommodate any operation so long as it is total and differentiable.

  \begin{figure}
    \begin{mathpar}
      \inferrule*[Right=\textsc{TVar}]
        {elem\ n\ \Gamma = \tau}
        {\Gamma \vdash \var{n} : \tau} \and
      \inferrule*[Right=\textsc{TAbs}]
        {(\sigma, \Gamma) \vdash t : \tau}
        {\Gamma \vdash \abs{t} : \sigma \rightarrow \tau} \\ \and
      \inferrule*[Right=\textsc{TApp}]
        {\Gamma \vdash t1 : \sigma \rightarrow \tau \\
          \Gamma \vdash t2 : \sigma}
        {\Gamma \vdash \app{t1}{t2} : \tau} \\ \and
      \inferrule*[Right=\textsc{TTuple}]
        {\Gamma \vdash t1 : \tau \\
          \Gamma \vdash t2 : \sigma}
        {\Gamma \vdash \tuple{t1}{t2} : \tau \times \sigma} \\ \and
      \inferrule*[Right=\textsc{TFst}]
        {\Gamma \vdash t : \tau \times \sigma}
        {\Gamma \vdash \first{t} : \tau} \and
      \inferrule*[Right=\textsc{TSnd}]
        {\Gamma \vdash t : \tau \times \sigma}
        {\Gamma \vdash \second{t} : \sigma} \\ \and
      \inferrule*[Right=\textsc{TRVal}]
        {r \in \denR}
        {\Gamma \vdash \rval{r} : \synR} \\ \and
      \inferrule*[Right=\textsc{TAdd}]
        {\Gamma \vdash r1 : \synR \\
          \Gamma \vdash r2 : \synR \\ }
        {\Gamma \vdash \add{r1}{r2} : \synR} \and
      \inferrule*[Right=\textsc{TMull}]
        {\Gamma \vdash r1 : \synR \\
        \Gamma \vdash r2 : \synR \\ }
      {\Gamma \vdash \mul{r1}{r2} : \synR} \and
    \end{mathpar}
    \caption{Type-inference rules for the base simply-typed lambda calculus}
    \label{fig:base_infer}
  \end{figure}

  % How we translated this into the well-typed intrinsic representation
  These can be translated into \<Coq> definitions in a reasonably straightforward manner, with each case keeping track of both how the typing context and types change.
  In the \<var> case, we need some way to determine what type the variable is referencing.
  Like many others previously\cite{Benton2011}\cite{Coquand1994}, instead of using indices into the list accompanied by a proof that the index does not exceed the length of the list, we make use of an inductively defined type evidence to type our variables as shown in \cref{lst:strong_stlc}.
  The cases for \<app> and \<abs> are as expected, where variables in the body of abstractions can reference their respective arguments.

  Note that in the original proof by Huot, Staton, and \Vakar{} \cite{huot2020correctness}, they made use of n-ary products accompanied by pattern matching expressions.
  We opted to implement binary projection products, as these are conceptually simpler while still retaining much of the same functionality expected of product types.
  The code for implemented products and included operations on real numbers is given in \cref{lst:stlc_prod_r}.

  \begin{listing}
    \begin{minted}{coq}
      Inductive tm ~(\Gamma : Ctx) : ty \rightarrow Type~ :=
        ...
        (* Binary projection products *)
        | tuple : ~forall {\tau \sigma},
          tm \Gamma \tau \rightarrow
          tm \Gamma \sigma \rightarrow
          tm \Gamma (\tau \synStar \sigma)~
        | first : ~forall {\tau \sigma}, tm \Gamma (\tau \synStar \sigma) \rightarrow tm \Gamma \tau~
        | second : ~forall {\tau \sigma}, tm \Gamma (\tau \synStar \sigma) \rightarrow tm \Gamma \sigma~
        (* Operations on reals *)
        | rval : ~forall r, tm \Gamma $\synR$~
        | add : ~tm \Gamma $\synR$ \rightarrow tm \Gamma $\synR$ \rightarrow tm \Gamma $\synR$~
        | mul : ~tm \Gamma $\synR$ \rightarrow tm \Gamma $\synR$ \rightarrow tm \Gamma $\synR$~
    \end{minted}
    \caption{Terms in our language related to product and real types.}
    \label{lst:stlc_prod_r}
  \end{listing}

  % \begin{listing}
  %   \begin{minted}{coq}
  % Definition Ctx : Type := list ty.

  % Inductive tm ~(\Gamma : Ctx) : ty \rightarrow Type~ :=
  %   (* Base STLC *)
  %   | var : ~forall \tau,
  %     \tau ∈ \Gamma \rightarrow tm \Gamma \tau~
  %   | app : ~forall \tau \sigma,
  %     tm \Gamma (\sigma \Rightarrow \tau) \rightarrow
  %     tm \Gamma \sigma \rightarrow
  %     tm \Gamma \tau~
  %   | abs : ~forall \tau \sigma,
  %     tm (\sigma::\Gamma) \tau \rightarrow tm \Gamma (\sigma \Rightarrow \tau)~

  %   (* Operations on real numbers *)
  %   | const : ~R \rightarrow tm \Gamma Real~
  %   | add : ~tm \Gamma Real \rightarrow tm \Gamma Real \rightarrow tm \Gamma Real~
  %   | mul : ~tm \Gamma Real \rightarrow tm \Gamma Real \rightarrow tm \Gamma Real~

  %   (* Binary projection products *)
  %   | tuple : ~forall {\tau \sigma},
  %     tm \Gamma \tau \rightarrow
  %     tm \Gamma \sigma \rightarrow
  %     tm \Gamma (\tau \times \sigma)~
  %   | first : ~forall {\tau \sigma}, tm \Gamma (\tau \times \sigma) \rightarrow tm \Gamma \tau~
  %   | second : ~forall {\tau \sigma}, tm \Gamma (\tau \times \sigma) \rightarrow tm \Gamma \sigma~
  %   \end{minted}
  %   \caption{\<Coq> definition of the base lambda calculus}
  %   \label{lst:stlc_base}
  % \end{listing}

  We use the same inductively defined macro on types and terms used by many previous authors to implement the forward-mode automatic differentiation macro\cite{Karczmarczuk98functionaldifferentiation}\cite{Mark2008NestingFA}\cite{Shaikha2019}.
  The forward-mode macro, $\D$, keeps track of both primal and tangent traces using tuples as respectively its first and second elements.
  In most cases, the macro simply preserves the structure of the language.
  The cases for real numbers such as addition and multiplication are the exception.
  Here, the element encoding the tangent trace needs to contain the proper syntactic translation of the derivative of the operation.

  Due to the intrinsic nature of our language representation, the macro also needs to be applied to both the types and typing context to ensure that the terms remain well-typed.
  In other words, for any well-typed term $\Gamma \vdash t : \tau$, applying the forward-mode macro results in a well-typed term in the macro-expanded context, $\D(\Gamma) \vdash \D(t) : \D(\tau)$.

  \begin{figure}
    \centering
    \begin{equation*}
      \begin{split}
        \D(\synR) &= \synR \synStar \synR \\
        \D(\tau \synStar \sigma) &= \D(\tau) \synStar \D(\sigma) \\
        \D(\tau \synFunc \sigma) &= \D(\tau) \synFunc \D(\sigma)
      \end{split}
      \;\;\;\;\;\;
      \begin{split}
        \D(\rval{n}) &= \tuple{(\rval{n})}{(\rval{0})} \\
        \D(\add{n}{m}) &= \tuple{(\add{n}{m})}{(\add{n'}{m'})} \\
        \D(\mul{n}{m}) &= \tuple{(\mul{n}{m})} \\
          &{(\add{(\mul{n'}{m})}{(\mul{m'}{n})})})
      \end{split}
    \end{equation*}
    \caption{Macro on base simply-typed lambda calculus}
    \label{eqn:macro_base}
  \end{figure}

  Applying the macro to a term gives the syntactic counterparts of both their primal and tangent denotations as a tuple.
  These terms can be accessed with projections to implement the various derivative implementations of the operations on real terms included in the language.
  Note that applying the macro to the case for variables does nothing as the macro is also applied to the typing context, so variables implicitly already reference macro-applied terms.
  The macro on types as well as real numbered values is given in \cref{eqn:macro_base}.

  As we restrict our language to total constructions and excluding concepts such as general recursion and iteration, it suffices to give our language a set-theoretic denotational semantics.
  In this case the types $\synR, \synFunc, \synStar$ directly correspond to their \<Coq> equivalent, respectively $\denR, \denFunc, \denStar$.
  Well-typed terms of type $\tau$, given typing context $\Gamma$, will denotate to functions $\llbracket \Gamma \rrbracket \to \llbracket \tau \rrbracket$.

  \begin{figure}
    \centering
    \begin{gather*}
      \begin{aligned}
        \llbracket \synR \rrbracket &= \denR \\
        \llbracket \tau \synStar \sigma \rrbracket &=
          \llbracket \tau \rrbracket \denStar \llbracket \sigma \rrbracket \\
        \llbracket \tau \synFunc \sigma \rrbracket &= \llbracket \tau \rrbracket \denFunc \llbracket \sigma \rrbracket \\
        \\
        \llbracket \<Top> \rrbracket &= \<hd> \\
        \llbracket \<Pop>\ v \rrbracket &= \llbracket v \rrbracket \circ \<tl> \\
      \end{aligned}
      \;\;\;\;\;\;
      \begin{aligned}
        \llbracket \var{v} \rrbracket &=
          \lambda x. lookup\ \llbracket v \rrbracket\ x \\
        \llbracket \app{t_1}{t_2} \rrbracket &=
          \lambda x. (\llbracket t_1 \rrbracket(x)) (\llbracket t_2 \rrbracket(x)) \\
        \llbracket \abs{t} \rrbracket &=
          \lambda x\ y. \llbracket t \rrbracket(y :: x) \\
        \llbracket \add{t_1}{t_2} \rrbracket &=
          \lambda x. \llbracket t_1 \rrbracket(x) + \llbracket t_2 \rrbracket(x) \\
        \llbracket \mul{t_1}{t_2} \rrbracket &=
          \lambda x. \llbracket t_1 \rrbracket(x) * \llbracket t_2 \rrbracket(x) \\
        \llbracket \tuple{t_1}{t_2} \rrbracket &=
          \lambda x. (\llbracket t_1 \rrbracket(x), \llbracket t_2 \rrbracket(x)) \\
        \llbracket \first{t} \rrbracket &=
          \lambda x. fst(\llbracket t \rrbracket(x)) \\
        \llbracket \second{t} \rrbracket &=
          \lambda x. snd(\llbracket t \rrbracket(x)) \\
      \end{aligned} \\ \\
      \begin{aligned}
        fst &=
          \lambda x. let\ (x, y) \coloneqq \llbracket t \rrbracket(x)\ in\ x \\
        snd &=
          \lambda x. let\ (x, y) \coloneqq \llbracket t \rrbracket(x)\ in\ y \\
      \end{aligned}
    \end{gather*}
    \caption{Denotations of the base simply-typed lambda calculus}
    \label{eqn:denotation_base}
  \end{figure}

  Denotating the terms in our language now corresponds to finding the appropriate inhabitants in the denotated types.
  As typing contexts, $\Gamma$, are represented by lists of types.
  The appropriate way to denotate these would be to map the denotation function over the list.
  The resulting heterogeneous list contains the denotations of each type in the list in the correct order.
  The specific implementation of heterogeneous lists used in the proof corresponds to the one given by Adam Chlipala\cite{ChlipalaCPDT}.
  In this implementation, heterogeneous lists consist of an underlying list of some type $A$ and an accompanying function $A \to Set$, which in our use case are, respectively, the typing context and the denotation function.

  When giving the constructs in our language their proper denotations, most of the cases are straightforward.
  Notable is the case for variables, where we made use of the inductively defined type evidence to type our terms.
  Remember that to type variables in our term language, we have to also give the exact position of the type we are referencing in the typing context.
  Similarly as denotations, we are able to transform this positional information to generate a specialized lookup function, which given a valid typing context, gives a term denotation with the correct type.
  Essentially, we do a lookup into the heterogeneous list of denotations corresponding to the typing context.
  The denotations for the various types and terms in our base language is shown in \cref{eqn:denotation_base}

  \begin{minted}{coq}
    Equations denote_v ~$\Gamma$~ ~$\tau$~ (v: ~$\tau \in \Gamma$~) : ~$\llbracket \Gamma \rrbracket \rightarrow \llbracket \tau \rrbracket$~ :=
    denote_v (Top ~$\Gamma$~ ~$\tau$~) := hd;
    denote_v (Pop ~$\Gamma$~ ~$\tau$~ ~$\sigma$~ v) := denote_v v ~$\circ$~ tl.
  \end{minted}

  \begin{example}[Square]
    $abs\ (mul\ (var\ Top)\ (var\ Top))$ denotates to the square function $\lambda x. x * x$.
    \begin{proof}
      This follows from the definition of our denotation functions.
      \begin{align*}
        \llbracket abs\ &(mul\ (var\ Top)\ (var\ Top)) \rrbracket\ [] \\
          &\equiv \lambda x.
            \llbracket mul\ (var\ Top)\ (var\ Top) \rrbracket\ [x] \\
          &\equiv \lambda x.
            \llbracket var\ Top \rrbracket\ [x] *
              \llbracket var\ Top \rrbracket\ [x] \\
          &\equiv \lambda x. x * x \qedhere
      \end{align*}
    \end{proof}
  \end{example}

  Using the denotation rules in \cref{eqn:denotation_base}, syntactically well-typed terms in our language of the form $x_1 : \synR, \dots, x_n : \synR \vdash t : \synR$ can be interpreted as their corresponding smooth functions $f : \denR^n \to \denR$.
  Intuitively, the free variables in the syntactic term $t$ correspond to the parameters of the denotation function $f$.

  Although Barthe et al.\cite{barthe2020versatility} gave a syntactic proof of correctness of the macro, our formal proof follows the more denotational style of proof given by Huot, Staton and \Vakar{}\cite{huot2020correctness}.
  Likewise, our proof of correctness will follow a similar logical relations argument.
  While both approaches have their merits, the proof using the denotational semantics requires less technical bookkeeping of open and closed terms.

  Informally, the correctness statement of the forward-mode macro will consists of the assertion that the denotation of any macro-applied term of type $x_1 : \synR, \dots, x_n : \synR \vdash t : \synR$ will result in a pair of both the denotation of the original term and the derivative of that denotation.
  Note that while both the free variables and result type of the term $t$ are restricted to type $\synR$, $t$ itself can consist of subterms of higher-order types.

  Proving this correctness statement directly by induction on the structure of our small language, however, does not work.
  Instead, we use logical relations argument.
  The logical relation will ensure that both the smoothness and the derivative calculating properties are preserved over higher-order types.
  As expected, when a logical relations argument is required, the indicative issue lies in the cases for function application.
  To be specific, the induction hypothesis would indicate that both the argument and the function terms would satisfy our notion of correctness.
  The induction hypothesis, however, would be too weak to establish this property for the result of the application.
  To get the proof to go through, we need to strengthen the induction hypothesis at function types to include precisely the assertion that if the argument of our function application is valid with respect to the logical relation, then the result, too, is valid.

  We define the logical relation as a type-indexed relation between denotations of both terms and their macro-applied variants, so for any type $\tau$, $S_\tau$ is the relation between functions $R \rightarrow \llbracket \tau \rrbracket$ and $R \rightarrow \llbracket \D(\tau) \rrbracket$.
  For our ground types, $\synR$, the logical relation should establish that the macro-applied term should be denotationally equivalent to the original denotation and its corresponding derivative.
  Some care has to be taken when defining the relation for product types.
  Notably, the denotations of the subterms, $R \rightarrow \llbracket \tau \rrbracket$ and $R \rightarrow \llbracket \sigma \rrbracket$, should be existentially quantified as these are dependent on the original denotation $R \rightarrow \llbracket \tau \times \sigma \rrbracket$.
  One also has to supply proofs that both subterms also satisfy the logical relation.

  \begin{definition}(Logical relation)
    Denotation functions $f$ and their corresponding derivatives $g$ are inductively defined on the structure of our types such that they follow the relation
    \begin{equation*}
      S_\tau(f, g) =
        \left\{
          \begin{array}{ll}
            smooth\ f \wedge
              g = \lambda x. (f(x), \frac{\partial f}{\partial x}(x))
              & : \tau = \synR \\
            \exists f_1, f_2, g_1, g_2,
              & : \tau = \sigma \synStar \rho \\
              \;\;\;\;S_\sigma(f_1, f_2), S_\sigma(g_1, g_2). \\
              \;\;\;\;f = \lambda x. (f_1(x), g_1(x)) \wedge \\
              \;\;\;\;g = \lambda x. (f_2(x), g_2(x)) \\
            \forall f_1, f_2.
              & : \tau = \sigma \synFunc \rho \\
              \;\;\;\;S_\sigma(f_1, f_2) \Rightarrow \\
              \;\;\;\;S_\rho(\lambda x. f(x)(f_1(x)),\lambda x. f(x)(f_2(x)))
          \end{array}
        \right.
    \label{eqn:lr_base}
    \end{equation*}
  \end{definition}

  The next step involves proving that the well-typed terms, which correspond to functions $\denR^n \denFunc \denR$, are semantically correct with respect to the logical relation.
  In other words, the relation needs to be proven valid for any term $x_1 : \synR, \dots, x_n : \synR \vdash t : \synR$ and argument function $f : \denR \rightarrow \denR^n$ such that $S_\tau(\llbracket t \rrbracket \circ f, \llbracket \D(t) \rrbracket \circ \D_n \circ f)$.
  This proposition is what we will use to derive our final correctness statement once we generalize it to arbitrary types, contexts and implicitly, substitutions.
  If this were a syntactic proof, one would need to show that the relation is preserved when applying substitutions consisting of arbitrary terms, possibly containing higher-order constructs.
  In this style of proof, however, the same concept needs to be formulated in a denotational manner.

  The key in formulating these denotationally lies in the argument function $f : \denR \denFunc \denR^n$.
  Previously, the function was used to indicate the open variables or function arguments.
  Generalized to $\Gamma = x_1 : \tau_1, \dots, x_n : \tau_n$, this function can be interpreted as supplying for each open variable $x_1, \dots, x_n$ a corresponding denotated term with type $\llbracket \tau_1 \rrbracket, \dots, \llbracket \tau_n \rrbracket$.
  So the argument function now becomes the pair of functions $s : \denR \denFunc \llbracket \Gamma \rrbracket$ and $s_D : \denR \denFunc \llbracket \D(\Gamma) \rrbracket$, which intuitively speaking, form the denotational counterparts of syntactic substitutions.
  Notably, for the functions $s$ and $s_D$ to be valid with respect to the logical relation, they are required to be built from the denotations of terms such that these denotations also follow the logical relation.
  We phrase this requirement as a definition.

  \begin{definition}(Instantiation)
    Argument functions $s : R \rightarrow \llbracket \Gamma \rrbracket$ and $s_D : R \rightarrow \llbracket \D(\Gamma) \rrbracket$ are inductively defined such that they follow
    \begin{equation}
      \inst{\Gamma}(f, g) =
        \left\{
          \begin{array}{ll}
            f = (\lambda x. \denNil) \wedge g = (\lambda x. \denNil)
              & : \Gamma = \denNil \\
            \forall f_1, f_2, g_1, g_2.
              & : \Gamma = (\tau :: \Gamma') \\
              \;\;\inst{\Gamma'}(f_1, g_1) \wedge S_\tau(f_2, g_2) \\
              \;\;\;\; \to f = (\lambda x. f_2(x) :: f_1(x)) \wedge \\
              \;\;\;\;\;\; g = (\lambda x. g_2(x) :: g_1(x))
          \end{array}
        \right.
    \label{eqn:inst_base}
    \end{equation}
  \end{definition}

  Using this notion of instantiations, we can now formulate our fundamental lemma.
  Informally, this states that given the correct argument functions, any well-typed term is semantically correct with respect to the logical relation.
  \begin{lemma}[Fundamental lemma]\label{thm:fundamental_lemma}
    For any well-typed term $\Gamma \vdash t : \tau$, and instantiation functions $s : \denR \rightarrow \llbracket \Gamma \rrbracket$ and $s_D : \denR \rightarrow \llbracket \D(\Gamma) \rrbracket$ such that they follow $\inst{\Gamma}(s, s_D)$, we have that $S_\tau(\llbracket t\rrbracket \circ s, \llbracket \D(t)\rrbracket \circ s_D)$.
  \end{lemma}

  \begin{proof}
    This is proven by induction on the typing derivation of the well-typed term $t$. The majority of cases follow from the induction hypothesis.
    The case for \<var> follows from $\inst{}$ which ensures that any term referenced is semantically well-typed with respect to the relation.
    Proving the cases used to encode the operators on reals such as \<add> and \<mul> involve proving both smoothness and giving the witness of the derivative.
  \end{proof}

  We can now step-by-step specialize the fundamental lemma to our final correctness theorem.
  The first step involves limiting the arguments of the functions to real numbers.
  The previous need to use the instantiation relation to establish that the typing context satisfies the logical relation disappears if we specialize these contexts to real numbers.
  In its place, we have to manually establish the connection between real numbers and their dual number counterparts.
  We establish this connection using a denotational translation from real to dual numbers in the definition of $\Darg$, as shown in \cref{eqn:argument_df}.
  This function couples each primal value with their corresponding tangent value.

  \begin{definition}[Initialization of dual number representations]
    For any $n$-length argument function $f : \denR \rightarrow \llbracket \synR^n \rrbracket$, couple each argument with its corresponding tangent such that the resulting transformed function becomes $\Darg_n(f, x) : \denR \rightarrow \llbracket \D(\synR)^n \rrbracket$. This can be defined as the following higher-order function:
    \begin{equation*}
      \Darg_n(f, x) =
        \left\{
          \begin{array}{ll}
            f(x) & : n = 0 \\
            ((hd \circ f)(x), \frac{\partial{(hd \circ f)}}{\partial{x}}(x)) :: \Darg_{n'}(tl \circ f, x) & : n = n' + 1 \\
          \end{array}
        \right.
    \label{eqn:argument_df}
    \end{equation*}
  \end{definition}

  Note that because we make use of the derivatives of our arguments, we also need to require that every argument be differentiable.
  This requirement is non-negotiable and will be present in the theorem.
  We formulate this requirement as \cref{thm:differentiable}.

  \begin{definition}[Differentiability of arguments]\label{thm:differentiable}
    Any argument function $f : \denR \rightarrow \llbracket \synR^n \rrbracket$ with arity $n$, is inductively constructed such that it follows the following proposition.
    \begin{equation}
      \differentiable{n}(f) =
        \left\{
          \begin{array}{ll}
            \differentiable{0}(\lambda x. [])
              & : n = 0 \\
            \forall g, h.
              \differentiable{n'}(g) \rightarrow
              & : n = n' + 1 \\
              \;\;\;\;\;\;smooth\ h \rightarrow \\
              \;\;\;\;\;\;f = (\lambda x. h(x) :: g(x))
          \end{array}
        \right.
    \label{eqn:differentiable_base}
    \end{equation}
  \end{definition}

  \begin{corollary}[Fundamental property]\label{thm:fundamental_property}
    For any term $x_1 : \synR, \dots, x_n : \synR \vdash t : \synR$, $\llbracket\D(t)\rrbracket$ gives the dual number representation of $\llbracket t \rrbracket$, such that for any argument function $f : \denR \to \denR^n$ that follows $\differentiable{n}(f)$, we have that $S_\tau(\llbracket t\rrbracket \circ f, \llbracket \D(t)\rrbracket \circ \D_n \circ f)$.
  \end{corollary}

  \begin{proof}
    This follows from the fundamental lemma. We lastly need to prove $\inst{\synR^n}$.
    This is proven by induction on $n$.
    If $n = 0$, the goal is trivial due to the argument function $f$ being extensionally equal to $\synConst{[]}$, which directly corresponds to $\inst{[]}$.
    The induction step is proven by both the induction hypothesis and the assumption that the denotations of the arguments supplied are smooth.
  \end{proof}

  The final step includes specializing the output variable of our functions to real numbers.
  Simplifying the logical relation for $\synR$ gives the correctness statement we set out to prove.

  \begin{corollary}[Macro correctness]\label{thm:macro_correctness}
    For any well-typed term $x_1 : \synR, \dots, x_n : \synR \vdash t : \synR$, $\llbracket\D(t)\rrbracket$ gives the dual number representation of $\llbracket t \rrbracket$, such that for any argument function $f : \denR \rightarrow \denR^n$ that follows $\differentiable{n}(f)$, we have that $\llbracket \D(t) \rrbracket \circ \D_n \circ f = \lambda x. (\llbracket t \rrbracket \circ f, \sfrac{\partial{(\llbracket t \rrbracket \circ f)}}{\partial{x}})$.
  \end{corollary}

  \begin{proof}
    This is proven by showing that the goal follows from the logical relation which itself is implied by \cref{thm:fundamental_property}.

    \begin{align*}
      \llbracket \D(t) &\rrbracket \circ \D_n \circ f = \lambda x. (\llbracket t \rrbracket \circ f, \sfrac{\partial{(\llbracket t \rrbracket \circ f)}}{\partial{x}}) \\
      & \Vdash \text{(By definition of $S_R$ with $f := \llbracket t \rrbracket \circ f$ and $g := \llbracket \D(t) \rrbracket \circ \D_n \circ f$)} \\
      & S_R(\llbracket t \rrbracket \circ f, \llbracket \D(t) \rrbracket \circ \D_n \circ f) \\
      & \Vdash \text{(Fundamental property (\cref{thm:fundamental_property}))}
    \end{align*}
  \end{proof}
