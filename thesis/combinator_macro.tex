\subsection{Defining the Macro and Target Language}\label{sec:combinator-macro}
  The reverse-mode macro we will discuss in this chapter, like the forward-mode macro splits any type $\tau$ into tuples $(\tau_1, \tau_2)$, where $\tau_1$ and $\tau_2$ represent, respectively, primal and tangent values.
  As expected, for $\tau_1$ the macro essentially preserves the behavior expected of any of the combinators.
  The intuition what happens with reverse-mode AD gives a hint to what $\tau_2$.
  Essentially, starting from an input vector and the derivative of the output variable, reverse-mode AD calculates the derivatives of each of the input variables.
  On the type level, this can be formulated as transforming combinators $\comb{\tau}{\sigma}$ into a tuple of combinators $(\comb{(fst(\Drev(\tau)))}{(fst(\Drev(\sigma)))}, \comb{(fst(\Drev(\tau)) \synStar snd(\Drev(\sigma)))}{(snd(\Drev(\tau)))})$.

  Defining this transformation poses an issue however, as we will illustrate with an example.
  Consider the combinators \<exl>, \<exr> and \<dupl> from \cref{sec:combinator-core}.
  \begin{align*}
    \Drev(\synExl) &= (\synExl, \synSeq{\synDupl}{\synCross{\synExr}{(\synSeq{\synNeg}{\synO})}}) \\
    \Drev(\synExr) &= (\synExr, \synSeq{\synDupl}{\synCross{(\synSeq{\synNeg}{\synO})}{\synExr}}) \\
    \Drev(\synDupl) &= (\synDupl, \synSeq{\synExr}{\synP})
  \end{align*}
  Note the usage of the monoidal combinators $(\synO, \synP)$.
  As mentioned in the previous section, these combinators encode the dual structures to the contraction and weakening rules present in programming language inference rules.
  This encodes how we will treat the fan-out problem, where $\synP : \comb{\tau \synStar \tau}{\tau}$ combines two terms of type $\tau$ in a generic way.
  Likewise, $\synO : \comb{\synUnit}{\tau}$ formulates how to zero out variables of arbitrary type $\tau$.
  We can define these combinators by induction on their types, where $\vec{0}$ is the regular vector consisting of just zeroes.
  \begin{align*}
    \synO_\tau =
      \left\{
        \begin{array}{ll}
          \synCrval{0} & : \tau = \synR \\
          \synCmrval{\vec{0}} & : \tau = \synRn \\
          \synNeg & : \tau = \synUnit \\
          \langle \synO_\sigma, \synO_\rho \rangle & : \tau = \sigma \synStar \rho \\
          \synCurry{(\synSeq{\synExl}{\synO_\rho})} & : \tau = \sigma \synFunc \rho \\
        \end{array}
      \right.
  \end{align*}
  Defining $\synO_\tau$ is straightforward as we only need to ensure that it preserves the structure of our types
  For ground types $\synR$ and $\synRn$, we generate their respective interpretations of zero.
  For function types, we can ignore the input of argument type and recursively call $\synO$ on the result type.
  \begin{align*}
    \synP_\tau =
      \left\{
        \begin{array}{ll}
          \synCplus & : \tau = \synR \\
          \synCmplus & : \tau = \synRn \\
          \synNeg & : \tau = \synUnit \\
          \langle
            \synSeq{\langle \synSeq{\synExl}{\synExl}, \synSeq{\synExr}{\synExl} \rangle}{\synP_\sigma}, & : \tau = \sigma \synStar \rho \\
            \;\;\;\synSeq{\langle \synSeq{\synExl}{\synExr}, \synSeq{\synExr}{\synExr} \rangle}{\synP_\rho} \rangle \\
          \synCurry{ \\
            (\synSeq{\langle
              \synSeq{\langle \synSeq{\synExl}{\synExl}, \synExr \rangle}{\synEv}, & : \tau = \sigma \synFunc \rho \\
              \;\;\;\synSeq{\langle \synSeq{\synExl}{\synExr}, \synExr \rangle}{\synEv} \rangle}{\synP_\rho})} \\
        \end{array}
      \right.
  \end{align*}
  With $\synP_\tau$, we can make recursive usage of the operator to combine subterms.
  For tuples, this involves creating a new tuple where the left and right components consist of, respectively, the combinations of all left and right projections.
  With function types, both left and right input functions need to be evaluated separately before they can be combined.

  We can reuse the same reasoning we used to derive how the combinators should be transformed for reverse-mode, to define the macro on the function types.
  This reasoning leads to the incomplete definition of
  \begin{multline}
    \Drev(\tau \synFunc \sigma) = (fst(\Drev(\tau)) \synFunc (fst(\Drev(\sigma)) \synStar (snd(\Drev(\sigma)) \synFunc snd(\Drev(\tau)))),
      \\ fst(\Drev(\tau)) \synStar snd (\Drev(\sigma)))
    \label{eqn:combinator_d_incorrect}
  \end{multline}
  % TODO: Explain faulty use of products as tangent values.
  In a sense, the tangent value at function types keeps track of the possibly incomplete adjoints currently calculated, as these can be combined with other adjoints down the line.
  Remember that in reverse-mode AD, the problem of fan-out involves combining each usage of a variable in the reverse pass.
  We solved the issue of combining variable usages using the $\synP$ combinator, but still have to find something to keep track of which terms to combine.

  In \cref{eqn:combinator_d_incorrect}, we attempt to keep track of the adjoints using product types, which are easily combined using $\synP$.
  This, however, does not ensure that the linear properties of both $(\synO, \synP)$ are preserved.
  Indeed, for any program consisting of combinators such that it has type $(\comb{\tau}{\synR})$, if $\llbracket \synO_\tau \rrbracket$ is given as input in the denotational sense, the resulting program should correspond to $0$.
  In other words, we need to establish that the denotations of our combinators are linear with respect to both $(\synO, \synP)$.

  % TODO: Introduce need for linear types
  \Vakar{} poses the usage of an auxiliary target language with a limited variant of linear types as the result of evaluating the reverse-mode macro\cite{vkr2020reverse}.
  The linear types used include both linear function types, $\synLin$, and tensor products, $\synTens{(-)}{(-)}$.
  This target language would be almost identical to the source combinator language described in \cref{sec:combinator-core}.
  The added types and terms included are shown in \cref{fig:combinator_linear}.

  % | exl_l : forall A B,
  %   target (~$\synTens{A}{B}$~) A
  % | exr_l : forall A B,
  %   target (~$\synTens{A}{B}$~) B

  \begin{figure}
    \centering
    \begin{minted}{coq}
      Inductive ty : Type :=
        ...
        | ~$\synTens{\_}{\_}$~ : ty ~$\rightarrow$~ ty ~$\rightarrow$~ ty
        | ~$\synLin$~  : ty ~$\rightarrow$~ ty ~$\rightarrow$~ ty
      .
      Inductive target : ty ~$\rightarrow$~ ty ~$\rightarrow$~ Type :=
        ...
        (* Linear *)
        | ev_l : forall A B, target ((A ~$\synLin$~ B) ~$\synStar$~ A) B
        | curry_l : forall A B C,
          target (A ~$\synStar$~ B) C ~$\rightarrow$~ target A (B ~$\synLin$~ C)
        (* Tensor *)
        | mempty : forall ~$A$~ ~$B$~,
        target ~$\synUnit$~ (~$\synTens{A}{B}$~)
        | msingleton : forall ~$A$~ ~$B$~,
          target (~$A$~ ~$\synStar$~ ~$B$~) (~$\synTens{A}{B}$~)
        | mplus : forall ~$A$~ ~$B$~,
          target ((~$\synTens{A}{B}$~) ~$\synStar$~ (~$\synTens{A}{B}$~)) (~$\synTens{A}{B}$~)
        | mfold : forall ~$A$~ ~$B$~ ~$C$~,
          target (~$\synTens{A}{B}$~ ~$\synStar$~ (~$A$~ ~$\synStar$~ ~$B$~ ~$\synLin$~ ~$C$~)) ~$C$~
      .
    \end{minted}
    \caption{}
    \label{fig:combinator_linear}
  \end{figure}

  Because we extended the target language with both linear functions and tensor products, we also have to extend both $(\synO, \synP)$ with these types.
  \begin{align*}
    \synO_\tau &=
      \left\{
        \begin{array}{ll}
          \dots \\
          \synCurryL{(\synSeq{\synExl}{\synO})} & : \tau = \sigma \synFunc \rho \\
          \synMempty & : \tau = \synTens{\sigma}{\rho} \\
        \end{array}
      \right. \\
    \synP_\tau &=
      \left\{
        \begin{array}{ll}
          \dots \\
          \synCurryL{ \\
            (\synSeq{\langle
              \synSeq{\langle \synSeq{\synExl}{\synExl}, \synExr \rangle}{\synEvL}, & : \tau = \sigma \synFunc \rho \\
              \;\;\;\synSeq{\langle \synSeq{\synExl}{\synExr}, \synExr \rangle}{\synEvL} \rangle}{\synP_\rho})} \\
          \synMplus & : \tau = \synTens{\sigma}{\rho} \\
        \end{array}
      \right.
  \end{align*}

  Using these additional constructs, we can finally define the macro as shown in \cref{fig:macro_combinator}.
  Note the usages of the $(\synO, \synP)$ combinators.
  While $\synO$ is used to explicitly zero out the unused terms in both \<exl> and \<exr>, $\synP$ is used to recombine terms in \<dupl>.

  \begin{figure}
    \centering
    \begin{gather*}
      \begin{multlined}
        \Drev(\tau \synFunc \sigma) =
          (fst(\Drev(\tau)) \synFunc
            (fst(\Drev(\sigma)) \synStar
              (snd(\Drev(\sigma)) \synLin snd(\Drev(\tau)))),
          \\ \synTens{fst(\Drev(\tau))}{snd(\Drev(\sigma))})
      \end{multlined} \\
      \begin{aligned}
        \Drev(\synId) &= (\synId, \synExr) \\
        \Drev(\synSeq{t_1}{t_2}) &= (\synSeq{d_1}{d_2},
        (\synSeq{\synSeq{\synSeq{\synCross{\synDupl}{\synId}}{assoc1}}{
          \\&\;\;\; (\synCross{\synId}{(\synCross{d_1}{\synId})})}}{
          (\synCross{\synId}{(\synSeq{d_2'}{d_1'})})})) \\
        \Drev(\synCross{t_1}{t_2}) &= (\synCross{d_1}{d_2},
          \synSeq{assoc1}{
            \synSeq{\synCross{\synId}{(
              \synSeq{assoc2}{
                \synSeq{\\&\;\;\;(\synCross{sym}{\synId})}{assoc1}})}}{
                  \synSeq{assoc2}{\synCross{d_1'}{d_2'}}}}) \\
        \Drev(\synNeg) &= (\synNeg, \synSeq{\synExr}{\synO}) \\
        \Drev(\synExl) &= (\synExl, \synSeq{\synDupl}{\synCross{\synExr}{(\synNeg ;; \synO)}}) \\
        \Drev(\synExr) &= (\synExr, \synSeq{\synDupl}{\synCross{(\synNeg ;; \synO)}{\synExr}}) \\
        \Drev(\synDupl) &= (\synDupl, \synSeq{\synExr}{\synP}) \\
        \Drev(\synEv) &= (\synSeq{\synEv}{\synExl}, \langle \synSeq{\langle \synSeq{\synExl}{\synExr}, \synExr \rangle}{\synMsingleton},
          \\&\;\;\; \synSeq{\synCross{(\synSeq{\synEv}{
            \synSeq{\synExr}{\synCurry{\synEvL}}})}{\synId}}{\synEv} \rangle) \\
        \Drev(\synCurry{t}) &= (\synCurry (\langle d, \synSeq{\synCurry{d'}}{\synExr} \rangle),
        \\&\;\;\; \synSeq{\synCross{\synId}{(\synSeq{\langle \synId, \synSeq{\synNeg}{\synCurry{\synExr}}\rangle}{\synMfold})}}
          {\synSeq{assoc2}{\synSeq{d'}{\synExl}}})
      \end{aligned}
    \end{gather*}
    \caption{Macro on the combinator language}
    \label{fig:macro_combinator}
  \end{figure}

  % These lead to the monoidal operations $\Gamma \vdash \synO : \tau$ and $\Gamma \vdash t_1 \synP t_2 : \tau$, where $\Gamma \vdash t_1, t_2 : \tau$.
  % Ensuring the higher-order types respect $(\synO, \synP)$ requires the usage of minimal linear types.
  % Notice that these two operations naturally correspond to, respectively, the zeroing and fan-out problems associated with reverse-mode AD.
