In the original proof of correctness by Huot, Staton and \Vakar{}\cite{huot2020correctness}, they encoded the smoothness requirements in the denotational semantics.
This meshed well with the existing mathematical literature in the form of diffeological spaces.
In contrast, we were able to encode this requirement directly within the logical relation used in the proof, while keeping the denotational semantics we needed as simple as possible.
In retrospect, this reduced a significant amount of the proof load needed as we did not have to redefine many of the properties of functions on smooth functions.
Unfortunately, however, the set-theoretic denotational semantics limits the proof possibilities to total programming langauges.

In added value of creating the formalized correctness proof of the forward-mode automatic differentiation macro is immediately visible in the fact that we were also able to simplify the fundamental lemma in the original pen-and-paper proof.
Excluding as much of the syntactic constructions such as substitutions were critical in ensuring the proof went as smmoothly as possible.
If formulated incorrectly, the additional complexity involved with typing contexts and variable management muddies the corresponding cases of \<app> and \<abs> in the proof of the fundamental lemma.
An unfortunate amount of time was spent in this hybrid approach of both syntactic and denotational structures, before focussing on the full denotational approach.

Shaikhha, et al.\cite{Shaikha2019} showed that the performance of forward-mode AD can approach that of reverse-mode AD given the right optimizations.
It is useful to realize that the various macros discussed in this thesis all ultimately calculate the same values, but do so using different expressions.
These expressions differ merely in the number of primitive operations, and as such are equal modulo the distributive law.
It may then be reasonable to suggest that an appropriately intelligent compiler would be able to, in a fully generic sense, optimize a forward-mode based gradient algorithm to the performance expected of a reverse-mode algorithm.

\Vakar{}\cite{vkr2020reverse} gave a novel extension to the categorical approach to automatic differentiation by Elliott\cite{Elliott-2018-ad-icfp}.
Unfortunately, however, we were unable to finish a formalized proof of correctness in time.
One of the principle issues was the difficulty in applying the simple set-theoretic denotational semantics we have used so vehemently throughout our other proofs.
It is presumed that departing from this to a one based on setoids may be more fruitful, but more research is required.
