\subsection{Adding Sums and Primitive Recursion}\label{sec:sum_prim}
  Now that correctness has been verified for the base simply-typed lambda calculus, the next goal will be to add in sum types.
  In the interest of testing the flexibility of both the representation and the proof technique, we also add natural number types and primitive recursion.
  The inference rules for the new language constructs added for sum and number types are given in \cref{fig:sum_prim_infer}, while their intrinsic representations are shown in \cref{lst:stlc_sums}.

  \begin{figure}
    \begin{mathpar}
      \inferrule*[Right=\textsc{TCase}]
        {\Gamma \vdash e : \tau \synPlus \sigma \\
          \Gamma \vdash t_1 : \tau \synFunc \rho \\
          \Gamma \vdash t_2 : \sigma \synFunc \rho }
        {\Gamma \vdash \case{e}{t_1}{t_2} : \rho} \\ \and
      \inferrule*[Right=\textsc{TInl}]
        {\Gamma \vdash t : \tau}
        {\Gamma \vdash \inl{t} : \tau \synPlus \sigma} \and
      \inferrule*[Right=\textsc{TInr}]
        {\Gamma \vdash t : \sigma}
        {\Gamma \vdash \inr{t} : \tau \synPlus \sigma} \\ \and
      \inferrule*[Right=\textsc{TNVal}]
        {n \in \synN}
        {\Gamma \vdash \nval{n} : \synN} \and
      \inferrule*[Right=\textsc{TNSucc}]
        {\Gamma \vdash t : \synN}
        {\Gamma \vdash \nsucc{t} : \synN} \\ \and
      \inferrule*[Right=\textsc{TPrim}]
        {\Gamma \vdash f : \tau \synFunc \tau \\
          \Gamma \vdash n : \synN \\
          \Gamma \vdash t : \tau }
        {\Gamma \vdash \nrec{f}{n}{t} : \tau}
    \end{mathpar}
    \caption{Type-inference rules for language constructs for sum types and primitive recursion}
    \label{fig:sum_prim_infer}
  \end{figure}

  \begin{listing}
    \begin{minted}{coq}
      Inductive tm ~($\Gamma$ : Ctx) : ty $\to$ Type~ :=
        ...
        (* Sums *)
        | case : ~forall $\tau$ $\sigma$ $\rho$,
          tm $\Gamma$ ($\tau$ $\synPlus$ $\sigma$) $\to$
          tm $\Gamma$ ($\tau \synFunc \rho$) $\to$
          tm $\Gamma$ ($\sigma \synFunc \rho$) $\to$
          tm $\Gamma$ $\rho$~
        | inl : ~forall $\tau$ $\sigma$,
          tm $\Gamma$ $\tau$ $\to$ tm $\Gamma$ ($\tau$ $\synPlus$ $\sigma$)~
        | inr : ~forall $\tau$ $\sigma$,
          tm $\Gamma$ $\sigma$ $\to$ tm $\Gamma$ ($\tau$ $\synPlus$ $\sigma$)~
        (* Primitive recursion *)
        | nval : ~forall n, tm \Gamma \synN~
        | nsucc : ~tm $\Gamma$ \synN $\to$ tm $\Gamma$ \synN~
        | nrec : ~forall $\tau$,
          tm $\Gamma$ ($\tau \synFunc \tau$) $\to$ tm $\Gamma$ \synN $\to$ tm $\Gamma$ $\tau$ $\to$ tm $\Gamma$ $\tau$~
    \end{minted}
    \caption{Terms in our language related to sum types.}
    \label{lst:stlc_sums}
  \end{listing}

  Binary sum types are included in the language using \<inl> and \<inr> as introducing terms.
  The \<case> term encodes case-analysis given a function term for each possibility.
  Primitive recursion is implemented using simple integers, where a endomorphic function is recursively applied a bounded number of times to a start value.
  For convenience, an additional successor function is added in the form of the \<nsucc> term.

  In terms of denotations, \<case> expressions will follow the same lines as \<app> as they both involve applying a function to an argument.
  Note that we first destruct the sum term to be able to determine which function branch to apply.
  Both \<inl> and \<inr> will map to their \<Coq> counterparts.
  For \<nrec>, the number of applications should be dependent on the input integer.
  The specific denotations chosen is given in \cref{eqn:denotation_sums_prim}.

  \begin{figure}
    \centering
    \begin{equation*}
      \begin{split}
        \dots \\
        \llbracket \tau \synPlus \sigma \rrbracket &= \llbracket \tau \rrbracket \denPlus \llbracket \sigma \rrbracket \\
        \llbracket \synN \rrbracket &= \denN \\
        \\
        \llbracket \case{e}{t_1}{t_2} \rrbracket &= \lambda x.
          \left\{
            \begin{array}{ll}
              (\llbracket t_1 \rrbracket(x))(t)
                & : \llbracket e \rrbracket(x) = inl(t) \\
              (\llbracket t_2 \rrbracket(x))(t)
                & : \llbracket e \rrbracket(x) = inr(t) \\
            \end{array}
          \right. \\
        \llbracket \inl{t} \rrbracket &= \lambda x. inl(\llbracket t \rrbracket(x)) \\
        \llbracket \inr{t} \rrbracket &= \lambda x. inr(\llbracket t \rrbracket(x)) \\
        \llbracket \nval{n} \rrbracket &= n \\
        \llbracket \nsucc{t} \rrbracket &= \lambda x. \llbracket t \rrbracket(x) + 1 \\
        \llbracket \nrec{f}{i}{t} \rrbracket &= \lambda x. fold(\llbracket f \rrbracket(x), \llbracket i \rrbracket(x), \llbracket t \rrbracket(x)) \\
        fold(f, i, t) &=
          \left\{
            \begin{array}{ll}
              t &: i = 0 \\
              f(fold(f, i', t))
                &: i = i' + 1 \\
            \end{array}
          \right. \\
      \end{split}
    \end{equation*}
    \caption{Denotations of the sum and integer terms}
    \label{eqn:denotation_sums_prim}
  \end{figure}

  Both sums and integer terms are structure-preserving with respect to the forward-mode macro.
  Note that we only take the derivative of values of type \synR, so when integers are encountered, these are largely ignored.
  More specifically, we do not keep track of derivatives at natural number types as the tangent space is $0$-dimensional.
  For a similar reason, the logical relation at integer types only needs to establish that the denotations of integer terms tracked by the logical relation, which will be of type $\denR \to \denN$, are constant.
  For sum terms, the functions tracked are either the left or right tag of the sum, which we can neatly define using a logical disjunction.

  \begin{figure}
    \centering
    \begin{equation*}
      \begin{split}
        \dots \\
        \D(\tau \synPlus \sigma) &= \D(\tau) \synPlus \D(\sigma) \\
        \D(\synN) &= \synN \\ \\
        \D(\inl{t}) &= \inl{\D(t)} \\
        \D(\inr{t}) &= \inr{\D(t)} \\
        \D(\case{e}{t_1}{t_2}) &= \case{\D(e)}{\D(t_1)}{\D(t_2)} \\
        \D(\nval{n}) &= \nval{n} \\
        \D(\nsucc{n}{m}) &= \nsucc{\D(n)}{\D(m)} \\
        \D(\nrec{f}{i}{t}) &= \nrec{\D(f)}{\D(i)}{\D(t)}
      \end{split}
    \end{equation*}
    \caption{Macro on the sum and integer terms}
    \label{eqn:macro_sums_prim}
  \end{figure}

  \begin{equation}
    S_\tau(f, g) =
      \left\{
        \begin{array}{ll}
          \dots \\
          f = g \wedge
            \exists n. f = const(n)
            & : \tau = \synN \\
          (\exists f_1, f_2, \\
            \;\;\;\;S_\sigma(f_1, f_2) \wedge f = inl \circ f_1 \wedge g = inl \circ g_1) \\
            \;\;\;\;\;\;\;\;\vee & : \tau = \sigma \synPlus \rho \\
          (\exists f_1, f_2,\\
            \;\;\;\;S_\rho(f_1, f_2) \wedge f = inr \circ f_1 \wedge g = inr \circ g_1) \\
        \end{array}
      \right.
  \label{eqn:lr_sums_prim}
  \end{equation}

  The only lemma or theorem that requires extension to deal with these new terms is the fundamental lemma, as the validity of every other statement is independent of the types or terms added to our language.
  With terms of integer type, the proof for the fundamental lemma is trivial using the definition of our denotation functions.
  The \<nrec> case for primitive recursion is only slightly more difficult as  we have to do case-analysis on the denotation of the iteration term.
  The $0$ and $n+1$ case are proven using the induction hypotheses derived from, respectively, the initial and function terms.
  As expected with the \<case> term for sums, the denotation of the term under scrutiny needs to be destructed to properly apply the two disjunct induction hypotheses to their corresponding cases.
