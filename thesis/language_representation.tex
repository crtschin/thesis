\subsubsection{Language representation}
\label{sec:language_repr}
% TODO: Work through MV feedback

\begin{figure}
  \begin{mathpar}
    \inferrule*[Right=\textsc{TVar}]
      {elem\ n\ \Gamma = \tau}
      {\Gamma \vdash \var{n} : \tau} \and
    \inferrule*[Right=\textsc{TAbs}]
      {(\sigma, \Gamma) \vdash t : \tau}
      {\Gamma \vdash t : \sigma \rightarrow \tau} \\ \and
    \inferrule*[Right=\textsc{TApp}]
      {\Gamma \vdash t_1 : \sigma \rightarrow \tau \\
        \Gamma \vdash t_2 : \sigma}
      {\Gamma \vdash t_1\ t_2 : \tau}
  \end{mathpar}
  \label{fig:stlc_infer}
  \caption{Type-inferrence rules for a simply-typed lambda calculus using De-Bruijn indices}
\end{figure}

When defining a simply-typed lambda calculus, there are two main possibilities\cite{plfa2019}.
The arguably simpler variant, known as an extrinsic representation, is traditionally the one introduced to new students learning \<Coq>.
In the extrinsic representation, the terms themselves are untyped and typing judgments are defined separately as relations between the types and terms. A basic example of working with this is given by Pierce, et al.\cite{Pierce:SF2}.
This, however, required many additional lemmas and machinery to be proved to be able to work with both substitutions and contexts as these are defined separate from the terms.
As an example, the preservation property which states that reduction does not change the type of a term, needs to be proven explicitly.
The other approach, also called an intrinsic representation, makes use of just a single well-typed definition.
Ill-typed terms are made impossible by the type checker.
This representation, while beneficial in the proof load, however complicates much of the normal machinery involved in programming language theory.
One example is how one would define operations such as substitutions or weakening.

But even when choosing an intrinsic representation, the problem of variable binding persists.
Much meta-theoretical research has been done on possible approaches to this problem each with their own advantages and disadvantages.
The POPLmark challenge gives a comprehensive overview of each of the possibilities in various proof assistants\cite{Aydemir2005}.
One example of such an approach is the nominal representation where every variable is named.
Code \cref{lst:nominal_stlc} gives an example of how one would define a simply-typed lambda calculus in such a representation.
While this does follow the standard format used in regular mathematics, problems such as alpha-conversion and capture-avoidance appears.

\begin{listing}[h]
  \begin{minted}{coq}
  Inductive ty : Type :=
    | ~unit~ : ty
    | ~\Rightarrow~ : ty ~\rightarrow~ ty ~\rightarrow~ ty.

  Inductive tm : Type :=
    | var : string ~\rightarrow~ tm
    | abs : string ~\rightarrow~ ty ~\rightarrow~ tm ~\rightarrow~ tm
    | app : tm ~\rightarrow~ tm ~\rightarrow~ tm.
  \end{minted}
  \caption{Simply typed \lambda-calculus using an nominal extrinsic representation.}
  \label{lst:nominal_stlc}
\end{listing}

The approach used in the rest of this thesis is an extension of the De-Bruijn representation which numbers variables relative to the binding lambda term.
In this representation the variables are referred to as well-typed De-Bruijn indices.
A significant benefit of this representation is that the problems of capture avoidance and alpha equivalence are avoided.
As an alternative, instead of using numbers to represent the distance, indices within the typing context can be used to ensure that a variable is always well-typed and well-scoped.
While the idea of using type indexed terms has been both described and used by many authors\cite{Altenkirch99}\cite{McBride04}\cite{Adams06}, the specific formulation used in this thesis using separate substitution and renaming operations was fleshed out in \<Coq> by Nick Benton, et al.\cite{Benton2011}, and was also used as one of the examples in the second POPLmark challenge which deals with logical relations\cite{poplmark_reloaded}.
While this does avoid the problems present in the nominal representation, it unfortunately does have some problems of its own.
Variable substitutions have to be defined using two separate renaming and substitution operations.
Renaming is formulated as extending the typing context of variables, while substitution actually swaps the variables for terms.
Due to using indices from the context as variables, some lifting boilerplate is required to manipulate contexts.

\begin{listing}[h]
  \begin{minted}{coq}
  Inductive ~\tau \in \Gamma~ : Type :=
    | Top : ~\forall \Gamma \tau, \tau \in (\tau::\Gamma)~
    | Pop : ~\forall \Gamma \tau \sigma, \tau \in \Gamma \rightarrow \tau \in (\sigma::\Gamma)~.

  Inductive tm ~\Gamma \tau~ : Type :=
    | var : ~\forall \Gamma \tau, \tau \in \Gamma \rightarrow tm \Gamma \tau~
    | abs : ~\forall \Gamma \tau \sigma, tm (\sigma::\Gamma) \tau \rightarrow tm \Gamma (\sigma \Rightarrow \tau)~
    | app : ~\forall \Gamma \tau \sigma, tm \Gamma (\sigma \Rightarrow \tau) \rightarrow tm \Gamma \sigma \rightarrow tm \Gamma \tau~.
  \end{minted}
  \caption{Basis of a simply-typed \lambda-calculus using the strongly typed intrinsic formulation.}
  \label{lst:strong_stlc}
\end{listing}

% TODO: Work out how substitutions work
