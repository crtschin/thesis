\documentclass[11pt, letterpaper]{article}
\usepackage[margin=1in]{geometry}
\usepackage{amsthm}
\usepackage{amsmath}
\usepackage[
backend=biber,
bibstyle=ieee,
citestyle=ieee,
sorting=ynt,
hyperref=true,
backref=true
]{biblatex}
\addbibresource{refs.bib}
\usepackage[hidelinks]{hyperref}

\title{Formalization of the Denotation Semantics of
    \\ Automatic Differentiation in Coq
    \\ Project Proposal
}
\author{Curtis Chin Jen Sem \\ (5601118)}
\date{April 2020}

\begin{document}

\maketitle

\section{Introduction}

AI and machine learning research has sparked a lot of new interest in recent times due to its many applications and ability to solve very difficult problems.


% Neural networks
% - optimizations
% - correctness
% - extending
% - expressiveness

% Ideally?:
% Contributing an extendable proof of a simply typed lambda calculus
% On which both the correctness of optimizations and language
% extensions could be proven

\section{Background}

\subsection{Automatic Differentation}

\subsection{Coq}

\subsection{Denotation Semantics}

\subsubsection{Domain Theory}

\section{Preliminary Results}

\subsection{Preliminary Proof}

\subsection{Proof Extension}

\section{Timetable and Planning}

\subsection{Deadlines}
\subsection{Fill in missing holes}
\subsection{Generalize prototype for variants and recursive types}

\printbibliography

\end{document}