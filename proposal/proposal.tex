\documentclass[a4, 12pt, final]{article}
\usepackage[english]{babel}
\usepackage[utf8]{inputenc}
\usepackage{graphicx}
\usepackage{fancyhdr}
\usepackage{amsthm}
\usepackage[cache]{minted}
\setlength{\marginparwidth}{3cm}
\usepackage{todonotes}
% \usepackage{libertine}
% \usepackage{libertinust1math}
\usepackage{amsmath}
\usepackage[justification=centering]{caption}
\usepackage{csquotes}
\usepackage{showkeys}
\usepackage[
  backend=biber,
  bibstyle=ieee,
  citestyle=ieee,
  sorting=ynt,
  hyperref=true,
  backref=true
]{biblatex}
\usepackage[hidelinks]{hyperref}
\usepackage{fontspec}
\setmainfont{TeX Gyre Pagella}
\setsansfont{TeX Gyre Heros}[Scale=MatchLowercase]
\setmonofont{Inconsolata}[Scale=MatchLowercase]

\addbibresource{references.bib}
\pagestyle{fancy}
\lhead{Utrecht University}
\cfoot{\thepage}
\rfoot{Curtis Chin Jen Sem}
\renewcommand{\headrulewidth}{0.5pt}
\renewcommand{\footrulewidth}{0.5pt}
\setminted[coq]{escapeinside=~~,mathescape,autogobble}
\renewcommand{\listingscaption}{Code snippet}

\begin{document}

\begin{titlepage}

\begin{figure}
   \begin{minipage}{0.48\textwidth}
   \begin{flushleft}
    %  \includegraphics[scale=0.5]{Images/UU_LOGO.png}
   \end{flushleft}
   \end{minipage}\hfill
   \begin{minipage}{0.48\textwidth}
   \begin{flushright}
    %  \includegraphics[scale=0.2]{Images/Logo.png}
   \end{flushright}
   \end{minipage}
\end{figure}

\thispagestyle{fancy}

\vspace{1in}

\center

\textsc{\large Master Thesis Proposal}

\vspace{0.5in}

\noindent\makebox[\linewidth]{\rule{\linewidth}{1.2pt}}
\textsc{\textbf{\large Formalized Proof of Automatic Differentiation in Coq}}
\noindent\makebox[\linewidth]{\rule{\linewidth}{1.2pt}}

\vspace{0.5in}

\begin{minipage}{0.48\textwidth}
    \begin{flushleft}
        \textit{Student:} \\
        Curtis Chin Jen Sem \\
        % crtschin@gmail.com
    \end{flushleft}
\end{minipage}
\begin{minipage}{0.48\textwidth}
    \begin{flushright}
    \textit{Supervisors:} \\
    Mathijs V\'{a}k\'{a}r \\
    Wouter Swierstra \\
    % Email
    % \end{flushright}
    % \begin{flushright}
    % Email
    \end{flushright}
\end{minipage}

\vspace{2in}

\textbf{\large Department of Information and Computing Science} \\

\today

\end{titlepage}

\newpage

\setcounter{page}{2}
\tableofcontents
\newpage

\section{Introduction}

AI and machine learning research has sparked a lot of new interest in recent times due to its many applications and ability to solve very difficult problems very quickly.
One of the principle techniques in practice is the use of the gradient descent algorithm, which takes an optimization problem and tries to find a solution by iteratively moving towards the optimum.

This is regularly done using a technique called automatic differentiation.
\todo{Name examples of recent papers}There has been a recent surge of interest in formulating languages for defining automatic differentiable functions.
This could have many benefits such as both applying many of the established high and low level optimizations known in programming languages research, ease defining functions for use in a gradient descent optimization through higher order functions and correctness through the use of a possible type system.

We aim to formalize an extendable proof of an implementation of automatic differentiation on a simply typed lambda calculus in the Coq proof assistant, opening up further possibilities for formally proving the correctness of more complex language features in the future.
Our formalization is based on a recent proof by Stanton Huot, and V\'{a}k\'{a}r\cite{huot2020correctness}.
They proved, using a denotational model of diffeological spaces, that their forward mode emulating macro is correct when applied to a simply typed lambda calculus with products, co-products and inductive types.

\todo{Fill in}

With this thesis we will aim for the following goals:
\begin{itemize}
  \item Contribute a formalized proof of forward-mode automatic differentiation in Coq.
  \item Extend the original proof with iteration and possibly recursion.
  \item Prove that well-known optimizations such as the partial evaluation, are correct with respect to automatic differentiation.
  \item Formulate the proofs such that it facilitates further extensions
\end{itemize}

% Ideally?:
% Contributing an extendable proof of a simply typed lambda calculus
% On which both the correctness of optimizations and language
% extensions could be proven
% Proof of optimizations maintaining correctness (partial evaluation)

As a notational convention, we will use specialized notation in the definitions themselves instead of declaring them separately, which Coq normally requires.

\section{Background}

\subsection{Automatic differentiation}

% http://jmlr.org/papers/volume18/17-468/17-468.pdf

One of the principal techniques used in machine learning is back propagation, which calculates the gradient of a function.
The idea being to use the gradient in the gradient descent algorithm\cite{Baydin2015AutomaticDI}.
Automatic is a generalization of backpropagation.
Automatic differentiation has a long and rich history, where its purpose is to calculate the derivative of a function, or in other words, calculate the derivative of function described by an arbitrary program.
So the semantics which one would normally expect in programming language is extended with relevant concepts such as derivative values and the chain rule.

Automatic or algorithmic differentiation is beneficial over other methods of automatically calculating the derivatives of functions such as numerical differentiation or symbolic differentiation due to its balance between speed and computational complexity.
There are two main modes of automatic differentiation which I will both discuss.
These are namely forward and reverse mode AD.
In this paper we will prove the semantics of a forward mode AD algorithm correct.
The algorithm being a very simple macro on the syntax of a simply typed lambda calculus.

\todo{Fill in}

\subsection{Denotational semantics}
% A formal semantics of programming language: An introduction

The notion of denotational semantics, created by Dana Scott and Christopher Strachey\cite{Scott1977}, tries to find underlying mathematical objects able to explain the properties of programming languages.
Their original search for a solution for lambda calculi led them to well-known concepts such as partial orderings and least fixed points.
In this model, programs are interpreted as partial functions and computation by taking fixpoint of such functions.
Non-termination on the other hand is symbolized by a value bottom that is lower in the ordering relation than any other element.
This search for an underlying mathematical foundation for languages is also known as domain theory.

In our specific case, we try to find a satisfactory model we can use to show that our implementation of forward mode automatic differentiation is correct when applied to a simply typed lambda calculus.
In the original pen and paper proof of automatic differentiation this thesis is based on, the mathematical models used were diffeological spaces, which are generalization of smooth manifolds.
For the purpose of this thesis, however, this was deemed excessive and much too difficult and time consuming to implement in a mathematically sound manner in Coq.
As such, we chose to make use of Coq's existing types as denotation and base the relation on the denotations instead of the syntactic structures.

\todo{Fill in}

\subsection{Coq}

Coq is a proof assistant created by Thierry Coquand as an implementation of his calculus of constructions type theory\cite{Coquand1988}.
In the 30 years since it has been released, research has contributed to extending the proof assistant with additional features such as inductive and co-inductive data types\cite{Coquand1990}, dependent pattern matching\cite{Sozeau2010} and advanced modular constructions for organizing colossal mathematical proofs\cite{Sozeau2008}\cite{Mahboubi2013}.

The core of this type theory is based on constructive logic and so many of the laws known in classical logic are not present.
Examples include the law of the excluded middle, $\forall A, A \vee \neg A$, or the law of functional extensionality, $\forall x, f(x) = g(x) \rightarrow f = g$.
They can, however, be safely added to Coq without making its logic inconsistent.
Due to its usefulness in proving propositions over functions, we will make use of the functional extensionality axiom in Coq.

\subsubsection{Language representation}

When defining a simply typed lambda calculus, there are two main representations.
The arguably simpler variant, known as an extrinsic representation, is traditionally the one introduced to new students learning Coq.
In the extensional representation, the terms themselves are untyped and typing judgments are defined separately.
The other approach, also called an intrinsic representation or strongly typed terms, makes use of just a single well-typed definition.
Ill-typed terms are made impossible by the type checker.
This representation, while beneficial in the proof load, however complicates much of the normal machinery involved in programming language theory.
One example is how one would define operations such as substitutions.

But even when choosing an intrinsic representation, the problem of variable binding persists.
Much meta-theoretical research has been done on possible approaches to this problem each with their own advantages and disadvantages.
The POPLmark challenge gives a comprehensive overview of each of the possibilities in various proof assistants\cite{Aydemir2005}.
An example of an approach is the nominal representation where every variable is named.
While this does follow the standard format used in regular mathematics, problems such as alpha-conversion and capture-avoidance arises.

\begin{listing}
  \begin{minted}{coq}
  Inductive ty : Type :=
    | ~()~ : ~ty~
    | ~\rightarrow~ : ~ty \rightarrow ty \rightarrow ty~.

  Inductive tm : Type :=
    | var : ~string \rightarrow tm ~
    | abs : ~string \rightarrow ty \rightarrow tm \rightarrow tm~
    | app : ~tm \rightarrow tm \rightarrow tm~.
  \end{minted}
  \caption{Simply typed \lambda-calculus using an extrinsic nominal representation.}
  \label{lst:nominal_stlc}
\end{listing}

The approach used in the rest of this thesis is an extension of the de-bruijn representation, which numbers variables relative to the binding lambda term.
In this representation the variables are referred to as de-bruijn indices.
A significant benefit of this representation is that the problems of capture avoidance and alpha equivalence are avoided.
As an alteration, instead of using numbers to represent the distance, indices within the typing context can be used to ensure that a variable is always well-scoped.
The specific formulation used in this thesis was fleshed out by Nick Benton, et. al. in \cite{Benton2011}, and subsequently used as one of the examples in the second POPLmark challenge which deals with logical relations\cite{poplmark_reloaded}.
While this does subvert the problems present in the nominal representation, it unfortunately does have problems of its own.
Variable substitutions are defined using two separate renaming and substitution operations.
Renaming is formulated as extending the typing context of variables, while substitution actually swaps the variables for terms.
Due to using indices from the context as variables, some lifting boilerplate is needed to manipulate contexts.

\begin{listing}
  \begin{minted}{coq}
  Inductive ~\tau \in \Gamma~ : Type :=
    | Top : ~\forall \Gamma \tau, \tau \in (\tau::\Gamma)~
    | Pop : ~\forall \Gamma \tau \sigma, \tau \in \Gamma \rightarrow \tau \in (\sigma::\Gamma)~.

  Inductive tm ~\Gamma \tau~ : Type :=
    | var : ~\forall \Gamma \tau, \tau \in \Gamma \rightarrow tm \Gamma \tau~
    | abs : ~\forall \Gamma \tau \sigma, tm (\sigma::\Gamma) \tau \rightarrow tm \Gamma (\sigma \rightarrow \tau)~
    | app : ~\forall \Gamma \tau \sigma, tm \Gamma (\sigma \rightarrow \tau) \rightarrow tm \Gamma \sigma \rightarrow tm \Gamma \tau~.
  \end{minted}
  \caption{Basis of a simply typed \lambda-calculus using a strongly typed intrinsic formulation.}
  \label{lst:strong_stlc}
\end{listing}
\todo{Extend example}

\subsubsection{Dependent types in Coq}

In Coq, one can normally write function definitions using either case-analysis as in other functional languages, or using Coq's tactics.
If proof terms are present in the function definition, however, it is customary to write it using tactics because of the otherwise cumbersome and verbose code needed to pattern-match on the arguments.
Writing definitions using tactics also has its limitations, the user is allowed definitions that later are uncovered to be unprovable, the definitions are opaque such that the standard `simpl` tactic which invokes Coq's reduction mechanic is not able to reduce the term.
This often requires the user to write and use proofs of the functions reducibility at its argument.
As an example, we will work through defining a length indexed list and a corresponding head function, which is well known to be partial.

Using the Coq keyword return, it is possible to let the return type of a match expression depend on the result of one of the type arguments.
This makes it possible to specify what the return type of the empty list should be.
In snippet \ref{lst:dt_ilist}, we use the unit type which contains just one inhabitant, \texttt{tt}.

\begin{listing}[h]
  \begin{minted}{coq}
  Inductive ilist : ~Type \rightarrow nat \rightarrow Type~ :=
    | nil : ~\forall A, ilist A 0~
    | cons : ~\forall A n, A \rightarrow ilist A n \rightarrow ilist A (S n)~

  Definition hd {A} n (ls : ilist A n) :=
    match ls in (ilist A n) return
      (match n with
      | O => unit
      | S _ => A end) with
    | nil => tt
    | cons h _ => h
  end.
  \end{minted}
  \caption{Definition of a length indexed list and hd using the return keyword, adapted from \cite{ChlipalaCPDT}.}
  \label{lst:dt_ilist}
\end{listing}

In \cite{Sozeau2006} and \cite{Sozeau2007} Sozeau introduces an extension to Coq via a new keyword \texttt{Program} which allows the use of case-analysis in more complex definitions.
To be more specific, it allows definitions to be specified separately from its accompanying proofs, possibly filling them in automatically if possible.
While this does improve on the previous situation, using the definitions in proofs can often be unwieldy due to the amount of boilerplate introduced.
This makes debugging error messages even harder than it already is in a proof assistant. This approach was used by Benton in his formulation of strongly typed terms.

Sozeau further improves on this in \cite{Sozeau2010} and \cite{Sozeau2019} by introducing a method for user-friendlier dependently typed programming in Coq as the \texttt{Equations} library.
This introduces `Agda` like dependent pattern matching with with-clauses.
It makes use of a notion called coverings where a set of equations should be an exhaustive covering of the signature of the function it defines.
There are two main ways to integrate this in a dependently typed environment, externally where it is integrated as high-level constructs in the pattern matching core as Agda does it, or internally by using the existing type theory and finding witnesses of the covering to prove the definition correct, which is the approach used by Sozeau.
This was invaluable when defining the substitution operators in the well-scoped well-typed de-bruijn representation discussed in the previous section.

\begin{listing}[h]
  \begin{minted}{coq}
  Equations hd {A n} (ls : ilist A n) (pf : n <> 0%nat) : A :=
  hd nil pf with pf eq_refl := { | x :=! x };
  hd (cons h n) _ := h.
  \end{minted}
  \caption{Definition of hd using Equations}
  \label{lst:dt_ilist_hd_equations}
\end{listing}

\todo{Fill in}

\subsection{Logical relations}

Logical relations, otherwise known as Tait's method, is technique often employed when proving programming language properties\cite{skorstengaard2019introduction}.
They are defined as relations over the types of the language.
There are two main ways they are used, namely as unary and binary relations.
Unary logical relations, also known as logical predicates, are predicates over single terms and are typically used to prove language characteristics such as type safety or strong normalization.
Binary logical relations on the other hand are used to prove program equivalences, usually in the context of denotational semantics.
A logical relations proof essentially consists of 2 steps.
The first usually states that well-typed terms are in the relation, while the second states that the property of interest follows from the relation.
It should be evident that the proof itself is usually routine compared to defining the relation itself.

The proof given in the paper this thesis is based on, is a logical relations proof on the denotation semantics using diffeological spaces as its domains\cite{huot2020correctness}.
A similar, independent proof of correctness was given in \cite{barthe2020versatility} using an syntactic relation.

\section{Preliminary Results}
\todo{Fill in}

\subsection{Preliminary Proof}
\todo{Fill in}

\subsection{Proof Extension}
\todo{Fill in}

\section{Timetable and Planning}
\todo{Fill in}

\subsection{Deadlines}
\todo{Fill in}

\subsection{Extensions}
\todo{Fill in}

\printbibliography
% \bibliographystyle{IEEEtran}
\end{document}